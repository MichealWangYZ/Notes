\documentclass[UTF8]{ctexart}

\usepackage{abstract}
\usepackage{enumerate}
\usepackage{amsmath, amssymb, amsthm}
\usepackage{cite}
\usepackage{graphicx}
\usepackage{subfigure}
\usepackage[colorlinks, linkcolor=blue, anchorcolor=black, citecolor=black]{hyperref}
\usepackage{physics}

\newcommand*{\argmax}{\mathrm{argmax}}
\newcommand*{\argmin}{\mathrm{argmin}}
\newcommand*{\reals}{\mathbb{R}}
\newcommand*{\otherwise}{\text{otherwise}}
\newcommand*{\range}{\mathrm{range}\;}

\title{方阵和有限维线性算子}

\begin{document}

\maketitle

\begin{abstract}
    分析有限维向量空间上的线性算子,以及它们的矩阵表示。
\end{abstract}

符号约定:设$U, V, W$是实数域或者复数域上的有限维向量空间,
$\vb*{u}_1, \vb*{u}_2, \ldots$、$\vb*{v}_1, \vb*{v}_2, \ldots$、$\vb*{w}_1, \vb*{w}_2, \ldots$
分别是$U, V, W$的一组基。
设$\dim U = \dim V = n$。
\newcommand*{\ubasis}{(\vb*{u}_1, \ldots, \vb*{u}_n)}
\newcommand*{\vbasis}{(\vb*{v}_1, \ldots, \vb*{v}_n)}

\section{矩阵表示}
设$\vb*{A}$是有限维线性算子$T \in \\mathcal{L}(U, V)$在基$\ubasis$和$\vbasis$下的矩阵表示,
也即$\vb*{A} = \\mathcal{M}(T, \ubasis, \vbasis)$。
这也等价于
\[
    T \vb*{u}_i = \sum_{j=1}^n \vb*{v}_j A_{ji}, \quad i = 1, \ldots, n.
\]

现在讨论算子间的关系和矩阵表示的关系的对应。
最简单的这种关系就是算子加法对应矩阵加法,算子数乘对应矩阵数乘,而算子复合对应矩阵乘法。


\section{内积空间上的算子}

\section{复空间上的算子}

\section{实空间上的算子}

\end{document}