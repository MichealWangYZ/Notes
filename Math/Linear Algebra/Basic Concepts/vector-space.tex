\hypertarget{sec:vector-space}{%
\section{向量空间}\label{sec:vector-space}}

对一组向量$\boldsymbol{v}_1, \ldots, \boldsymbol{v}_n \in V$,定义它们的\textbf{张成}$\mathrm{span}(\boldsymbol{v}_1, \ldots, \boldsymbol{v}_n)$为它们的所有线性组合构成的构成的集合。并且,我们定义空组$()$的张成为$\{\boldsymbol{0}\}$。

如果一组向量的张成等于某个向量空间,我们称这组向量是这个向量空间的一个张成组。一个向量组中的全部向量都属于某个向量空间,当且仅当,它们的张成就是这个向量空间。

可以证明,一个向量空间中一组向量的张成是包含这组向量的最小子空间。

如果一个向量空间能够由它的有限个元素张成,那么这个空间就是一个\textbf{有限维的向量空间}。否则,它是\textbf{无限维的}。无限维的向量空间允许长度没有上限的线性无关组。

有限维空间的子空间也是有限维的。事实上,设有有限维空间$V$和它的一个子空间$U$,若$\boldsymbol{v}_1, \ldots, \boldsymbol{v}_n$是$V$的一个张成组,那么一定可以从中找出一组能够张成$U$的向量。

\hypertarget{ux7ebfux6027ux76f8ux5173ux4e0eux7ebfux6027ux65e0ux5173}{%
\subsection{线性相关与线性无关}\label{ux7ebfux6027ux76f8ux5173ux4e0eux7ebfux6027ux65e0ux5173}}

一组向量是\textbf{线性相关的},当且仅当,存在一组不全为零的系数,使得它们按照这组系数的线性组合为零向量。反之,它们\textbf{线性无关}。

显然,一组向量线性相关,当且仅当,它们当中至少有一个向量可以写成其他向量的线性组合。反之,如果$\boldsymbol{v}_1, \ldots \boldsymbol{v}_n$线性无关,而$\boldsymbol{v}_1, \ldots \boldsymbol{v}_n, \boldsymbol{u}_1, \ldots, \boldsymbol{u}_m$线性相关,那么$\boldsymbol{u}_1, \ldots, \boldsymbol{u}_m$中至少有一个可以写成$\boldsymbol{v}_1, \ldots \boldsymbol{v}_n$的线性组合。从而,如果一组向量线性相关,那么一定可以从它们当中去掉一个向量,使剩下的向量的张成和原向量组的张成一样。

线性无关向量组的长度小于等于张成向量组的长度。

考虑到一组全部为零的系数肯定能够让这些向量的线性组合为零,我们有:一系列向量$\boldsymbol{v}_1, \ldots, \boldsymbol{v}_n$是线性无关的,当且仅当,若$a_1 \boldsymbol{v}_1 + \ldots + a_n \boldsymbol{v}_n = \boldsymbol{0}$,则$a_1 = \ldots = a_n=0$。

这样一来容易看出,如果$\boldsymbol{v}_1, \ldots, \boldsymbol{v}_n$是线性无关的,那么,从$a_1 \boldsymbol{v}_1 + \ldots + a_n \boldsymbol{v}_n = a_1' \boldsymbol{v}_1 + \ldots + a_n' \boldsymbol{v}_n$就可以得出$a_1 = a_1', \ldots, a_n = a_n'$。因此,如果一个向量能够被线性无关的向量组$\boldsymbol{v}_1, \ldots, \boldsymbol{v}_n$线性表示,那么表示方法一定唯一。

事实上,我们还可以把上面这个命题反过来叙述。设$\boldsymbol{u} \in \mathrm{span}(\boldsymbol{v}_1, \ldots, \boldsymbol{v}_n)$,则有:
$\boldsymbol{v}_1, \ldots, \boldsymbol{v}_n$线性无关,当且仅当,$\boldsymbol{u}$使用$\boldsymbol{v}_1, \ldots, \boldsymbol{v}_n$的线性表示是唯一的。

另有结论:如果$\boldsymbol{v}_1, \ldots \boldsymbol{v}_n \in V$且$V$中每一个元素都可以写成它们的线性组合,那么$V = \mathrm{span} (\boldsymbol{v}_1, \ldots \boldsymbol{v}_n)$。特别的,如果$U,V$是$W$的子空间且$W$中的每一个元素都可以写成$U,V$中分别取出的一个元素之和,那么$W=U+V$。

\hypertarget{ux57fa}{%
\subsection{基}\label{ux57fa}}

一个有限维向量空间中的一组向量是这个空间的\textbf{基},当且仅当,向量空间中的每一个向量都可以写成这组向量的线性组合。每一个有限维向量空间都有基。

向量空间的任何一个张成组在删去一些向量之后都可以化简成一个基。每一个线性无关向量组在扩充了一些向量之后都可以成为一个基。

设$V$是一个有限维向量空间,$U$是一个$V$的子空间,那么存在另一个子空间$W$使$V=U \oplus W$。此时,$U$的基可以划分成不相交的两组,一组是$U$的基,一组是$W$的基。

给定$V$的一组基$\boldsymbol{v}_1, \ldots, \boldsymbol{v}_n$,$V$的每个子空间都可以以这个向量组的部分或全部为基。反过来,给定了一组基,也就完全确定了一个向量空间。因此$V$的子空间和$\boldsymbol{v}_1, \ldots, \boldsymbol{v}_n$的子列一一对应。

$\boldsymbol{e}_1, \ldots, \boldsymbol{e}_n$是$V$的基,当且仅当$\boldsymbol{e}_1, \ldots, \boldsymbol{e}_n$张成$V$且$\dim V = n$。

\hypertarget{ux7ef4ux6570}{%
\subsection{维数}\label{ux7ef4ux6570}}

有限维向量空间任意两个基的长度都相同。我们于是将这个长度定义为\textbf{维数},即$\dim V$。一个向量空间的维数一定大于等于它的子空间的维数。

有限维向量空间$V$中每一个长度等于$\dim V$的张成向量组或者线性无关向量组都是它的一个基。张成向量组的长度大于等于维数,而线性无关向量组的长度小于等于维数。

\[
\dim (U_1 + U_2) = \dim U_1 + \dim U_2 - \dim (U_1 \cap U_2).
\]

\[
\dim (U_1 + \ldots + U_n) \leq \dim U_1 + \ldots + \dim U_n
\]

设$V=V_1 + \ldots + V_n$,则下面的说法是等价的: -
$V = V_1 \oplus \ldots \oplus V_n$ -
$\dim V = \dim V_1 + \ldots + \dim V_n$

两个有限维向量空间同构,当且仅当它们的维数相等。如果两个向量空间同构且两者有非空交集,那么它们一定相等。

\hypertarget{ux5411ux91cfux7ec4}{%
\subsection{向量组}\label{ux5411ux91cfux7ec4}}

两个向量组\textbf{等价},当且仅当,它们张成同样的向量空间,也即,它们可以相互线性表示。

设向量组$S$是向量组$R$的一部分。称$S$是$R$的一个\textbf{极大无关组},当且仅当,$S$本身线性无关,且把不属于$S$而属于$R$的任意一个向量加入$S$之后得到的向量组线性相关。

容易看出,如果$S$是$R$的一个极大无关组,那么$R$中不属于$S$的向量全部可以使用$S$线性表示,
因此,$S$构成了$\mathrm{span} R$的一组基。
因此同一个向量组的所有极大无关组的长度都相同。我们记这个长度为$\mathrm{rank}\; R$,称其为$R$的\textbf{秩}。
但是如果这个向量组的极大无关组当中有零向量(也就是说,这个向量组全为零,并且极大无关组就是单独一个零),那么规定其秩为0。
这是为了保证我们始终有$\mathrm{rank} \; R = \dim \mathrm{span} R$。

两个向量组$S_1, S_2$等价,当且仅当,$\mathrm{rank}S_1 = \mathrm{rank}S_2 = \mathrm{rank}S_3$,其中$S_3$指的是将$S_1, S_2$拼接起来得到的向量组。

一个向量组$S$总是有一个只有$\mathrm{rank}S$个非零向量的等价向量组,并且我们可以通过一系列的线性变换从$S$变换过去。

对一个线性无关的向量组,将其中每一个向量都加上除了这个向量以外的其它向量的某个线性组合之后得到的向量组线性无关;将其中每一个向量乘上一个非零倍数之后得到的向量组线性无关。

推论:主对角元全部非零的矩阵必可逆。

\hypertarget{ux6709ux9650ux7ef4ux5411ux91cfux7a7aux95f4ux7684ux76f4ux548cux5206ux89e3}{%
\subsection{有限维向量空间的直和分解}\label{ux6709ux9650ux7ef4ux5411ux91cfux7a7aux95f4ux7684ux76f4ux548cux5206ux89e3}}

若$\boldsymbol{u}_1, \ldots, \boldsymbol{u}_m$,$\boldsymbol{v}_1, \ldots, \boldsymbol{u}_n$这两组向量中的每一个都不能使用另一组向量线性表示,那么
\[
\mathrm{span}(\boldsymbol{u}_1, \ldots, \boldsymbol{u}_m, \boldsymbol{v}_1, \ldots, \boldsymbol{u}_n) = \mathrm{span}(\boldsymbol{u}_1, \ldots, \boldsymbol{u}_m) \oplus \mathrm{span}(\boldsymbol{v}_1, \ldots, \boldsymbol{u}_n).
\]
并且,设$\boldsymbol{w} \in \mathrm{span}(\boldsymbol{u}_1, \ldots, \boldsymbol{u}_m, \boldsymbol{v}_1, \ldots, \boldsymbol{v}_n), \boldsymbol{w} = \boldsymbol{u} + \boldsymbol{v}, \boldsymbol{u} \in \mathrm{span}(\boldsymbol{u}_1, \ldots, \boldsymbol{u}_m), v \in \mathrm{span}(\boldsymbol{v}_1, \ldots, \boldsymbol{v}_n)$,则它在这一组基下的表示的前$m$个元素是$\boldsymbol{u}$在$\mathrm{span}(\boldsymbol{u}_1, \ldots, \boldsymbol{u}_m)$下的表示,后$n$个元素是$\boldsymbol{v}$在$\mathrm{span}(\boldsymbol{v}_1, \ldots, \boldsymbol{v}_n)$的表示。

设$V=U\oplus W$,$\boldsymbol{v}_1, \ldots, \boldsymbol{v}_n \in V, \boldsymbol{v}_i = \boldsymbol{u}_i + \boldsymbol{w}_i, \boldsymbol{u}_i \in U, \boldsymbol{w}_i \in W$,则$\boldsymbol{v}_1, \ldots, \boldsymbol{v}_n$线性相关时$\boldsymbol{u}_1, \boldsymbol{u}_n$线性相关。

设$U$是$V$的子空间,那么$V$的一组基一定有一个子列能够作为$U$的基,而剩下的向量则可以张成另一个空间$W$,容易看出$V=U \oplus W$。这样对给定的$\boldsymbol{v} \in V$,有且仅有一组$\boldsymbol{u}\in U, \boldsymbol{v} \in W$满足$\boldsymbol{v} = \boldsymbol{u} + \boldsymbol{w}$。于是我们将$U$分解成了两个子空间。称$\boldsymbol{u}$为$\boldsymbol{v}$在$V$下的投影。

一组向量$\boldsymbol{v}_1, \ldots, \boldsymbol{v}_n \in V$在$V$的任何子空间下的投影都线性相关,当且仅当,$\boldsymbol{v}_1, \ldots, \boldsymbol{v}_n \in V$线性相关。
