\section{算子的特征向量和特征值}\label{sec:eigenvalue}

设$T\in \mathcal{L}(V)$,$U$是$V$的子空间。我们称$U$在$T$下不变,当且仅当,我们有
\[
\forall \boldsymbol{u} \in U, Tu \in U.
\]

$\mathrm{null} \; T, \mathrm{range} \; T$在$T$下不变。

我们讨论一类特殊的不变子空间。设$\boldsymbol{u} \in V$且$\boldsymbol{u}$不是零向量,下面的空间$\{a\boldsymbol{u} | a \in \mathbb{F}\}$构成了$V$的一个子空间,而且是一个一维的子空间。如果$U$是一个一维不变子空间,那么$\exists \lambda \in \mathbb{F}, T\boldsymbol{u}=\lambda \boldsymbol{u}$。

于是,我们称 \[
T\boldsymbol{u} = \lambda \boldsymbol{u} 
\]
中的$\lambda$为\textbf{特征值},当且仅当$\boldsymbol{u}$非零的时候。在$\boldsymbol{u}$非零的时候,称$\boldsymbol{u}$为$\lambda$对应的\textbf{特征向量}。显然,$T$有一维不变子空间,当且仅当,它有特征值。

刚才的方程等价于$(T-\lambda I)\boldsymbol{u}=\boldsymbol{0}$。于是我们得出:下面几个结论是等价的:
\begin{itemize}
    \item $T$有特征值 
    \item $(T-\lambda I)$不是单射 
    \item $(T-\lambda I)$不是满射
    \item $(T-\lambda I)$不可逆
\end{itemize}

$T$相应于$\lambda$的特征向量的集合是$\mathrm{null} \; (T - \lambda I)$,这也是$V$的子空间。

$V$的互不相同的特征值对应的特征向量彼此线性无关,并且$V$最多有$\dim V$个不同的特征值。

下面几个结论等价:

\begin{itemize}
    \item $T$关于基$\boldsymbol{v}_1, \ldots, \boldsymbol{v}_n$的矩阵是上三角的
    \item $\forall k \in 1..n, T\boldsymbol{v}_k \in \mathrm{span}(\boldsymbol{v}_1, \ldots, \boldsymbol{v}_k)$
    \item $\forall k \in 1.. n$,$\mathrm{span}(\boldsymbol{v}_1, \ldots, \boldsymbol{v}_k)$在$T$下是不变的
\end{itemize}

算子$T \in \mathcal{L}(V)$可以写成对角矩阵,当且仅当,$T$的特征向量构成$V$的一组基,当且仅当,$V$可以分解成$T$的一系列不变一维子空间的直和。以$T$的特征向量为基的时候,$T$的矩阵表示正好就是一个对角矩阵,并且对角线上的各个值就是特征值。

使用特征多项式什么的可以定义特征值的\textbf{重数}。可以证明$\mathrm{null}(T-\lambda I)$的维数小于$\lambda$的重数。事实上,$T$可以对角化,当且仅当,它的每一个特征值的重数都等于$\mathrm{null}(T-\lambda I)$的维数。

如果$T$的不同的特征值数量就是$\dim V$,那么$T$一定可以写成对角矩阵。但是有特征值重复或者缺漏不代表$T$一定没有对角矩阵。

如果特征值含有零,那么算子一定不可逆。

\hypertarget{ux5b9eux5411ux91cfux7a7aux95f4ux4e0b}{%
\subsection{实向量空间下}\label{ux5b9eux5411ux91cfux7a7aux95f4ux4e0b}}

\hypertarget{ux590dux5411ux91cfux7a7aux95f4ux4e0b}{%
\subsection{复向量空间下}\label{ux590dux5411ux91cfux7a7aux95f4ux4e0b}}

接下来取$\mathbb{F} = \mathbb{C}$。

算子$T$对应一个方阵。

事实上,$T$一定会关于某一组基具有上三角矩阵。

使用上三角矩阵有助于分析算子是不是可逆。一个算子可逆,当且仅当它的上三角矩阵对角线上的元素都不是0。

上三角矩阵对角线上的元素是它对应的算子的全部本征值。
