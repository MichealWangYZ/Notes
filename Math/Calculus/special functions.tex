\documentclass[UTF8]{ctexart}

\usepackage{geometry}
\usepackage{titling}
\usepackage{titlesec}
\usepackage{abstract}
\usepackage{footnote}
\usepackage{enumerate}
\usepackage{amsmath, amssymb}
\usepackage{cite}
\usepackage{graphicx}
\usepackage{subfigure}
\usepackage{physics}
\usepackage[colorlinks, linkcolor=black, anchorcolor=black, citecolor=black]{hyperref}

\geometry{left=2.5cm,right=2.5cm,top=2.5cm,bottom=2.5cm}
\setlength{\droptitle}{-5em}
\preauthor{\vspace{-10pt}\begin{center}}
\postauthor{\par\end{center}}
\newcommand*{\res}{\mathrm{res}\;}
\newcommand*{\natnums}{\mathbb{N}}
\newcommand*{\integers}{\mathbb{Z}}
\newcommand*{\reals}{\mathbb{R}}
\newcommand*{\complexes}{\mathbb{C}}
\newcommand*{\taylor}[1]{\sum_{#1 = 0}^\infty}
\newcommand*{\taylorfrom}[2]{\sum{#1=#2}^\infty}
\newcommand*{\laurent}[1]{\sum_{#1=-\infty}^\infty}
\DeclareMathOperator{\gammafunc}{\Gamma}
\DeclareMathOperator{\betafunc}{B}
\DeclareMathOperator{\legpoly}{P}
\newcommand*{\ii}{\mathrm{i}}
\newcommand*{\ee}{\mathrm{e}}

\title{特殊函数、特殊积分以及一些具体计算}
\author{wujinq}

\begin{document}

\maketitle

\begin{abstract}
    关于特殊函数、特别的积分,以及一些具体计算结果的记录。没有条理的东西可以放在这里。
\end{abstract}

以下如无特殊说明,$n, k \in \integers$,$x, y \in \reals$,$z \in \complexes$。

\section{延拓得到的函数}

\subsection{Gamma函数与阶乘}

在实空间上定义\textbf{Gamma函数}
\begin{equation}
    \gammafunc(x) = \int_0^\infty \ee^{-t} t^{x-1} \dd t.
\end{equation}
这个定义可以自然地延拓到复平面上:
\begin{equation}
    \gammafunc(z) = \int_0^\infty \ee^{-t} t^{z-1} \dd t.
    \label{eq:gamma-first-def}
\end{equation}
被积函数在一般的情况下是多值的,而我们简单地要求在正实轴上被积函数取其实数值,也就是要求在正实轴上$\arg t = 0$。
可以证明,$\Re z > 0$上\eqref{eq:gamma-first-def}是解析的。采用下面的递归定义
\[
\begin{split}
    \gammafunc(z) = \frac{\gammafunc(z+1)}{z}=\frac{\gammafunc(z+n+1)}{z(z+1)\cdots(z+n)}, \\
    \quad -(n+1) < \Re z \leq -n, z \neq 0, -1, \ldots, -n
\end{split}
\]
可以将函数延拓到除了可数个点以外的整个复平面上。

容易验证,此时$\gammafunc$除了单极点$0, -1, -2, \ldots$以外处处解析。因此它是复平面上的亚纯函数。
此外还有下面的性质:
\begin{itemize}
    \item Gamma函数是阶乘的推广:
    \begin{equation}
        \gammafunc(n+1) = n!, \quad n \in \natnums
    \end{equation}
    进一步,
    \begin{equation}
        (z+m) (z+m+1) \cdots (z+n) = \frac{\gammafunc(z+n+1)}{\gammafunc(z+m)}
    \end{equation}
    从而
    \begin{equation}
        \begin{aligned}
            (z+2m)(z+2m+2) \cdots (z+2n) &= 2^{n-m+1} \frac{\gammafunc(z/2+n+1)}{\gammafunc(z/2+m)} \\
            (z+2m-1)(z+2m+1) \cdots (z+2n-1) &= \frac{\gammafunc(z+2n) \gammafunc(z/2 + m)}{2^{n-m} \gammafunc(z+2m-1) \gammafunc(z/2+n)}
        \end{aligned}
    \end{equation}
    \item 设$z$不是整数则
    \begin{equation}
        \gammafunc(z) \gammafunc(1-z) = \frac{\pi}{\sin \pi z}
    \end{equation}
    特别的$\gammafunc(1/2)=\sqrt{\pi}$
    \item 设$z \neq 0, -\frac{1}{2}, -1, \ldots$,则
    \begin{equation}
        \gammafunc(2z) = \frac{2^{2z-1}}{\sqrt{\pi}} \gammafunc(z) \gammafunc(z + \frac{1}{2})
    \end{equation}
\end{itemize}

\subsection{B函数}

在$x>0, y>0$上可以定义
\begin{equation}
    \betafunc(x, y) = \int_0^1 t^{(x-1)} (1-t)^{(y-1)} \dd t
\end{equation}
同样可以容易地在$\Re p>0, \Re q>0$做延拓
\begin{equation}
    \betafunc(p, q) = \int_0^1 t^{(p-1)} (1-t)^{(q-1)} \dd t
\end{equation}
被积函数具有多值性。还是要求被积函数在正实轴上取值和实函数形式一致,也就是$\arg t = \arg (1-t) = 0$。
可以证明,在$\Re p>0, \Re q>0$上
\begin{equation}
    \betafunc(p, q) = \frac{\gammafunc(p) \gammafunc(q)}{\gammafunc(p+q)}
    \label{eq:relation-beta-gamma}
\end{equation}
由于$\gammafunc$已经几乎被延拓到了整个复平面,可以使用\eqref{eq:relation-beta-gamma}将$\betafunc$做延拓,最后得到的函数在$p$平面上的奇点为$0, -1, -2, \ldots$,在$q$平面上的奇点$0, -1, -2, \ldots$。

\end{document}