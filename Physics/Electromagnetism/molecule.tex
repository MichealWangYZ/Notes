\documentclass[UTF8, a4paper]{ctexart}

\usepackage{geometry}
\usepackage{titling}
\usepackage{titlesec}
\usepackage{paralist}
\usepackage{footnote}
\usepackage{enumerate}
\usepackage{amsmath, amssymb, amsthm}
\usepackage{cite}
\usepackage{graphicx}
\usepackage{subfigure}
\usepackage{physics}
\usepackage[colorlinks, linkcolor=black, anchorcolor=black, citecolor=black]{hyperref}

\geometry{left=3.28cm,right=3.28cm,top=2.54cm,bottom=2.54cm}
\titlespacing{\paragraph}{0pt}{1pt}{10pt}[20pt]
\setlength{\droptitle}{-5em}
\preauthor{\vspace{-10pt}\begin{center}}
\postauthor{\par\end{center}}

\newcommand*{\ee}{\mathrm{e}}
\newcommand*{\ii}{\mathrm{i}}
\newcommand*{\st}{\quad \text{s.t.} \quad}
\newcommand*{\const}{\mathrm{const}}
\newcommand*{\natnums}{\mathbb{N}}
\newcommand*{\reals}{\mathbb{R}}
\newcommand*{\complexes}{\mathbb{C}}
\DeclareMathOperator{\timeorder}{T}
\newcommand*{\ogroup}[1]{\mathrm{O}(#1)}
\newcommand*{\sogroup}[1]{\mathrm{SO}(#1)}
\DeclareMathOperator{\legpoly}{P}

\title{分子光学}
\author{吴何友}

\begin{document}

\maketitle

\section{光场和非相对论性粒子的相互作用}

\subsection{哈密顿量}

\subsubsection{经典光场}

考虑与电磁场发生相互作用的粒子。
我们假定粒子做低速运动,从而不需要使用相对论性的理论描述粒子;另一方面,电磁场足够强以至于难以看到单光子效应,而又足够弱以至于能量不至于强到需要考虑量子电动力学的圈图修正,这样就可以使用经典电动力学描述整个系统。
粒子轨道部分的哈密顿量是以下保证局部$U(1)$规范对称性的极小耦合:
\begin{equation}
    \hat{H}_\text{orbit} = \frac{1}{2m} (\hat{\vb*{p}} - q \vb*{A})^2 + q \phi,
\end{equation}
自旋-磁场相互作用还会引入以下哈密顿量:
\begin{equation}
    \hat{H}_\text{spin} = - \frac{q}{m} \hat{\vb*{S}} \cdot \vb*{B},
\end{equation}
而场的哈密顿量是
\begin{equation}
    \hat{H}_\text{field} = \frac{\epsilon_0}{2} \int \dd[3]{\vb*{r}} (\vb*{E}^2 + c^2 \vb*{B}^2),
\end{equation}
则体系的总哈密顿量
\begin{equation}
    \hat{H} = \sum_i \left( \frac{1}{2m_i} (\hat{\vb*{p}_i} - q_i \vb*{A})^2 + q_i \phi \right) + \hat{H}_\text{field} + \hat{H}_\text{int} + \hat{H}_\text{ext},
\end{equation}
其中$\hat{H}_\text{int}$和$\hat{H}_\text{ext}$分别表示粒子间相互作用和外加势场。

\subsubsection{光场的量子化}

\section{经典弱场极限}

\subsection{弱场极限的哈密顿量}

考虑单粒子和光场的相互作用的哈密顿量。假定场较弱,则可以略去$\vb*{A}$的高阶项,从而得到坐标表象下带电粒子和电磁场发生相互作用的哈密顿量:
\[
    \hat{H} = - \frac{\hbar^2 \laplacian}{2m} + \frac{\ii \hbar q}{m} \vb*{A} \cdot \grad + \frac{\ii \hbar q}{2 m} \div\vb*{A} + q \phi - \frac{q}{m} \hat{\vb*{S}} \cdot \vb*{B}.
\]
第一项就是粒子动能;可以通过适当选取规范,让第三、四项消失,于是我们就得到弱场下带电粒子和经典光场的相互作用哈密顿量:
\begin{equation}
    \hat{H}_\text{light} = \frac{\ii \hbar q}{m c} \vb*{A} \cdot \grad - \frac{q}{m} \hat{\vb*{S}} \cdot \vb*{B} = \underbrace{- \frac{q}{m} \hat{\vb*{p}} \cdot \vb*{A}}_{\hat{H}_\text{1}} \underbrace{- \frac{q}{m} \hat{\vb*{S}} \cdot \vb*{B}}_{\hat{H}_2}.
\end{equation}
实际上,磁场对自旋的取向作用$\hat{H}_2$是很弱的。设电磁波波长的尺度为$\lambda$,则
\[
    \vb*{B} = \curl{\vb*{A}} \sim \frac{A}{\lambda},
\]
电子的活动范围的尺度和原子半径$a_0$同阶,由不确定性关系,
\[
    p a_0 \sim \hbar.
\]
于是
\[
    \frac{H_2}{H_1} \sim \frac{\hbar \frac{A}{\lambda}}{\frac{\hbar}{a_0} A} = \frac{a_0}{\lambda}.
\]
波长通常在几百纳米级别,而原子半径在纳米级别以下,从而$\hat{H}_1$远大于$\hat{H}_2$。

下面给出了一种更加接近静电势多级展开的方法。首先取适当的规范让矢势消失,于是就得到
\[
    \hat{H} = - \frac{\hbar^2 \laplacian}{2m} + q \phi - \frac{q}{m} \hat{\vb*{S}} \cdot \vb*{B},
\]
粒子的运动高度定域,于是可以将坐标系原点选取在粒子活动区域的“中心”,做多极展开
\[
    \phi(\vb*{r}) = \phi(0) + \vb*{r} \cdot \grad{\phi} + \frac{1}{2} \vb*{r} \vb*{r} : \grad{\grad{\phi}} + \cdots,
\]
第一项是一个无关紧要的能量零点,第二项是电偶极辐射,等等。
如通常所做的那样定义偶极矩
\begin{equation}
    \hat{\vb*{d}} = q \hat{\vb*{r}},
\end{equation}
这样就有
\begin{equation}
    \hat{H} = - \frac{\hbar^2 \laplacian}{2m} - \hat{\vb*{d}} \cdot \vb*{E} + \cdots - \frac{q}{m} \hat{\vb*{S}} \cdot \vb*{B}.
\end{equation}

\subsection{偶极辐射}



\end{document}