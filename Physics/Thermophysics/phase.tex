\documentclass[hyperref, UTF8, a4paper]{ctexart}

\usepackage{geometry}
\usepackage{titling}
\usepackage{titlesec}
\usepackage{paralist}
\usepackage{footnote}
\usepackage{enumerate}
\usepackage{amsmath, amssymb, amsthm}
\usepackage{cite}
\usepackage{graphicx}
\usepackage{subfigure}
\usepackage{physics}
\usepackage{tikz}
\usepackage[colorlinks, linkcolor=black, anchorcolor=black, citecolor=black]{hyperref}
\usepackage{prettyref}

\geometry{left=3.18cm,right=3.18cm,top=2.54cm,bottom=2.54cm}
\titlespacing{\paragraph}{0pt}{1pt}{10pt}[20pt]
\setlength{\droptitle}{-5em}
\preauthor{\vspace{-10pt}\begin{center}}
\postauthor{\par\end{center}}

\DeclareMathOperator{\timeorder}{T}
\DeclareMathOperator{\diag}{diag}
\newcommand*{\ii}{\mathrm{i}}
\newcommand*{\ee}{\mathrm{e}}
\newcommand*{\const}{\mathrm{const}}
\newcommand*{\comment}{\paragraph{注记}}

\newrefformat{sec}{第\ref{#1}节}
\newrefformat{note}{注\ref{#1}}
\renewcommand{\autoref}{\prettyref}

\newenvironment{bigcase}{\left\{\quad\begin{aligned}}{\end{aligned}\right.}

\title{相变}
\author{吴何友}

\begin{document}

\maketitle

\section{相变的热力学}

\subsection{一级相变的两相共存曲线}

考虑一个有两个可能的相的系统。
在两相共存曲线上,设两个相的化学势分别为$\mu_A$和$\mu_B$。两相平衡意味着
\begin{equation}
    \mu_A = \mu_B,
    \label{eq:equality-of-chemical-potential}
\end{equation}
否则将会出现两个相之间的粒子数转移。化学势相等马上又意味着方程
\[
    \pdv{G}{N_1} = 0
\]
有无穷多解,换而言之,固定$p, T$,物质在两相间怎么分配都是有可能的。一般来说,两个相单位分子占据的体积可以不同(例如水蒸气和液态水),因此这又意味着系统体积
\[
    V = V_A + V_B
\]
可以任意变化而不影响$p$和$T$(但会影响内能或者吉布斯自由能)。
换而言之,此时完整描述系统状态需要三个参量而不只是两个,因为$V$不能用$p, T$确定。
显然,在两相共存曲线的每一点都有\eqref{eq:equality-of-chemical-potential}成立,于是
\[
    \pdv{\mu_A}{p} \dd{p} + \pdv{\mu_A}{T} \dd{T} = \pdv{\mu_B}{p} \dd{p} + \pdv{\mu_B}{T} \dd{T}.
\]
由于
\[
    \pdv{\mu}{p} = v, \quad \pdv{\mu}{T} = -s,
\]
其中$v$和$s$分别表示单位粒子数的体积和熵,我们有
\[
    \dv{p}{T} = \frac{s_A - s_B}{v_A - v_B}.
\]
设$\lambda$为\textbf{每摩尔相变潜热},即单位粒子数的$B$相物质相变到$A$相需要吸收的热量,则
\[
    T(s_A - s_B) = \lambda,
\]
这个式子成立是因为相变发生时若$p, V$固定则温度也固定(否则就不会是两相共存曲线了),于是相变是等温过程。
这样就有
\begin{equation}
    \dv{p}{T} = \frac{\lambda}{T(v_A - v_B)} = \frac{L}{T(V_A - V_B)}.
\end{equation}
其中$L$指的是粒子数确定的一个系统的相变潜热。这就是\textbf{克拉珀龙方程}。
从克拉珀龙方程出发,可以发现出现两相共存的区域在$p-T$图上呈现为一条线而不是一个平面,称为\textbf{两相共存曲线}。
此外可以注意到相变时系统体积增大还是减小决定了两相共存曲线的斜率的正负。

\subsection{范德瓦尔斯方程中的相变}

考虑描述非理想气体的范德瓦尔斯方程
\begin{equation}
    (V-b)(p+\frac{a}{V^2}) = RT.
    \label{eq:van-de-waars-eq}
\end{equation}
这个方程考虑到了分子之间的相互作用,假定分子是彼此吸引的硬球(也即,当分子间距是分子直径时就会有强烈的排斥),这样分子能够在其中移动的体积必须扣除所有分子自身的体积,同时压强需要加上一个表示分子间相互作用的修正。
具体$a$和$b$是什么并不重要,本节只是要讨论\eqref{eq:van-de-waars-eq}展现出的相变行为。

表面上,随着$T$的变动,\eqref{eq:van-de-waars-eq}总是光滑的,似乎没有什么能够触发相变;其次液体和气体的$a$和$b$理应相同(因为是同一种物质),那么两者都应该遵循\eqref{eq:van-de-waars-eq},但这样就不能区分液体和气体。
实际上,任何一个描述气体和液体的模型都会遇到上述疑难。但这些疑难实际上只是假象。
\eqref{eq:van-de-waars-eq}的导出建立在液体/气体空间性质均一的基础上,但这个假设真的总是成立吗?

当$T$降低到一定程度时,随着$V$增大,$p$先减小后增大再减小。
$p$增大而$V$增大(即负压缩率)是非常奇特的现象,因为具有这种性质的系统是非常不稳定的:让$V$发生一个非常微小的增大,$p$就增大,然后$V$进一步增大。
换而言之,$T$足够低时\eqref{eq:van-de-waars-eq}中有一段$p$-$V$曲线给出的状态是不稳定的。
仅有的可能是,在这一段\eqref{eq:van-de-waars-eq}其实是一个非物理解。\eqref{eq:van-de-waars-eq}可以看成气体的低能有效理论,在负压缩率这一段它是不稳定不动点,会流到两个稳定不动点上,即出现分叉现象。
因此负压缩率段应该发生了相变,出现了气液共存(从而破坏了范德瓦尔斯方程隐含的系统空间均一的性质)。

在这样的思路下,可以把气液相变的临界点计算出来。当温度高于临界点时\eqref{eq:van-de-waars-eq}全程有效,当温度低于临界点时\eqref{eq:van-de-waars-eq}会有负压缩率段,即会有非物理解。
出现负压缩率就意味着$p$-$V$曲线出现拐点。根据\eqref{eq:van-de-waars-eq}的性质,应当有两个拐点。
这样,临界点就是两个拐点融合成一个时的温度。
两个拐点融合成一个意味着拐点处同时有
\[
    \pdv{p}{V} = 0, \quad \pdv[2]{p}{V} = 0.
\]
求解该方程,得到临界点的各个参数:
\begin{equation}
    V_\text{c} = 3b, \quad p_\text{c} = \frac{a}{27b^2}, \quad RT_\text{c} = \frac{8a}{27b}.
\end{equation}

如前所述气液共存段一旦给定了$T$,$p$就是确定的。现在我们尝试从范德瓦尔斯方程出发计算这个$p$。
设某个温度$T$下,增大$V$,在点1处相变开始,点2处相变结束。显然相变期间压强为
\[
    p^\text{trans} = p_1 = p_2.
\]
$p_1, V_1$和$p_2, V_2$均服从范德瓦尔斯方程。记$p^\text{vdw}$为范德瓦尔斯方程给出的(非物理的)压强。
体系自由能满足
\[
    p = - \eval{\pdv{F}{V}}_T,
\]
因此$T$不变时就有
\[
    \dd{F} = - p \dd{V}.
\]
点1和点2被$p$-$V$曲线\eqref{eq:van-de-waars-eq}连接起来,因此
\[
    F_2 - F_1 = - \int_1^2 p^\text{vdw} \dd{V},
\]
而实际的过程又满足
\[
    F_2 - F_1 = - p^\text{trans} (V_2 - V_1),
\]
于是就有
\begin{equation}
    p^\text{trans} (V_2 - V_1) = \int_1^2 p^\text{vdw} \dd{V}.
    \label{eq:maxwell-construction}
\end{equation}
\eqref{eq:maxwell-construction}意味着相变点1和2是直线$p=p^\text{trans}$与范德瓦尔斯方程绘制的曲线的三个交点的最左边和最右边两个,且直线$p=p^\text{trans}$与范德瓦尔斯方程围成的两块闭合面积相等。

分别以$V_\text{c}$,$p_\text{c}$,$T_\text{c}$为特征尺度对$V$,$p$,$T$做归一化,我们有
\begin{equation}
    \left(\bar{V} - \frac{1}{3}\right) \left(\bar{p} + \frac{3}{\bar{V}^2}\right) = \frac{8}{3} \bar{T}. 
\end{equation}
此方程和实验结果吻合得很好,除了在临界点附近。这是因为范德瓦尔斯方程实际上是一个平均场理论,在临界点附近(涨落很强)就不适用了。

\end{document}