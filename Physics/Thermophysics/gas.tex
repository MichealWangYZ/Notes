\documentclass[hyperref, UTF8, a4paper]{ctexart}

\usepackage{geometry}
\usepackage{titling}
\usepackage{titlesec}
\usepackage{paralist}
\usepackage{footnote}
\usepackage{enumerate}
\usepackage{amsmath, amssymb, amsthm}
\usepackage{cite}
\usepackage{graphicx}
\usepackage{subfigure}
\usepackage{physics}
\usepackage{tikz}
\usepackage[colorlinks, linkcolor=black, anchorcolor=black, citecolor=black]{hyperref}
\usepackage{prettyref}

\geometry{left=3.18cm,right=3.18cm,top=2.54cm,bottom=2.54cm}
\titlespacing{\paragraph}{0pt}{1pt}{10pt}[20pt]
\setlength{\droptitle}{-5em}
\preauthor{\vspace{-10pt}\begin{center}}
\postauthor{\par\end{center}}

\DeclareMathOperator{\timeorder}{T}
\DeclareMathOperator{\diag}{diag}
\newcommand*{\ii}{\mathrm{i}}
\newcommand*{\ee}{\mathrm{e}}
\newcommand*{\const}{\mathrm{const}}
\newcommand*{\comment}{\paragraph{注记}}

\newrefformat{sec}{第\ref{#1}节}
\newrefformat{note}{注\ref{#1}}
\renewcommand{\autoref}{\prettyref}

\newenvironment{bigcase}{\left\{\quad\begin{aligned}}{\end{aligned}\right.}

\title{气体理论}
\author{吴何友}

\begin{document}

\maketitle

标记气体的物理量肯定包括能量和粒子数,这两者的算符对易。通常气体会被约束在一个深势陷当中,由于气体具有均一性(不是说描述它的物理定律是均一的——描述其它物态的物理定律照样是均一的——而是说它的各物理量在不同空间位置是均匀的),势陷的形状不应该影响气体状态,因此气体与势陷的关系完全由体积$V$确定。
体积是外界给定的参数,因此和其它任何量都对易。于是可以使用$E, N, V$标记气体状态。
这三个参数也许够了,也许还需要更多参数——例如可以是混合气体,那么还需要每种组分的粒子数。

\section{经典理想气体}

理想气体指气体分子全同且彼此之间没有复杂的相互作用的一种物质。如果真的没有分子间相互作用,则其哈密顿量可以写成
\begin{equation}
    H = \sum_i \frac{p_i^2}{2m},
\end{equation}
当然,这不可能是真正的哈密顿量,否则分子之间没有相互作用,系统就不可能达到热平衡。因此实际上的哈密顿量还要加上一个碰撞项。

\subsection{状态方程}

经典的理想气体状态方程为
\begin{equation}
    pV=N k_B T,
\end{equation}
其中$T$为一个状态参数,称为\textbf{理想气体温标}。

实际上,只要我们确信压强和温度成正比,可以直接从熵出发非常简单地推导状态方程。平衡时熵是
\[
    S = k_B \ln \left(\frac{V}{\Delta V}\right)^N = N \ln \left( \frac{V}{\Delta V} \right),
\]
使用麦克斯韦关系式
\[
    \left( \pdv{S}{V} \right)_T = \left( \pdv{p}{T} \right)_V = \frac{p}{T},
\]
就得到
\[
    p = T \left( \pdv{S}{V} \right)_T = \frac{N k_B T}{V}.
\]
这就是全部了。

\subsection{理想气体的热力学}

外界压缩气体即能够做功,于是
\begin{equation}
    \var{W} = - p \dd{V}.
\end{equation}
热力学第一定律为
\begin{equation}
    \dd{U} = - p \dd{W} + \dd{S}, \quad \dd{S} \geq T \dd{S}.
\end{equation}
等号在可逆时取得。

理想气体温标实际上就是热力学温标。我们如下构造理想气体的卡诺过程:
\begin{enumerate}
    \item 在温度$T_1$下气体状态从$(p_1, V_1)$演化为$(p_2, V_2)$,即发生一个等温过程,此过程中从外界吸收热量;
    \item 
\end{enumerate}

计算得到的热机效率的形式和热力学温标下的热机效率是一致的,因此热力学温标和理想气体温标最多差一个常数因子,那么可以不失一般性地认为这两种温标相同。

一种把所有这些量都记住的口诀是:
Good physicists have studied under very fine teachers.
把每个词的首字母拿出来,排成一圈。

气体内能的公式必须通过哈密顿量得到,因此不能只通过热力学确定。然而,理想气体的内能实际上都仅仅依赖于温度。这是因为对平衡态气体有
\[
    \dd{U} = - p \dd{V} + T \dd{S},
\]
于是
\[
    \left(\pdv{S}{V}\right)_T = \left(\pdv{P}{T}\right)_V,
\]
从而
\[
    \begin{aligned}
        \left(\pdv{U}{V}\right)_T &= \left(\pdv{U}{V}\right)_S + \left(\pdv{U}{S}\right)_V \left(\pdv{S}{V}\right)_T \\
        &= - p + T \left(\pdv{P}{T}\right)_V.
    \end{aligned}
\]
对理想气体,
\[
    \left(\pdv{P}{T}\right)_V = \frac{p}{T},
\]
于是
\[
    \left(\pdv{U}{V}\right)_T = 0.
\]
因此内能只和温度有关。对别的类型的气体这个式子未必成立。

\subsection{麦克斯韦速度分布律}

在热平衡时理想气体中气体分子的速度分布是什么样的?实际上可以使用简单的对称性分析得到其速度分布律。
设$f(v_x),f(v_y),f(v_z)$分别是三个方向上的速度分布函数,$f(\vb*{v})$则是完整的速度分布函数,于是就有
\[
    f(v_x) = \int \dd{v_y} \int \dd{v_z} f(\vb*{v}),
\]
等等。

首先考虑碰撞项比较小的情况。此时碰撞项的作用只是促成热平衡,因此每个分子三个方向上的运动速度近似是解耦的,也即我们有
\[
    f(\vb*{v}) = f(v_x) f(v_y) f(v_z).
\]
由空间旋转对称性,$f(v_x),f(v_y),f(v_z)$这三个函数的形式是一样的,并且$f(\vb*{v})$只应该依赖于$\vb*{v}$的某个标量函数,从而有
\[
    f(\vb*{v}) = g(v_x^2+v_y^2+v_z^2).
\]
于是就得到函数方程
\[
    f(v_x) f(v_y) f(v_z) = g(v_x^2+v_y^2+v_z^2).
\]
将其中两个方向上的速度固定为零,我们有
\[
    f(v_x) \propto g(v_x^2),
\]
于是考虑到三个方向上的速度分布函数是一样的,就得到
\[
    C g(v_x^2) g(v_y^2) g(v_z^2) = g(v_x^2+v_y^2+v_z^2).
\]
能够满足这种条件的只有指数函数,于是我们就得到
\[
    g(v_x^2) \propto \ee^{A v_x^2},
\]
最后
\[
    f(v_x) = C \ee^{A v_x^2}.
\]

\end{document}