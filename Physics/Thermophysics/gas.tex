\documentclass[hyperref, UTF8, a4paper]{ctexart}

\usepackage{geometry}
\usepackage{titling}
\usepackage{titlesec}
\usepackage{paralist}
\usepackage{footnote}
\usepackage{enumerate}
\usepackage{amsmath, amssymb, amsthm}
\usepackage{autobreak}
\usepackage{cite}
\usepackage{graphicx}
\usepackage{subfigure}
\usepackage{physics}
\usepackage{tikz}
\usepackage[colorlinks, linkcolor=black, anchorcolor=black, citecolor=black]{hyperref}
\usepackage{prettyref}

\geometry{left=3.18cm,right=3.18cm,top=2.54cm,bottom=2.54cm}
\titlespacing{\paragraph}{0pt}{1pt}{10pt}[20pt]
\setlength{\droptitle}{-5em}
\preauthor{\vspace{-10pt}\begin{center}}
\postauthor{\par\end{center}}

\DeclareMathOperator{\timeorder}{T}
\DeclareMathOperator{\diag}{diag}
\newcommand*{\ii}{\mathrm{i}}
\newcommand*{\ee}{\mathrm{e}}
\newcommand*{\const}{\mathrm{const}}
\newcommand*{\comment}{\paragraph{注记}}

\newrefformat{sec}{第\ref{#1}节}
\newrefformat{note}{注\ref{#1}}
\renewcommand{\autoref}{\prettyref}

\newenvironment{bigcase}{\left\{\quad\begin{aligned}}{\end{aligned}\right.}

\title{气体和动理学}
\author{吴何友}

\begin{document}

\maketitle

标记气体的物理量肯定包括能量和粒子数,这两者的算符对易。通常气体会被约束在一个深势陷当中,由于气体具有均一性(不是说描述它的物理定律是均一的——描述其它物态的物理定律照样是均一的——而是说它的各物理量在不同空间位置是均匀的),势陷的形状不应该影响气体状态,因此气体与势陷的关系完全由体积$V$确定。
体积是外界给定的参数,因此和其它任何量都对易。于是可以使用$E, N, V$标记气体状态。
这三个参数也许够了,也许还需要更多参数——例如可以是混合气体,那么还需要每种组分的粒子数。

\section{压强-体积系统的热力学}

\subsection{热力学第一定律}

气体具有均一性,当平衡时封闭空间中的气体处处压强相等,而外界对气体做功为
\[
    \begin{aligned}
        \var{W} &= - \int \dd{\vb*{S}} \cdot (\vb*{n} \vb*{n} p) \cdot \var{\vb*{r}} \\
        &= - p \int \dd{\vb*{S}} \cdot \var{\vb*{r}} \\
        &= - p \dd{V}, 
    \end{aligned}
\]
于是气体内能变化就是
\begin{equation}
    \dd{U} = -p \dd{V} + T \dd{S}.
    \label{eq:first-law-of-gas}
\end{equation}
这样,只需要$V, T$两个参数——或者等价的任何其它两个参数——就可以确定平衡态时的气体的热力学状态。
要完整描述气体状态当然还需要诸如容器体积之类的信息,但它们对气体的热力学状态毫无影响。
如果气体和外界有交换,设$\{N_i\}$是气体中各种成分的分子数,那么
\begin{equation}
    \dd{U} = -p \dd{V} + T \dd{S} + \sum_i \mu_i \dd{N_i}.
    \label{eq:first-law-of-gas-extended}
\end{equation}
热力学第一定律\eqref{eq:first-law-of-gas}和\eqref{eq:first-law-of-gas-extended}对任何气体系统都适用。
由于它们的导出只用到了“系统表面只有处处相等的正应力”这个条件,它们实际上适用于所有这种类型的系统——例如,它们适用于液体系统,也适用于被放置在气体环境中的固体系统,虽然固体内部可以有应力变化和非正应力的分量。

要完整地确定气体的热力学性质还需要一些其它的公式。
回顾使用极大概然法分析系统的步骤,我们首先写出内能关于系统状态的表达式,并且通过分析状态数$\Omega$得到了熵关于系统状态的表达式,这两个表达式缺一不可(原因也很简单,内能一样的系统熵可以不一样)。
如果熵的表达式不很清楚,可以使用状态方程,也就是$p, V, T$之间的关系。不过无论如何要有两个方程。

从\eqref{eq:first-law-of-gas}出发,可以得到一系列其它的热力学势函数。

\begin{equation}
    C_p - C_V = \left( p + \left(\pdv{U}{V}\right)_T \right) \left(\pdv{V}{T}\right)_p = T \left(\pdv{V}{T}\right)_p \left(\pdv{p}{T}\right)_V,
\end{equation}
\begin{equation}
    \left(\pdv{U}{V}\right)_T + p = T \left(\pdv{S}{V}\right)_T = T \left(\pdv{p}{T}\right)_V,
\end{equation}

\section{经典理想气体}

理想气体指气体分子全同且彼此之间没有复杂的相互作用的一种物质。如果真的没有分子间相互作用,则其哈密顿量可以写成
\begin{equation}
    H = \sum_i \frac{p_i^2}{2m},
\end{equation}
当然,这不可能是真正的哈密顿量,否则分子之间没有相互作用,系统就不可能达到热平衡。因此实际上的哈密顿量还要加上一个碰撞项。

\subsection{状态方程}

经典的理想气体状态方程为
\begin{equation}
    pV=N k_B T,
\end{equation}
其中$T$为一个状态参数,称为\textbf{理想气体温标}。

实际上,只要我们确信压强和温度成正比,可以直接从熵出发非常简单地推导状态方程。平衡时熵是
\[
    S = k_B \ln \left(\frac{V}{\Delta V}\right)^N = N \ln \left( \frac{V}{\Delta V} \right),
\]
使用麦克斯韦关系式
\[
    \left( \pdv{S}{V} \right)_T = \left( \pdv{p}{T} \right)_V = \frac{p}{T},
\]
就得到
\[
    p = T \left( \pdv{S}{V} \right)_T = \frac{N k_B T}{V}.
\]
这就是全部了。

\subsection{理想气体的热力学}

外界压缩气体即能够做功,于是
\begin{equation}
    \var{W} = - p \dd{V}.
\end{equation}
热力学第一定律为
\begin{equation}
    \dd{U} = - p \dd{W} + \dd{S}, \quad \dd{S} \geq T \dd{S}.
\end{equation}
等号在可逆时取得。

理想气体温标实际上就是热力学温标。我们如下构造理想气体的卡诺过程:
\begin{enumerate}
    \item 在温度$T_1$下气体状态从$(p_1, V_1)$演化为$(p_2, V_2)$,即发生一个等温过程,此过程中从外界吸收热量;
    \item 
\end{enumerate}

计算得到的热机效率的形式和热力学温标下的热机效率是一致的,因此热力学温标和理想气体温标最多差一个常数因子,那么可以不失一般性地认为这两种温标相同。

一种把所有这些量都记住的口诀是:
Good physicists have studied under very fine teachers.
把每个词的首字母拿出来,排成一圈。

气体内能的公式必须通过哈密顿量得到,因此不能只通过热力学确定。然而,理想气体的内能实际上都仅仅依赖于温度。这是因为对平衡态气体有
\[
    \dd{U} = - p \dd{V} + T \dd{S},
\]
于是
\[
    \left(\pdv{S}{V}\right)_T = \left(\pdv{P}{T}\right)_V,
\]
从而
\[
    \begin{aligned}
        \left(\pdv{U}{V}\right)_T &= \left(\pdv{U}{V}\right)_S + \left(\pdv{U}{S}\right)_V \left(\pdv{S}{V}\right)_T \\
        &= - p + T \left(\pdv{P}{T}\right)_V.
    \end{aligned}
\]
对理想气体,
\[
    \left(\pdv{P}{T}\right)_V = \frac{p}{T},
\]
于是
\[
    \left(\pdv{U}{V}\right)_T = 0.
\]
因此内能只和温度有关。对别的类型的气体这个式子未必成立。

\begin{equation}
    S = Nk \ln V + \frac{i}{2} Nk \ln T + \const.
    \label{eq:entropy-from-thermodynamics}
\end{equation}

\subsection{麦克斯韦速度分布律}

在热平衡时理想气体中气体分子的速度分布是什么样的?实际上可以使用简单的对称性分析得到其速度分布律。
设$f(v_x),f(v_y),f(v_z)$分别是三个方向上的速度分布函数,$f(\vb*{v})$则是完整的速度分布函数,于是就有
\[
    f(v_x) = \int \dd{v_y} \int \dd{v_z} f(\vb*{v}),
\]
等等。

首先考虑碰撞项比较小的情况。此时碰撞项的作用只是促成热平衡,因此每个分子三个方向上的运动速度近似是解耦的,也即我们有
\[
    f(\vb*{v}) = f(v_x) f(v_y) f(v_z).
\]
由空间旋转对称性,$f(v_x),f(v_y),f(v_z)$这三个函数的形式是一样的,并且$f(\vb*{v})$只应该依赖于$\vb*{v}$的某个标量函数,从而有
\[
    f(\vb*{v}) = g(v_x^2+v_y^2+v_z^2).
\]
于是就得到函数方程
\[
    f(v_x) f(v_y) f(v_z) = g(v_x^2+v_y^2+v_z^2).
\]
将其中两个方向上的速度固定为零,我们有
\[
    f(v_x) \propto g(v_x^2),
\]
于是考虑到三个方向上的速度分布函数是一样的,就得到
\[
    C g(v_x^2) g(v_y^2) g(v_z^2) = g(v_x^2+v_y^2+v_z^2).
\]
能够满足这种条件的只有指数函数,于是我们就得到
\[
    g(v_x^2) \propto \ee^{A v_x^2},
\]
最后
\[
    f(v_x) = C \ee^{A v_x^2}.
\]

\subsection{配分函数}

首先考虑球形粒子,即只有质心三个自由度的粒子,那么理想气体的经典配分函数是
\begin{equation}
    Z = \int \prod_i \dd{p_i} \dd{q_i} \ee^{- \beta \sum_i \frac{p_i^2}{2m}} = \prod_i \int \dd{q_i} \int \dd{p_i} \ee^{- \frac{p_i^2}{2m} }.
    \label{eq:simplest-ideal-gas-partition-def}
\end{equation}
首先尝试经典的计算,即认为$p,q$可以连续取值。设共有$N$个粒子,则
\[
    \prod_i \int \dd{q_i} = V^{N},
\]
而
\[
    \prod_i \int \dd{p_i} \ee^{- \beta \frac{p_i^2}{2m}} = \left( \frac{2\pi m}{\beta} \right)^{\frac{3N}{2}},
\]
于是
\begin{equation}
    Z = V^{N} \left( \frac{2\pi m}{\beta} \right)^{\frac{3N}{2}}.
    \label{eq:simplest-ideal-gas-partition}
\end{equation}
如果我们把经典配分函数\eqref{eq:simplest-ideal-gas-partition-def}当成正确的配分函数用于计算自由能,那么
\[
    F = - k T \ln Z = - kNT \ln V - \frac{3}{2} kNT \ln (2\pi m T),
\]
从而可以计算熵为
\begin{equation}
    S = - \pdv{F}{T} = Nk \ln V + \frac{3}{2} kN \ln T + \frac{3}{2} kN + \frac{3}{2} kN \ln (2\pi m).
    \label{eq:naive-entropy}
\end{equation}
\eqref{eq:naive-entropy}和\eqref{eq:entropy-from-thermodynamics}是一致的——本该如此。\eqref{eq:naive-entropy}的不足之处在于它不满足可加性。
如果两个分别达到平衡,彼此只隔着一个隔板的同种粒子、温度相同的理想气体系统具有一样的压强,那么抽去隔板之后总系统立即达到平衡。
% TODO:为什么?虽然直觉上理应如此
然而,\eqref{eq:naive-entropy}不满足这个条件,因为显然其中的第一项不是线性的。
这就是吉布斯佯谬——本应具有可加性的熵变得不可加了。
为了让熵具有可加性我们不得不手动校准其零点,从而需要修改$F$,在其中引入一些不改变热力学预言的项,这表明\eqref{eq:simplest-ideal-gas-partition-def}不是正确的配分函数,它差了一些常数因子。

考虑全同粒子的不可分辨性(它意味着交换两个粒子虽然会让系统的正则坐标发生变化但并不改变系统状态,为了消除掉这种重复,在配分函数前面加上因子$1/N!$)以及相空间需要划分为相格(它在配分函数前面加上因子$1/h^{3N}$),将\eqref{eq:simplest-ideal-gas-partition}修改为
\begin{equation}
    Z = \frac{1}{N!} V^{N} \left( \frac{m}{2\pi \beta \hbar^2} \right)^{\frac{3N}{2}}.
\end{equation}
重复以上计算,得到
\[
    F = - kNT \ln V - \frac{3}{2} kNT \ln \left(\frac{m T}{2\pi \hbar^2}\right) + k T \ln N!,
\]
从而熵为
\[
    S = N k \ln V + \frac{3}{2} Nk \ln T + \frac{3}{2} kN + \frac{3}{2} kN \ln \left( \frac{m}{2\pi \hbar^2} \right) - k \ln N!.
\]
表面上这个熵表达式仍然没有可加性,但请注意由于经典统计仅仅适用于大粒子数系统,我们可以应用斯特林公式
\[
    \ln N! \approx N \ln N - N,
\]
然后得到的熵表达式就是可加的了。总之,为了导出正确的熵,不能使用代表点密度的归一化常数为配分函数,而必须加上必要的修正因子。

回顾以上修正。关于相空间应该分成相格的问题我们不做过多讨论,如果不加入这个修正因子,最后得到的熵实际上还是具有可加性的。加入这个因子的作用仅仅是为了确保通过经典统计物理计算出来的熵和量子统计的结果一致。
真正恢复了可加性的实际上是因子$1/N!$。我们知道,吉布斯佯谬只有在混合的两盒气体真正是全同的时候才是佯谬,如果混合的两盒气体是不一样的,那么混合它们就是会有熵增的。
换而言之,经典配分函数前面的修正因子必须能够区分全同粒子和非全同粒子。
$1/N!$正是这样一个因子。

\section{气体动理学}

本节讨论微分方程表述的宏观气体的动理学。由于热涨落,通常不能够直接使用哈密顿量做动力学演化而需要加入一些环境项而得到主方程。
由于我们关注的是宏观气体,可以认为气体几乎总是处于经典的状态,即动量和坐标近似对易,密度算符在坐标-动量基下对角化。

我们认为气体中的分子相互作用仅包含二分子相互作用,即
\begin{equation}
    H = \sum_i \frac{p_i^2}{2m} + \sum_{i \neq j} U(\abs*{\vb*{r}_i - \vb*{r}_j}).
    \label{eq:general-gas-hamiltonian}
\end{equation}
这个假设通常是合理的,因为气体分子发生相互作用的时间尺度通常远小于电磁场的传播需要的时间,因此在气体分子发生相互作用的时间尺度上,基本上只有静电学和静磁学现象。
这样,分子间相互作用无非是某种由于电荷分布不均匀而非常复杂的库伦相互作用。

\subsection{波尔兹曼方程}

\subsubsection{从分子混沌性假设导出玻尔兹曼方程}

考虑无粒子生灭的气体,假定
\begin{enumerate}
    \item 只考虑二体弹性碰撞,即不讨论碰撞导致化学反应,也不讨论多体过程(这个假设已经包含在\eqref{eq:general-gas-hamiltonian}中了);
    \item 气体相互作用时涉及到的自由度只有整体的坐标和动量(这个假设也已经包含在\eqref{eq:general-gas-hamiltonian}中了);
    \item 气体不非常密集以至于气体的平均自由时间相对于气体分子散射的时间尺度很大,从而碰撞可以看成是瞬时的;
    \item 气体分子间的相互作用随距离增长快速衰减,从而碰撞可以看成是定域的;
    \item 气体分子受外力作用的时间尺度远大于碰撞,即漂移和碰撞解耦;
    \item 气体分子的分布彼此独立。
\end{enumerate}
前几个假设要求气体分子是简单的点粒子且气体稀薄,最后一个假设(称为\textbf{分子混沌假设})并不是非常显然,但通过BBGKY层级可以证明它是正确的。
% TODO
把每个气体分子的坐标和动量都绘制在一个六维空间$(\vb*{r}, \vb*{p})$中,%
\footnote{注意这不是系统的相空间而是单个粒子的相空间,系统的相空间是一个$6N$维空间。}%
设此单粒子相空间中的分布函数为$f$,它是多粒子分布函数积掉$N-1$个粒子,只剩下单个粒子所得到的结果:
\begin{equation}
    f(\vb*{q}, \vb*{p}, t) = \int \prod_{i=1}^{N-1} \dd{\vb*{q}_i} \dd{\vb*{p}_i} \rho(\{\vb*{q}_i\}, \{\vb*{p}_i\}, t),
\end{equation}
那么对只涉及单粒子的物理量或者一系列仅涉及单粒子的物理量之和$A$,就有
\begin{equation}
    \expval*{A(t)} = \int \dd{\vb*{p}} \dd{\vb*{q}} f(\vb*{q}, \vb*{p}, t) A(\vb*{q}, \vb*{p}).
\end{equation}
我们来推导输运方程,即$f$服从的方程。

\begin{equation}
    \pdv{f}{t} + \vb*{v} \cdot \pdv{f}{\vb*{r}} + \frac{\vb*{F}}{m} \cdot \pdv{f}{\vb*{v}} = \int \dd{\vb*{v}_2} \dd{\Omega} \sigma \abs*{\vb*{r}-\vb*{r}_2} (\underbrace{f(\vb*{r}, \vb*{v}', t) f(\vb*{r}, \vb*{v}'_{2}, t)}_\text{in} - \underbrace{f(\vb*{r}, \vb*{v}, t) f(\vb*{r}, \vb*{v}_{2}, t)}_\text{out}).
    \label{eq:boltzmann-eq}
\end{equation}
等号右边的一长串称为碰撞积分。对一个相互作用势能$U(r)$,理论上总是可以计算出散射截面和碰撞之后的速度变化。

\subsubsection{细致平衡条件和麦克斯韦分布}

\eqref{eq:boltzmann-eq}意味着相空间中位于$\vb*{r}, \vb*{v}$附近的粒子数有进有出,并且粒子的转移是无记忆的。\textbf{细致平衡条件}是说,很大一类系统平衡时从状态$i$转移到状态$j$和从状态$j$转移到状态$i$的概率相同,在这里就是
\begin{equation}
    f(\vb*{r}, \vb*{v}'_1, t) f(\vb*{r}, \vb*{v}'_{2}, t) = f(\vb*{r}, \vb*{v}_1, t) f(\vb*{r}, \vb*{v}_{2}, t).
\end{equation}
这个方程有没有解,到现在还不得而知,不过马上可以看到\eqref{eq:boltzmann-eq}可以直接推导出麦克斯韦分布。
两边取对数得到
\[
    \ln f(\vb*{r}, \vb*{v}'_1, t) + \ln f(\vb*{r}, \vb*{v}'_{2}, t) = \ln f(\vb*{r}, \vb*{v}_1, t) + \ln f(\vb*{r}, \vb*{v}_{2}, t),
\]
这个方程意味着$\ln f_1 + \ln f_2$是某种守恒量。\eqref{eq:general-gas-hamiltonian}没有给出任何单粒子守恒量(除了粒子数以外),因此只能是
\[
    \ln f(\vb*{r}, \vb*{v}_1, t) + \ln f(\vb*{r}, \vb*{v}_{2}, t) = F(\vb*{v}_1, \vb*{v}_2).
\]
要让等式右边能够分解成一个关于$\vb*{v}_1$的函数加上一个关于$\vb*{v}_2$的函数之和,仅有的可能是
\[
    \ln f(\vb*{r}, \vb*{v}_1, t) + \ln f(\vb*{r}, \vb*{v}_{2}, t) = \sum_i \underbrace{(F_i(\vb*{v}_1) + F_i(\vb*{v}_2))}_{=\const}.
\]
\eqref{eq:general-gas-hamiltonian}的独立守恒量包括粒子数守恒、动量守恒(即任何方向上都有空间平移不变性)、动能守恒(即时间平移不变性)。
(角动量守恒可以从动量守恒推出)
因此,我们有
\[
    \ln f(\vb*{r}, \vb*{v}, t) = A + \vb*{\alpha} \cdot \vb*{v} + C v^2,
\]
或者等价的
\[
    \ln f(\vb*{r}, \vb*{v}, t) = A (\vb*{v} - \vb*{v}_0)^2.
\]
加入温度等常数,我们就得到了麦克斯韦分布。

\subsubsection{H定理与时间反演}

虽然表面上看起来,使用H定理让我们能够从牛顿方程推出热力学第二定律,但仔细考虑会发现几个很大的问题:
\begin{enumerate}
    \item 牛顿方程是时间反演不变的,热力学第二定律不是;
    \item 庞加莱回归意味着封闭系统的演化总是可以几乎回到其初始状态,即熵可以回到原点,这似乎和热力学第二定律矛盾;
    \item 
\end{enumerate}

如果一开始粒子之间的分布确实是完全无关的,发生几次散射之后它们的分布就产生关联了,或者说一个粒子的运动情况的信息会传递给别的粒子。分子混沌假设等于人为引入了环境扰动,使得关于粒子运动情况的信息会源源不断地被导向外部,这是熵增的根本原因。

\subsection{BBGKY序列}

\subsubsection{BBGKY序列的导出}

实际上,还可以直接从刘维尔方程出发推导气体的动理学方程。\eqref{eq:general-gas-hamiltonian}会导致如下的刘维尔方程:
\[
    \dv{P_N}{t} + \sum_i \left( \vb*{v}_i \cdot \pdv{P_N}{\vb*{r}_i} + \frac{\vb*{F}_i}{m} \cdot \pdv{P_N}{\vb*{v}_i} \right), \quad \vb*{F}_i = - \sum_{j \neq i} \pdv{U_{ij}}{\vb*{r}_i}, 
\]
其中$P_N$是$\{\vb*{q}_i, \vb*{p}_i\}$的函数,且随意交换两个粒子,$P_N$不变。考虑如下的$s$粒子边缘分布:
\begin{equation}
    P_N^{(s)} = \int \prod_{i \geq s+1} \dd{\vb*{r}_i} \dd{\vb*{v}_i} P_N,
\end{equation}
使用$P_N$的对称性以及积分边界项为零的事实,可以推导出
\begin{equation}
    \pdv{P_N^{(1)}}{t} + \vb*{v}_1 \cdot \pdv{P_N^{(1)}}{\vb*{r}_1} = \frac{N-1}{m} \int \dd{\vb*{r}_2} \dd{\vb*{v}_2} \pdv{U_{12}}{\vb*{r}_1} \cdot \pdv{P_N^{(2)}}{\vb*{v}_1},  
    \label{eq:from-p2-to-p1}
\end{equation}
以及类似的从$P^{(3)}_N$推导出$P^{(2)}_N$的方程
\begin{align}
    \begin{autobreak}
        \pdv{P_N^{(2)}}{t} + \vb*{v}_1 \cdot \pdv{P_N^{(2)}}{\vb*{r}_1} 
        + \vb*{v}_2 \cdot \pdv{P_N^{(2)}}{\vb*{r}_2} 
        - \frac{1}{m} \pdv{U_{12}}{\vb*{r}_1} \cdot \pdv{P_N^{(2)}}{\vb*{v}_1} 
        - \frac{1}{m} \pdv{U_{12}}{\vb*{r}_2} \cdot \pdv{P_N^{(2)}}{\vb*{v}_2} 
        = \frac{N-2}{m} \int \dd{\vb*{r}_3} \dd{\vb*{v}_3} \left( \pdv{P_N^{(3)}}{\vb*{v}_1} \cdot \pdv{U_{13}}{\vb*{r}_1} + \pdv{P_N^{(3)}}{\vb*{v}_2} \cdot \pdv{U_{23}}{\vb*{r}_2} \right),
    \end{autobreak}
    \label{eq:from-p3-to-p2}
\end{align}
还有从$P_N^{(4)}$推导出$P_N^{(3)}$等等的方程。这就是\textbf{BBGKY序列}。
将某个高阶$P_N^{(s)}$取为零,就可以做一个截断,从而得到一组自洽方程,可以从$P_N^{(s)}$计算$P_N^{(s-1)}$,最后计算出$P_N^{(1)}$,这就得到了$f$。

\subsubsection{BBGKY序列和玻尔兹曼方程}

\eqref{eq:from-p3-to-p2}右边的空间积分只有在两个粒子间距在相互作用力程$d$中,即满足
\[
    \abs*{\vb*{r}_i - \vb*{r}_j} \lesssim d
\]
时才有非零值,因此该积分应该和$d^3$同阶。另一方面,设分子间距的数量级为$\delta$,则系统体积满足
\[
    V \sim N \delta^3.
\]
最后,注意到对整个系统体积和动量空间积分,有
\[
    \int \dd{\vb*{r}_3} \dd{\vb*{v}_3} \pdv{P_N^{(3)}}{\vb*{v}_1} \cdot \pdv{U_{13}}{\vb*{r}_1} \sim \pdv{P_N^{(2)}}{\vb*{v}_1} \cdot \pdv{U_{13}}{\vb*{r}_1},
\]
于是合起来就有
\[
    \frac{N-2}{m} \int \dd{\vb*{r}_3} \dd{\vb*{v}_3} \pdv{P_N^{(3)}}{\vb*{v}_1} \cdot \pdv{U_{13}}{\vb*{r}_1} \sim \frac{1}{m} \pdv{P_N^{(2)}}{\vb*{v}_1} \cdot \pdv{U_{13}}{\vb*{r}_1} \frac{d^3}{\delta^3}.
\]
如果气体非常稀薄,那么$d/\delta$就是小量,于是\eqref{eq:from-p3-to-p2}右边可以略去,得到
\begin{equation}
    \pdv{P_N^{(2)}}{t} 
    + \vb*{v}_1 \cdot \pdv{P_N^{(2)}}{\vb*{r}_1} 
    + \vb*{v}_2 \cdot \pdv{P_N^{(2)}}{\vb*{r}_2} 
    - \frac{1}{m} \pdv{U_{12}}{\vb*{r}_1} \cdot \pdv{P_N^{(2)}}{\vb*{v}_1} 
    - \frac{1}{m} \pdv{U_{12}}{\vb*{r}_2} \cdot \pdv{P_N^{(2)}}{\vb*{v}_2} = 0.
    \label{eq:effective-2-particle}
\end{equation}
上式实际上可以写成一个全微分的形式:
\[
    \dv{P_N^{(2)}}{t}=0,
\]
这个全微分沿着哈密顿量
\begin{equation}
    H_\text{eff} = \frac{p_1^2}{2m} + \frac{p_2^2}{2m} + U(\abs*{\vb*{r}_1 - \vb*{r}_2})
    \label{eq:2-particle-hamiltonian}
\end{equation}
描写的相轨道。这当然是正确的,因为近似\eqref{eq:effective-2-particle}只使用了两对坐标-动量对,因此描述了一个近似只有两个粒子的系统,那么它显然应该服从二粒子系统的刘维尔定律。

假定不同粒子的分布基本独立(我们在此引入分子混沌性假设),从而
\[
    P^{(2)}_N(\vb*{r}_1, \vb*{v}_1, \vb*{r}_2, \vb*{v}_2, t) = P^{(1)}_N(\vb*{r}_1, \vb*{v}_1, t) P^{(1)}_N(\vb*{r}_2, \vb*{v}_2, t).
\]
将上式沿着\eqref{eq:2-particle-hamiltonian}描述的相轨道积分就得到
\[
    P^{(2)}_N(\vb*{r}_1, \vb*{v}_1, \vb*{r}_2, \vb*{v}_2, t) = P^{(1)}_N(\vb*{r}_{10}, \vb*{v}_{10}, t_0) P^{(1)}_N(\vb*{r}_{20}, \vb*{v}_{20}, t_0).
\]
注意这里的$\vb*{r}_{10}, \vb*{r}_{20}$是$t-t_0,\vb*{r}_1,\vb*{r}_2,\vb*{v}_1, \vb*{v}_2$的函数,而$\vb*{v}_{10}, \vb*{v}_{20}$是$\vb*{r}_1,\vb*{r}_2,\vb*{v}_1, \vb*{v}_2$的函数。
将上式代入\eqref{eq:from-p2-to-p1},并将$N-1$近似为$N$,考虑到$f=NP_N^{(1)}$,得到
\[
    \pdv{f}{t} + \vb*{v}_1 \cdot \pdv{f}{\vb*{r}_1} = \frac{1}{m} \int \dd{\vb*{r}_2} \dd{\vb*{v}_2} \pdv{U_{12}}{\vb*{r}_1} \cdot \pdv{f(\vb*{r}_{10}, \vb*{v}_{10}, t_0) f(\vb*{r}_{20}, \vb*{v}_{20}, t_0)}{\vb*{v}_1}.
\]
由于$f$要出现显著的变化需要在平均自由程的尺度上,而上式所述的积分在该尺度上是高度定域的($d$远小于平均自由程),我们有
\[
    \pdv{\vb*{r}}{t} \cdot \pdv{f}{\vb*{r}} \ll \pdv{U}{\vb*{r}} \cdot \pdv{f}{\vb*{v}},
\]
这样可以将\eqref{eq:effective-2-particle}中的时间偏导数项去掉,而对$\vb*{r}_2$的偏导数对$\dd{\vb*{r}_2}$积分之后得到表面项,为零,于是
\[
    \begin{aligned}
        \pdv{f}{t} + \vb*{v}_1 \cdot \pdv{f}{\vb*{r}_1} &= \int \dd{\vb*{r}_2} \dd{\vb*{v}_2} \left( \vb*{v}_1 \cdot \pdv{f(\vb*{r}_{10}, \vb*{v}_{10}, t_0) f(\vb*{r}_{20}, \vb*{v}_{20}, t_0)}{\vb*{r}_1} + \vb*{v}_2 \cdot \pdv{f(\vb*{r}_{10}, \vb*{v}_{10}, t_0) f(\vb*{r}_{20}, \vb*{v}_{20}, t_0)}{\vb*{r}_2} \right) \\
        &= \int \dd{\vb*{r}} \dd{\vb*{v}_2} \vb*{u} \cdot \pdv{f(\vb*{r}_{10}, \vb*{v}_{10}, t_0) f(\vb*{r}_{20}, \vb*{v}_{20}, t_0)}{\vb*{r}},
    \end{aligned}
\]
其中$\vb*{r}$和$\vb*{u}$分别是粒子1和粒子2的相对位移和相对速度。
以$\vb*{u}$的方向为$z$轴建立柱坐标系,得到
\[
    \pdv{f}{t} + \vb*{v}_1 \cdot \pdv{f}{\vb*{r}_1} = \int \rho \dd{\rho} \dd{\varphi} \dd{\vb*{v}_2} u (f(\vb*{r}_{10}, \vb*{v}_{10}, t_0) f(\vb*{r}_{20}, \vb*{v}_{20}, t_0))\big|_{z=-\infty}^\infty.
\]
请注意$\vb*{r}$是两个发生碰撞的粒子的相对位移,$\vb*{u}$指向碰撞发生的方向,则$z$趋于$\infty$意味着碰撞结束,而$z$趋于$-\infty$意味着碰撞开始。
无论是碰撞开始还是结束,做一个小的时间平移$t \longrightarrow t_0$都不会造成什么影响,而由于碰撞高度定域,碰撞前后位矢可认为基本不变,且均位于$\vb*{r}_1$附近。于是就得到
\[
    \begin{aligned}
        \pdv{f}{t} + \vb*{v}_1 \cdot \pdv{f}{\vb*{r}_1} &= \int \rho \dd{\rho} \dd{\varphi} \dd{\vb*{v}_2} u (f(\vb*{r}_{1}, \vb*{v}_{1}, t) f(\vb*{r}_{2}, \vb*{v}_{2}, t))\big|_{z=-\infty}^\infty \\
        &= \int \rho \dd{\rho} \dd{\varphi} \dd{\vb*{v}_2} u (f(\vb*{r}_{1}, \vb*{v}'_{1}, t) f(\vb*{r}_{1}, \vb*{v}'_{2}, t) - f(\vb*{r}_{1}, \vb*{v}_{1}, t) f(\vb*{r}_{1}, \vb*{v}_{2}, t)),
    \end{aligned}
\]
$\vb*{r}(t)$曲线组成一束束流管,微分形式$\rho \dd{\rho} \dd{\varphi}$实际上就是散射截面:
\[
    \rho \dd{\rho} \dd{\varphi} = \sigma \dd{\Omega},
\]
于是把$\vb*{v}_1$和$\vb*{r}_1$的下标去掉,最终得到
\begin{equation}
    \pdv{f}{t} + \vb*{v} \cdot \pdv{f}{\vb*{r}} = \int \dd{\vb*{v}_2} \dd{\Omega} \sigma \abs*{\vb*{r}-\vb*{r}_2} (f(\vb*{r}, \vb*{v}', t) f(\vb*{r}, \vb*{v}'_{2}, t) - f(\vb*{r}, \vb*{v}, t) f(\vb*{r}, \vb*{v}_{2}, t)).
\end{equation}

\subsection{和流体力学的联系}

\subsubsection{流动}

粒子流
\[
    \vb*{j} = \sum \vb*{v}
\]

\subsubsection{纳维-斯托克斯方程}

\begin{equation}
    \rho \left( \pdv{\vb*{v}}{t} + \vb*{v} \cdot \grad{\vb*{v}} \right) = - \grad{P} + \vb*{f}.
\end{equation}

\begin{equation}
    \pdv{\rho}{t} + \div{\rho \vb*{v}} = 0.
    \label{eq:transportation-eq}
\end{equation}

\subsubsection{声波}

当速度的时间变化相比于空间输运非常大时,即
\[
    \pdv{t} \gg \vb*{v} \cdot \grad
\]
时,近似有
\[
    \rho \pdv{\vb*{v}}{t} = - \grad{p},
\]
两边计算散度,并利用输运方程\eqref{eq:transportation-eq},得到
\[
    \laplacian{p} = \pdv[2]{\rho}{t},
\]
再假定压强变化不大,有
\[
    \rho = \eval{\pdv{\rho}{P}}_{P_0} (P - P_0) = \eval{\pdv{\rho}{P}}_{P_0} p,
\]
于是就得到波动方程
\begin{equation}
    \frac{1}{c^2} \pdv[2]{p}{t} = \laplacian{p},
\end{equation}
其中
\begin{equation}
    \frac{1}{c^2} = \eval{\pdv{\rho}{P}}_{P_0}.
\end{equation}
这就是说,快速振动而振幅不大的流体中会有线性机械波,这就是\textbf{声波}。

\section{非经典气体}

以下如无特殊说明,所有的气体系统都是三维的。

\subsection{玻色-爱因斯坦凝聚}

考虑一个非相对论自由理想玻色气体,也即,单个气体分子的能量完全为动能$k^2/2m$。
玻色气体在零温时必定处在玻色-爱因斯坦凝聚状态,我们来观察非相对论自由理想玻色气体的这种状态的具体行为。
近独立系统的粒子数期望为
\[
    n_i = \frac{1}{\ee^{\beta(\epsilon_i - \mu)} - 1},
\]
考虑单个粒子,在相空间体积$\dd[3]{\vb*{r}} \dd[3]{\vb*{p}}$中的量子态个数为
\[
    \dd{N} = \frac{\dd[3]{\vb*{r}} \dd[3]{\vb*{p}}}{h^3},
\]
由于动量空间的各向同性,只保留$\dd{p}$之后态密度为
\[
    \dd{N} = V \frac{4\pi p^2 \dd{p}}{h^3} = \frac{2\pi V}{h^3} (2m)^{3/2} \sqrt{\epsilon} \dd{\epsilon},
\]
其中$\epsilon$为自由费米子的能谱,即$p^2/2m$。
这样,近似有
\begin{equation}
    N = \int_0^\infty \frac{2\pi V}{h^3} (2m)^{3/2} \sqrt{\epsilon} \dd{\epsilon} \frac{1}{\exp(\beta (\epsilon - \mu)) - 1} = \frac{2V}{\sqrt{\pi}} \frac{1}{\lambda_T^3} g_{3/2}(\ee^{\beta \mu}),
\end{equation}
其中$g_\nu(z)$定义为
\[
    g_\nu(z) = \int_0^\infty \frac{x^{\nu-1}}{z^{-1} \ee^{x} - 1},
\]
而$\lambda_T$或者说\textbf{热德布罗意波长}是一个定义如下的长度尺度:
\begin{equation}
    \lambda_T = \sqrt{\frac{h^2}{2\pi m k_B T}}.
\end{equation}
上式成立的条件是能级间距非常密——这总是成立的——以及相邻能级上的粒子数变化得足够平滑——这却不总是成立。
请注意等式右边是有最大值的:当$z$取$1$时$g_{3/2}(z)$取最大值

\subsection{费米面}

考虑一个非相对论自由理想费米气体。此时气体分子倾向于待在最低的能级上,在$\vb*{p}$空间中填充一个球面。我们有
\begin{equation}
    \expval{n} = \begin{cases}
        1, \quad \epsilon < \mu, \\
        0, \quad \epsilon > \mu.
    \end{cases}
\end{equation}
$\mu$给出了有粒子填充的最高能级,称为\textbf{费米面}。费米面内部的粒子数为
\begin{equation}
    N = \int \dd{\epsilon} \underbrace{\frac{2\pi V}{h^3} (2m)^{3/2} \sqrt{\epsilon} \alpha}_\text{space density} = \frac{4\pi V (2m)^{3/2} \alpha}{3h^3} \epsilon_\text{F}^{3/2},
\end{equation}
其中$\alpha$指的是粒子的自旋可能的取值个数。
基态能量为(下面的推导用到了非相对论自由粒子的态密度)
\begin{equation}
    E = \int \dd{\epsilon} \underbrace{\frac{2\pi V}{h^3} (2m)^{3/2} \sqrt{\epsilon} \alpha}_\text{space density} \epsilon = \frac{4 \pi V (2m)^{3/2} \alpha}{5 h^3} \epsilon_\text{F}^{5/2} = \frac{3}{5} N \epsilon_\text{F}.
\end{equation}

\section{相互作用}

\end{document}