\documentclass[UTF8, a4paper]{ctexart}

\usepackage{geometry}
\usepackage{titling}
\usepackage{titlesec}
\usepackage{paralist}
\usepackage{footnote}
\usepackage{enumerate}
\usepackage{amsmath, amssymb, amsthm}
\usepackage{mathtools}
\usepackage{cite}
\usepackage{graphicx}
\usepackage{subfigure}
\usepackage{physics}
\usepackage[colorlinks, linkcolor=black, anchorcolor=black, citecolor=black]{hyperref}

\geometry{left=3.28cm,right=3.28cm,top=2.54cm,bottom=2.54cm}
\titlespacing{\paragraph}{0pt}{1pt}{10pt}[20pt]
\setlength{\droptitle}{-5em}
\preauthor{\vspace{-10pt}\begin{center}}
\postauthor{\par\end{center}}

\newcommand*{\ee}{\mathrm{e}}
\newcommand*{\ii}{\mathrm{i}}
\newcommand*{\st}{\quad \text{s.t.} \quad}
\newcommand*{\const}{\mathrm{const}}
\newcommand*{\natnums}{\mathbb{N}}
\newcommand*{\reals}{\mathbb{R}}
\newcommand*{\complexes}{\mathbb{C}}
\DeclareMathOperator{\timeorder}{T}
\newcommand*{\ogroup}[1]{\mathrm{O}(#1)}
\newcommand*{\sogroup}[1]{\mathrm{SO}(#1)}
\DeclareMathOperator{\legpoly}{P}
\newcommand*{\lsterm}[3]{$^{#1}${#2}$_{#3}$}

\title{原子物理}
\author{吴何友}

\begin{document}

\maketitle

\section{束缚态单电子系统}

\subsection{势阱}

有限高,无限厚的势阱:粒子可以有隧穿,但是在无穷远处衰减为零。

无限高,无限厚的势阱:没有任何隧穿。

\subsection{有心力场中的单电子}

\subsubsection{哈密顿量和薛定谔方程}

有心力场中的单个电子的哈密顿量为
\begin{equation}
    \hat{H} \psi = \frac{\hat{p}^2}{2m} \psi + V(r) \psi.
\end{equation}
求解此问题等价于求解坐标表象下的定态方程
\begin{equation}
    \frac{\hat{p}^2}{2m} \psi + V(r) \psi = - \frac{\hbar^2}{2m} \laplacian \psi + V(r) \psi = E \psi,
    \label{eq:centered-force-eq}
\end{equation}
使用球坐标系,以$\theta$为和$z$轴的夹角,$\varphi$为$x$-$y$平面上的转角,则
\[
    \hat{p}^2 = -\hbar^2 \laplacian = - \hbar^2 \left( \frac{1}{r^2} \pdv{r} r^2 \pdv{r} + \frac{1}{r^2 \sin \theta} \pdv{\theta} \sin \theta \pdv{\theta} + \frac{1}{r^2 \sin^2 \theta} \pdv[2]{\varphi} \right) .
\]
可以直接分离变量,但让我们首先采取一种物理意义比较明显的变形。注意到轨道角动量算符的平方为
\[
    \hat{L}^2 = \hat{\vb*{r}}^2 \hat{\vb*{p}}^2 - (\hat{\vb*{r}} \cdot \hat{\vb*{p}}) (\hat{\vb*{p}} \cdot \hat{\vb*{r}}),
\]
在球坐标系下$\vb*{r}$只在$r$方向上有分量,于是上式变成
\[
    \hat{L}^2 = r^2 \hat{p}^2 - r \left( - \hbar^2 \pdv[2]{r} \right) r,
\]
注意到(实际上这是动量-坐标对易关系的自然推论)
\[
    r \pdv[2]{r} r = \pdv{r} r^2 \pdv{r},
\]
这就给出了角动量长度平方的一个简洁的表达式:
\begin{equation}
    \hat{L}^2 = - \hbar^2 \left( \frac{1}{\sin \theta} \pdv{\theta} \sin \theta \pdv{\theta} + \frac{1}{\sin^2 \theta} \pdv[2]{\varphi} \right),
\end{equation}
当然也可以按照定义直接计算出这个表达式。相应的,哈密顿量在球坐标系下就是
\[
    \hat{H} = - \frac{\hbar^2}{2 m r} \pdv[2]{r} r + \frac{\hat{L}^2}{2 m r^2} + V(r).
\]
以上对哈密顿量的改写和经典情况可以类比。经典情况下,径向运动是有效势阱中的一维运动,且满足
\[
    \frac{p_r^2}{2m} + \frac{L^2}{2mr^2} - \frac{Ze^2}{4\pi \epsilon_0 r} = E,
\]
正好是将所有算符替换成实数之后的结果——本该如此。

现在,定态薛定谔方程成为
\[
    - \frac{\hbar^2}{2 m r} \pdv[2]{r} (r \psi) + \frac{\hat{L}^2}{2 m r^2} \psi + V(r) \psi = E \psi,
\]
算符$\hat{L}^2$仅仅和$\varphi$和$\theta$有关,因此我们可以将径向部分和角向部分做分离变量。
设
\[
    \psi = R(r) Y(\theta, \varphi),
\]
则径向部分的方程是
\begin{equation}
    \left( - \frac{\hbar^2}{2m r^2} \dv{r} r^2 \dv{r} + \frac{\hbar^2}{2m r^2} \alpha + V(r) \right) R = E R,
    \label{eq:original-r-equation}
\end{equation}
角向部分的方程为
\begin{equation}
    \hat{L}^2 Y = \alpha \hbar^2 Y.
    \label{eq:angle-equation}
\end{equation}
其中$\alpha$为常数。

\subsubsection{角向部分的方程的求解}

径向部分的方程包含一个可变的$V(r)$项,而角向部分的方程可以直接解出。
请注意角向部分实际上是一个拉普拉斯方程的角向部分,因此其解为球谐函数。
下面简单地展示其求解过程。设
\[
    Y(\theta, \varphi) = \Theta(\theta) \Phi(\varphi),
\]
代入
\[
    - \left( \frac{1}{\sin \theta} \pdv{\theta} \sin \theta \pdv{\theta} + \frac{1}{\sin^2 \theta} \pdv[2]{\varphi} \right) Y = \alpha Y,
\]
得到两个方程
\[
    \frac{1}{\Phi} \dv[2]{\Phi}{\varphi} = c,
\]
以及
\[
    - \left( \frac{1}{\sin \theta} \dv{\theta} \sin \theta \dv{\theta} + \frac{1}{\sin^2 \theta} c \right) \Theta = \alpha \Theta.
\]
$\Phi$必须是单值的,因为波函数应该是单值的,于是
\[
    c = m^2, \quad m = 0, 1, 2, \ldots,
\]
而
\[
    - \left( \dv{\cos \theta} (1 - \cos^2 \theta) \dv{\cos \theta} + \frac{1}{1 - \cos^2 \theta} m^2 \right) \Theta = \alpha \Theta.
\]
这是一个连带勒让德方程,为了保证$\Theta(\theta)$单值且有界,应有
\[
    \alpha = l(l+1), \quad l \geq \abs{m}, \quad l = 0, 1, 2, \ldots,
\]
于是\eqref{eq:angle-equation}的一组正交解为
\[
    Y(\theta, \varphi) = \ee^{\ii m \varphi} \legpoly_l^m(\cos \theta),
\]
其中$\legpoly_l^m$表示缔合勒让德多项式。归一化使用球坐标系的角向积分
\[
    \int \sin \theta \dd{\theta} \dd{\varphi},
\]
最后得到正交归一化的球谐函数
% TODO:(-1)^m因子?
\begin{equation}
    Y_{lm}(\theta, \varphi) = (-1)^m \sqrt{\frac{(2l+1)(l-\abs{m})!}{4\pi (l+\abs{m})!}} \legpoly_l^\abs{m} (\cos \theta) \ee^{\ii m \varphi}, \quad l = 0, 1, \ldots, m = \pm l, \pm (l-1), \ldots, 0 .
\end{equation}

球谐函数是$\hat{L}^2$的本征函数,但它带有两个量子数$l$和$m$,这意味着$\hat{L}^2$的本征函数存在简并;$l,m$分别标记了$\hat{L}^2$的本征值和某个不确定的可观察量的本征值。
注意到球坐标系中
\[
    \hat{L}_z = - \ii \hbar \pdv{\varphi},
\]
它正是$\Phi(\varphi)$满足的本征方程中的那个算符,因此$m$标记的是$\hat{L}_z$的本征值,$l$标记的是$\hat{L}^2$的本征值。
这是正确的,因为
\[
    m = \pm l, \pm (l-1), \ldots, 0
\]
正是角动量代数的重要性质。我们下面记此处的$m$为$m_l$,与自旋角动量在$z$方向上的取值$m_s = \pm \frac{1}{2}$相区分。
球谐函数$Y_{lm}(\theta, \varphi)$同时是$\hat{L}^2$和$L_z$的本征函数,相应的本征值为
\begin{equation}
    \hat{L}^2 Y_{lm} = l(l+1) \hbar^2 Y_{lm}, \quad \hat{L}_z Y_{lm} = m \hbar Y_{lm}.
\end{equation}

\subsubsection{量子数}\label{sec:quantum-number}

现在径向方程\eqref{eq:original-r-equation}成为了
\begin{equation}
    \left( - \frac{\hbar^2}{2m r^2} \dv{r} r^2 \dv{r} + \frac{\hbar^2}{2m r^2} l(l+1) + V(r) \right) R = E R.
    \label{eq:r-equation}
\end{equation}
这是一个单变量的本征值问题,它还会产生一个(而且也只有一个,因为一旦$E$确定了,$R$就确定了,不存在简并)量子数$n$,它标记不同的能量。
请注意由于角动量部分被分离变量出去了,实际上多出来了一个有效势$l(l+1)$项。

这样,\eqref{eq:centered-force-eq}有一组由$n, l, m_l$标记的正交归一化解,再考虑到自旋自由度,电子的状态就完全确定了。
通过求解过程可以看出前三个量子数标记了$\hat{H}, \hat{L}^2, \hat{L}_z$的本征值;
它们是通过分离变量求解坐标空间中的薛定谔方程得到的,因此随着时间变化,它们对应的物理量是守恒的且彼此对易,能够找到这样三个守恒且彼此对易的物理量当然是因为有心力系统的对称性。
目前尚未引入任何涉及自旋的机制,因此自旋也是恒定不变的。
于是我们找到了四个标记电子的好量子数:
\begin{enumerate}
    \item 主量子数$n$,它标记不同的能量,它是分立的,因为电子陷在一个势阱中,从而是离散谱;
    \item 角量子数$l$,它标记不同的角动量大小,对一部分势场形式,也参与标记不同的能量,它是分立的,因为波函数在角方向上是单值的;
    \item 磁量子数$m_l = 0, \pm 1, \ldots, \pm l$,它标记$z$轴上的角动量分量,同样也是分立的;
    \item 自旋量子数$m_s = \pm \frac{1}{2}$,它来自电子的内禀旋转自由度;自旋量子数和我们刚才讨论的轨道空间无关。
\end{enumerate}

这四个量子数直接决定了波函数的形状。主量子数决定了径向概率分布,角量子数和磁量子数决定了波函数的角向概率分布。
这四个量子数还可以确定其它一些量子数。例如,做宇称变换
\[
    (r, \theta, \varphi) \longrightarrow (r, \pi - \theta, \pi + \varphi),
\]
径向部分始终不变,若$l$为奇数则球谐函数会差一个负号,从而波函数为奇宇称,若$l$为偶数则球谐函数不变,从而波函数为偶宇称。

纯量子的理论展现出了和经典理论很不同的一些特性。请注意量子理论中电子可以完全没有角动量,这在经典理论下是不可能的——电子会直接落入有心力的力心,比如说原子核。
然而,哈密顿量\eqref{eq:columb-electron-hamiltonian}中各项不对易从而有量子涨落,因此如果角动量确定为零,那么电子的位置就不能够确定,因此电子并不会落入原子核。

\subsubsection{库伦势场}

库伦势场中的单个电子的哈密顿量为
\begin{equation}
    \hat{H} \psi = \frac{\hat{p}^2}{2m} \psi - \frac{Z}{4\pi \epsilon_0} \frac{e^2}{\abs{\vb*{r}}} \psi.
    \label{eq:columb-electron-hamiltonian}
\end{equation}
我们常常将这样的体系称为类氢原子,因为它和氢原子的结构除了$Z$可能不一样以外完全一致。
此时\eqref{eq:r-equation}为
\[
    \left( - \frac{\hbar^2}{2m r^2} \dv{r} r^2 \dv{r} + \frac{\hbar^2}{2m r^2} l(l+1) - \frac{1}{4\pi \epsilon_0} \frac{e^2}{r} \right) R = E R,
\]
显然这是一个束缚在势阱中的电子的方程,它必定有束缚态解,从而可以提供我们需要的主量子数。

% TODO:拉盖尔方程

\[
    l = 0, 1, 2, \ldots, n-1.
\]

可以依稀从量子力学中的氢原子看出一些经典的图像。在经典的原子模型中,粒子可以做椭圆运动,但是运动的能量仅仅关乎一个参数即椭圆的半长轴$a$,而和半短轴$b$无关;半短轴$b$则决定角动量等。
因此能量和角动量是分开的。角动量可以有不同的指向,因此角动量长度和它在$z$轴上的投影也没有必然的关系(当然,角动量在$z$轴上的投影不可能超过总的角动量长度)

求解出氢原子的薛定谔方程的完整解之后,可以发现主量子数为$n$的能级上有$n+1$个可能的角量子数,每个角量子数又允许$2l+1$个磁量子数,而最后还有两个自旋量子数,因此主量子数为$n$的能级上有
\[
    \frac{1}{2} (1 + (n-1)) n = \frac{1}{2}n^2
\]
个轨道,有$n^2$个电子。

$n-l$决定了径向峰值的数目

\subsection{跃迁和发光}

电子和电磁场耦合,因此可以在不同能级之间跃迁而发射或吸收光子。
跃迁包括受激跃迁(电子首先吸收光子,然后发生跃迁)以及自发跃迁(电子直接发生跃迁)。
对这一过程的完整计算涉及量子电动力学的束缚态,但通常对能标不是非常高的过程,使用量子化的原子和经典电动力学就足够计算一些问题。
受激跃迁只需要量子化原子加上经典电动力学即可完全解释,而自发跃迁不能使用经典电动力学解释而必须将光场量子化,因为“自发”意味着光场的真空涨落,这要使用量子理论处理。

\subsubsection{爱因斯坦的唯象理论}

考虑温度为$T$的空腔中有大量相同的原子,显然处于定态$i$和$j$的原子需要满足玻尔兹曼分布率
\[
    N_i \propto \ee^{-\frac{E_i}{k_\text{B} T}},
\]
或者写成
\[
    \frac{N_j}{N_i} = \ee^{-\hbar \omega_{ji} / kT}.
\]
能够达到热力学平衡意味着电子需要在不同能级之间跃迁。
电子和电磁场有耦合,因此电子在不同能级上跃迁确实是可以的。跃迁发生的机制可能有这么几种:
\begin{enumerate}
    \item 自发发射,也就是电子放出一个光子,跃迁到较低的能级;
    \item 受激发射,即电子先吸收一个光子再放出一个光子,然后发生跃迁;
    \item 吸收,即电子吸收一个光子然后跃迁到较高的轨道上。
\end{enumerate}
请注意受激发射和吸收可以使用量子化的原子和一个经典电磁场耦合来解释,但是自发发射在这个框架下是很难解释的,因为一个激发态的原子放在完全没有电磁场的空间内照样会有自发发射。
对这一现象的完整解释显然涉及真空涨落,因此需要量子电动力学。

设温度为$T$的光场中频率为$\omega$附近的能量密度为$u(\omega, T)$。设有两个能级$i$和$j$,且$E_j > E_i$。
我们假定(之后会通过量子力学严格证明)自发发射的跃迁率和$u$无关,而受激发射和吸收的跃迁率正比于$u(\omega_{ji}, T)$,其中
\begin{equation}
    \hbar \omega_{ji} = E_j - E_i,
    \label{eq:photon-energy}
\end{equation}
这样在这两个能级之间的自发发射、受激发射、吸收的跃迁率分别是
\[
    A_{ji} N_j, \quad B_{ji} N_j u(\omega_{ji}, T), \quad C_{ij} N_i u(\omega_{ji}, T).
\]
能级$i$向上跃迁到$j$的跃迁率为
\[
    \lambda_{ij} = C_{ij} u(\omega_{ji}, T),
\]
能级$j$向下跃迁到$i$的跃迁率为
\[
    \lambda_{ji} = B_{ji} u(\omega_{ji}, T) + A_{ji}.
\]
平衡时两者相等,即有
\[
    N_i C_{ij} u(\omega_{ji}, T) = N_j (B_{ji} u(\omega_{ji}, T) + A_{ji})
\]
$T \to \infty$,不同能级上原子分布的个数差别变得很小,$u \to \infty$,而上式仍然成立,因此$C_{ij} = B_{ji}$%
\footnote{请注意对温度的依赖被完全归入$u(\omega, T)$中,系数$C$和$B$由电子和光场耦合的方式决定,因此不依赖温度。}%
,这样就有
\[
    u(\omega_{ji}) = \frac{A_{ji} / B_{ji}}{\ee^{\omega_{ji} \hbar / k T} - 1}.
\]
由于原子能级可以随意调整,我们有
\[
    u(\omega) = \frac{A_{ji} / B_{ji}}{\ee^{\omega \hbar / k T} - 1}.
\]

\[
    u = \frac{\hbar \omega^3}{\pi^2 c^3} \frac{1}{\ee^{\hbar \omega / kT} - 1}
\]
自发发射跃迁率为
\[
    A_{ji} = \frac{\omega_{ji}^3}{3\pi \epsilon_0 \hbar c^3}
\]

\subsubsection{跃迁系数和选择定则}

\begin{equation}
    B_{ji} = \frac{4\pi^2 e^2}{3 \hbar^2} \abs{\expval*{\vb*{r}_{ji}}}^2,
\end{equation}
其中
\begin{equation}
    e \expval*{\vb*{r}_{ji}} = e \int \dd[3]{\vb*{r}} \psi_i^*(\vb*{r}) \vb*{r} \psi_j(\vb*{r})
    \label{eq:electro-dipole}
\end{equation}
为电偶极矩。也即,受激发射系数主要由电偶极矩确定。对波长非常短的光需要考虑电四极辐射。

\eqref{eq:electro-dipole}中的波函数对$(r, \theta, \varphi)$是分离变量的,而
\[
    \begin{aligned}
        r_x &= r \sin \theta \cos \varphi, \\
        r_y &= r \sin \theta \sin \varphi, \\
        r_z &= r \cos \theta 
    \end{aligned}
\]
也是分离变量的。
记$\psi_1$和$\psi_2$的量子数分别是$n_1, l_1, m_1$和$n_2, l_2, m_2$。
\eqref{eq:electro-dipole}给出非零结果的必要条件是其角部分均不为零。
在$\varphi$方向上,积分是
\[
    \int_0^{2\pi} \dd{\varphi} \ee^{ - \ii m_1 \varphi} \cos \varphi \ee^{\ii m_2 \varphi} \vb*{e}_x + \int_0^{2\pi} \dd{\varphi} \ee^{ - \ii m_1 \varphi} \sin \varphi \ee^{\ii m_2 \varphi} \vb*{e}_y + \int_0^{2\pi} \dd{\varphi} \ee^{ - \ii m_1 \varphi} \ee^{\ii m_2 \varphi} \vb*{e}_z,
\]
让三个分量不全为零的可能取值是:
\[
    m_2 - m_1 = \pm 1, 0.
\]
在$\theta$方向上,积分是
\[
    \int_0^\pi \dd{\theta} \sin \theta \legpoly_{l_1}^{m_1} (\cos \theta) \legpoly_{l_2}^{m_2} (\cos \theta) (\vb*{e}_x + \vb*{e}_y) + \int_0^\pi \dd{\theta} \cos \theta \legpoly_{l_1}^{m_1} (\cos \theta) \legpoly_{l_2}^{m_2} (\cos \theta) \vb*{e}_z,
\]
使用勒让德多项式的性质,可以证明让三个分量不全为零的可能取值为
\[
    l_2 - l_1 = \pm 1.
\]
总之,要让受激发射系数不为零,需要
\begin{equation}
    \Delta m = 0, \pm 1, \quad \Delta l = \pm 1.
\end{equation}
这就是\textbf{单电子原子跃迁的选择定则}。

实际上,也可以通过对称性分析得到这个结论。
% TODO:这个过程到底涉及多少光子??

不满足选择定则的跃迁称为\textbf{禁戒跃迁}。通过磁偶极跃迁、电四极子跃迁、双光子跃迁等方法,禁戒跃迁也是可以发生的,但是相对来说发生概率不大,从而对应的能级为亚稳态。

\subsection{磁性}

到目前为止的讨论,磁量子数$m_l$都是能量简并的,而如果加入一个磁场,那么就会有一个特定的空间方向,从而破缺$m_l$简并,导致能级进一步分裂。

\subsubsection{磁矩}

一些系统在外加静磁场时能量会增加一项
\[
    E_\text{M} = - \vb*{\mu} \cdot \vb*{B},
\]
其中的矢量$\vb*{\mu}$就称为磁矩。
磁矩和电荷的周期性运动具有非常密切的关系。一个没有内部结构的电荷做周期性运动相当于产生了一个环状电流,因此会产生一个磁矩,称为\textbf{轨道磁矩}。
首先采用经典理论分析轨道磁矩。电子轨道角动量的公式为
\[
    \vb*{L} = m_\text{e} \vb*{r} \times \dv{\vb*{r}}{t} = 2 m_\text{e} \dv{\vb*{S}}{t},
\]
电子被束缚在原子核周围时做平面周期性运动,这样它就产生了一个大小为
\[
    I = - \frac{e}{\tau}
\]
的电流,其中$\tau$是运动周期。
在电磁学中,一个电流为$I$,围绕的面积为$\vb*{S}$的平面线圈的磁矩为
\[
    \vb*{\mu} = I \vb*{S} = - \frac{e \vb*{S}}{\tau},
\]
而由于电子做周期性运动,由角动量守恒我们有
\[
    \dv{\vb*{S}}{t} = \frac{\vb*{S}}{\tau},
\]
这样轨道磁矩就是
\[
    \vb*{\mu} = - \frac{e}{2m_\text{e}} \vb*{L}.
\]
为了与原子物理的背景保持一致,引入下标$l$表示这是来自轨道角动量的磁矩,并且设
\begin{equation}
    \mu_\text{B} = \frac{e\hbar}{2m_\text{e}}
\end{equation}
称为\textbf{玻尔磁子},于是
\begin{equation}
    \vb*{\mu}_l = - \frac{\mu_\text{B}}{\hbar} \vb*{L}.
    \label{eq:orbit-magnetic-moment}
\end{equation}
虽然\eqref{eq:orbit-magnetic-moment}是在经典力学中导出的,但它也适用于量子理论。

量子理论中还有自旋角动量,这是不是会引入自旋磁矩?确实会,不过自旋磁矩的值和把电子当成带电小球计算出来的值并不相同。实际上,自旋磁矩是
\begin{equation}
    \vb*{\mu}_s = - \frac{e}{m_\text{e}} \vb*{S}.
\end{equation}
当然这也不奇怪,因为自旋在粒子图像中并没有经典对应。实际上自旋磁矩的严格计算直接来自QED。

总之,原子的总磁矩为
\begin{equation}
    \hat{\vb*{\mu}} = - \frac{\mu_\text{B}}{\hbar} (\vb*{L} + 2\vb*{S}),
\end{equation}
而对应的哈密顿量为
\begin{equation}
    \hat{H}_\text{mag} = - \hat{\vb*{\mu}} \cdot \vb*{B}.
    \label{eq:magnetic-hamiltonian}
\end{equation}

\subsubsection{半经典图像}

轨道角动量有明确的经典意义,可以使用半经典理论描述它。例如如果角动量的长度和$z$轴分量保持不变,那么就会发生\textbf{拉莫尔进动},即角动量矢量在一个对称轴就是$z$轴的锥面上运动,长度保持不变。

磁矩对电子运动的影响无非是让电子受力(即破缺平移不变性)或是受力矩(即破缺旋转不变性)。
如果磁场是均匀的,那么电子肯定不会受力,但会受到一个力矩,因为磁矩的方向的变动会让$\vb*{\mu} \cdot \vb*{B}$发生变化,即
\[
    \pdv{(\vb*{\mu} \cdot \vb*{B})}{\vb*{\varphi}} \neq 0,
\]
但是电子位置的变动当然不会让$\vb*{\mu}$发生任何变化。
如果磁场是不均匀的,那么磁矩不仅受到力矩还受力,因为空间平移不变性被破缺了,或者说
\[
    \pdv{(\vb*{\mu} \cdot \vb*{B})}{\vb*{r}} \neq 0.
\]

\subsubsection{朗道能级}

本节给出外加磁场的二维电子气的运动状况。我们采用\textbf{对称规范},设磁场方向为$z$方向,且设
\begin{equation}
    A_x = - \frac{1}{2} B y, \quad A_y = \frac{1}{2} B x,
\end{equation}
这样通过
\[
    \vb*{B} = \curl{\vb*{A}}
\]
就得到一个$z$方向上大小为$B$的磁场。系统的哈密顿量为
\begin{equation}
    \hat{H} = \frac{\hbar^2}{2m} \left( \grad - \frac{\ii e}{\hbar c} \vb*{A} \right)^2.
    \label{eq:magnetic-hamiltonian}
\end{equation}
我们把$\hbar$放回了定义中,因为本节将需要频繁地用到这个能量尺度。
\eqref{eq:magnetic-hamiltonian}中的导数被协变导数替代了,以满足$U(1)$对称性。
通过量纲分析可以发现\eqref{eq:magnetic-hamiltonian}中有一个特征长度
\begin{equation}
    l_0 = \sqrt{\frac{\hbar c}{e B}},
\end{equation}
相应的可以定义一个特征磁通量
\begin{equation}
    \Phi_0 = \frac{h c}{e} \sim 2 B \pi l_0^2.
\end{equation}
这样,两个方向上的协变导数为
\begin{equation}
    D_x = \partial_x + \frac{\ii}{2l_0^2} y, \quad D_y = \partial_y - \frac{\ii}{2l_0^2}x.
\end{equation}
哈密顿量就转化为
\[
    \hat{H} = - \frac{\hbar^2}{2m} (D_x^2 + D_y^2).
\]
容易验证$D_x$和$D_y$之间具有对易关系
现在定义
\begin{equation}
    D^\pm = D_x \pm \ii D_y,
\end{equation}
并定义复变量$z$为
\begin{equation}
    z = x + \ii y,
\end{equation}
则可以得到
\begin{equation}
    \comm*{D^-}{D^+} = - \frac{2}{l_0^2}.
\end{equation}
这意味着它们差一个常数就是一对升降算符。定义
\begin{equation}
    \hat{a} = \ii \frac{l_0}{\sqrt{2}} D^-,
\end{equation}
就有
\begin{equation}
    \hat{H} = \frac{\hbar^2}{m l_0^2} \left(\hat{a}^\dagger \hat{a} + \frac{1}{2} \right).
\end{equation}
因此系统具有分立的能级,称为\textbf{朗道能级(Landau Lowest Level, LLL)}。
回过头看,$\frac{\hbar^2}{m l_0^2}$实际上是经典图景中电子在磁场下做匀速圆周运动的圆频率。

实际上,朗道能级对应的波函数可以写成一个全纯函数乘以一个高斯因子。设$\Psi$是基态波函数,我们做拟设
\[
    \Psi(x, y) = f(z, z^*) \ee^{- z z^* / 4l_0^2},
\]
由于
\[
    D^- \Psi = 0,
\]
展开计算可以发现
\[
    \partial_{z^*} f = 0,
\]
这表明$f$一定是全纯函数,即得所求证。

朗道能级的简并度

注意到由于$\Psi(\vb*{r})$可以被恰当地归一化,它在电子气占据的范围内不应该有任何奇点,于是可以做泰勒展开
\[
    f(z) = c_0 + c_1 z + c_2 z^2 + \cdots,
\]
实际上可以证明,$\{z^m \ee^{- \abs{z}^2 / 4 l_0^2}\}$构成一组完备正交基。
现在寻找波函数幅值最大的位置,即计算
\[
    \pdv{\abs{\Psi_n(\vb*{r})}^2}{r} = 0
\]
的解,其解为
\[
    r_n = \sqrt{2n} l_0.
\]
换而言之我们得到了一组年轮一样的基矢量。设基态简并度为$n$则
\[
    S = \sqrt{2n} l_0,
\]
从而
\[
    n = S \cdot \frac{eB}{\hbar c} = \frac{\Phi}{\Phi_0}.
\]
这就是

\subsubsection{精细结构}

即使没有外加磁场,电子的自旋和轨道角动量仍然会出现小的耦合。这是相对论效应的结果。本节将给出对这一现象的一个半经典讨论。
为方便起见,以下称相对于原子实静止的参考系为$O$系,相对于某一时刻的电子静止的参考系(这仍然是一个惯性系,因为它并不是每一时刻都和电子保持静止)为$e$系。
设电子在$O$系中运行速度为$\vb*{v}$,$O$系中原子实施加给电子一个静电场$\vb*{E}_0$,则$e$系中$\vb*{E}_0$将变换成如下磁场:
\[
    \vb*{B}' = \frac{\vb*{E}_0 \times \vb*{v}}{\sqrt{c^2 - v^2}},
\]
由于$\vb*{v}$相对于光速很小,有
\begin{equation}
    \vb*{B}' = \frac{1}{c^2} \vb*{E}_0 \times \vb*{v}.
\end{equation}
$e$系中任何一个角动量的进动都是% TODO:经典力学
\[
    \vb*{\omega}' = \frac{e}{m_\text{e}} \vb*{B}',
\]
而在$O$系中$e$系的坐标轴以
\[
    \vb*{\omega}_T = - \frac{e}{2 m_\text{e}} \vb*{B}'
\]
的角速度进动,因此最后$O$系中任何一个角动量都在以
\begin{equation}
    \vb*{\omega} = \frac{e}{2 m_\text{e}} \vb*{B}'
\end{equation}
的角速度进动。这等价于$O$系中多出来了一个磁场,% 这也太扯了。。我觉得比较好的推导是,证明$e$系中哈密顿量会多出来一项,然后这一项和$O$系是相同的,当然哈密顿量未必是洛伦兹标量,所以也很麻烦。。。
因此需要在哈密顿量中引入一项
\begin{equation}
    \hat{H}_{LS} = - \vb*{\mu} \cdot \vb*{B}_\text{eff} = \frac{1}{2c^2} \frac{e}{m} \vb*{S} \cdot (\vb*{E}_0 \times \vb*{v}).
\end{equation}
如果是类氢原子,有
\begin{equation}
    \hat{H}_{LS} = \frac{Ze^2}{8 \pi \epsilon_0 c^2 m^2} \frac{\vb*{S} \cdot \vb*{L}}{r^3}.
    \label{eq:spin-ortibal-coupling}
\end{equation}
因此轨道角动量和自旋角动量是有耦合的,即所谓\textbf{自旋-轨道耦合},两者同向时能量较高,两者反向时能量较低。
这会导致谱线分裂,即所谓的\textbf{精细结构}。%
\footnote{但请注意分裂出的两个能级很难发生彼此之间的跃迁,因为两者的角量子数和磁量子数都一样,因此两者之间的跃迁违背选择定则。}%

\section{多电子系统}

多电子处于同一系统时会产生更多有趣的结果。由于电子是费米子,体系的波函数(同时包括轨道部分和自旋部分)一定是交换反对称的。
如果不考虑轨道-自旋耦合,为了保证反对称性,轨道部分对称则自旋部分反对称;轨道部分反对称则自旋部分对称。

\subsection{双电子原子}

\subsubsection{角动量代数}

首先考虑一个双电子原子。如果两个电子的轨道运动相同,那么轨道部分的波函数一定是对称的(如果是反对称的就变成零了),那么自旋部分的波函数一定是反对称的,并且两个电子的自旋一定不相同(否则所有状态都相同,违反泡利不相容原理)。
这样,自旋部分的波函数就是
\begin{equation}
    \chi = \frac{1}{\sqrt{2}} (\chi_{\uparrow 1} \chi_{\downarrow 2} - \chi_{\uparrow 2} \chi_{\downarrow 1}).
    \label{eq:asym-spin}
\end{equation}
不需要其它任何条件,自旋部分的波函数就完全确定了(可以差一个因子但这无关紧要)。
因此轨道部分相同的两个电子的自旋角动量代数是单态的,即$l=0, m=0$。

如果两个电子的轨道运动不同,那么轨道部分的波函数可以是对称的也可以是反对称的。
假定它是对称的,那么自旋部分的波函数一定反对称,因此两个电子的自旋不可能相等。这就意味着自旋部分的波函数还是\eqref{eq:asym-spin}。
而如果轨道部分的波函数是反对称的,那么自旋部分的波函数是对称的。下面我们考虑自旋本征态。
两个电子的自旋如果相等,那么自旋波函数就是以下二者之一:
\[
    \chi_{\uparrow 1} \chi_{\uparrow 2}, \quad \chi_{\downarrow 1} \chi_{\downarrow 2}.
\]
自旋波函数当然也可以是两个不同的自旋的线性组合,即它是$\chi_{\uparrow 1} \chi_{\downarrow 2}$及其交换的线性组合,并且满足对称条件,从而为
\[
    \chi = \frac{1}{\sqrt{2}} (\chi_{\uparrow 1} \chi_{\downarrow 2} + \chi_{\uparrow 2} \chi_{\downarrow 1}).
\]
因此自旋本征态为
\begin{equation}
    \chi = \chi_{\uparrow 1} \chi_{\uparrow 2}, \quad \chi_{\downarrow 1} \chi_{\downarrow 2}, \quad \frac{1}{\sqrt{2}} (\chi_{\uparrow 1} \chi_{\downarrow 2} + \chi_{\uparrow 2} \chi_{\downarrow 1}).
    \label{eq:sym-spin}
\end{equation}
\eqref{eq:sym-spin}是一个三重态,$l=1, m=0, \pm 1$。

从角动量代数的角度,两个电子放在一起,它们的角动量代数的复合是两个$l=1/2$的角动量代数的复合,所得结果的$l$取值范围为$0, 1$,和刚才推导得到的一致。
具体$l$取多少由轨道波函数的情况决定。轨道部分如果对称,那么自旋部分必须反对称%
\footnote{注意这是交换对称不是空间对称,空间对称由宇称描述。}%
,这对应$l=0$;反之,轨道部分反对称,则自旋部分必须对称,对应$l=1$。

\subsubsection{交换能}

双电子原子的波函数必须满足交换对称或者反对称条件还意味着,电子之间的库伦能也会发生改变。
将两个电子之间的库伦相互作用看成微扰,计算该微扰造成的能量本征值变化的一阶修正,就是
\begin{equation}
    E = \int \dd[3]{\vb*{r}_1} \dd[3]{\vb*{r}_2} \psi^*(\vb*{r}_1, \vb*{r}_2) \frac{1}{4\pi \epsilon_0} \frac{e^2}{\abs{\vb*{r}_1 - \vb*{r}_2}} \psi(\vb*{r}_1, \vb*{r}_2).
\end{equation}
对对称态或者反对称态
\[
    \psi(\vb*{r}_1, \vb*{r}_2) = \frac{1}{\sqrt{2}} (\psi_1(\vb*{r}_1) \psi_2(\vb*{r}_2) \pm \psi_2(\vb*{r}_1) \psi_1(\vb*{r}_2)),
\]
我们有
\begin{equation}
    \begin{aligned}
        E &= \int \dd[3]{\vb*{r}_1} \dd[3]{\vb*{r}_2} \psi_1^*(\vb*{r}_1) \psi_1(\vb*{r}_1) \frac{1}{4\pi \epsilon_0} \frac{e^2}{\abs{\vb*{r}_1 - \vb*{r}_2}} \psi_2^*(\vb*{r}_2) \psi_2(\vb*{r}_2) \\
        &\pm \int \dd[3]{\vb*{r}_1} \dd[3]{\vb*{r}_2} \psi_2^*(\vb*{r}_1) \psi_1(\vb*{r}_1) \frac{1}{4\pi \epsilon_0} \frac{e^2}{\abs{\vb*{r}_1 - \vb*{r}_2}} \psi_1^*(\vb*{r}_2) \psi_2(\vb*{r}_2).
    \end{aligned}
\end{equation}
等式右边第一项就是将电子云密度看成电荷密度计算出来的库伦能,第二项则是一个没有经典对应的项,称为\textbf{交换能}。
可以看到在两个电子的波函数没有很大重叠时交换能可以略去,这也是合理的。

交换能意味着两个电子的自旋角动量发生了耦合。要看出这是为什么,定义
\begin{equation}
    J = \int \dd[3]{\vb*{r}_1} \dd[3]{\vb*{r}_2} \psi_2^*(\vb*{r}_1) \psi_1(\vb*{r}_1) \frac{1}{4\pi \epsilon_0} \frac{e^2}{\abs{\vb*{r}_1 - \vb*{r}_2}} \psi_1^*(\vb*{r}_2) \psi_2(\vb*{r}_2),
\end{equation}
若$l=1$则轨道部分反对称,交换能为$-J$,若$l=0$则轨道部分对称,交换能为$J$。这样,交换能在哈密顿量中就引入这样一项:
\[
    \hat{V}_\text{ex} = -J ( \dyad{\uparrow \uparrow} + \dyad{\downarrow \downarrow} + \frac{1}{2} (\ket{\uparrow \downarrow} + \ket{\downarrow \uparrow}) (\bra{\uparrow \downarrow} + \bra{\downarrow \uparrow}) ) + J \frac{1}{2} (\ket{\uparrow \downarrow} - \ket{\downarrow \uparrow}) (\bra{\uparrow \downarrow} - \bra{\downarrow \uparrow}),
\]
可以验证,这实际上就是
\begin{equation}
    \hat{V}_\text{ex} = - \frac{1}{2} J (1 + 4 \hat{\vb*{s}}_1 \cdot \hat{\vb*{s}}_2),
\end{equation}
现在我们看出,交换能实际上会让电子的自旋倾向于趋于一致。

实际上可以看到,只要两个电子之间有相互作用,就肯定会有交换相互作用。

\subsection{中心场近似下的多电子原子}

本节中单电子物理量用小写,整个原子的物理量(通常就是对应的单电子物理量之和)用大写。

\subsubsection{中心场近似}\label{sec:centric-field}

将一个电子受到其它电子的作用看成一个平均场,即认为原子核受到的屏蔽作用是固定不变的。对称性分析表明这个平均场一定是一个有心力场$S(r)$,因此称之为\textbf{中心场}。
除了中心场以外的相互作用称为\textbf{剩余相互作用}。

在哈密顿量的势能项当中加入中心场之后,会发现能量和角量子数有关,这是因为$n$相同$l$不同的原子径向分布不同,因此受到的屏蔽也不同。角动量大的电子近核概率小,屏蔽效应强,能量高。
$n, l$完全决定了波函数的径向部分。(见\autoref{sec:quantum-number})

原子中所有电子在单电子能级上的分布情况称为\textbf{电子组态},它给出了全部电子的能级的组合,也即,给出了$n$和$l$的组合。$n$和$l$相同的电子称为\textbf{同科电子}。
不需要知道完整的电子状态就可以得到电子组态。
电子组态可以使用标准的spdf记号给出。
我们称不同的主量子数对应的全部电子组成一个\textbf{壳层}。$n=1, 2, 3, \ldots$对应着K,L,M,N,O,P等壳层,但是现在很少用这些字母符号了。
在每一个壳层内部,不同的角量子数$l$给出不同的\textbf{支壳层}。
在中心场近似下,每个支壳层内部的电子能量都是一样的。
支壳层内部的$m_l$可以发生变化,每个$m_l$给出一个\textbf{轨道},每个轨道容纳自旋不同的两个电子。%
\footnote{当然,这是以$n,l,m_l,m_s$为好量子数之后的半经典叙述。实际上电子可以处于这组表象下的叠加态。}%
$l$支壳层有$2(2l+1)$个电子,即$2l+1$个轨道,$2$个自旋,因此$n$壳层有
\[
    \sum_{l=0}^{n-1} 2(2l+1) = 2n^2
\]
个电子。
外层电子的能量主要由$l$决定;这就导致了所谓的能级交错现象,即主量子数小的支壳层如果角量子数适当能量反而比较大。\textbf{洪特规则}给出了不同支壳层能量的大小顺序。

以上图景可以解释一些实验中观察到的现象。
首先是\textbf{原子幻数},即
\[
    Z=2, 10, 18, 36, 54, 86, \ldots
\]
时第一电离能位于峰值,然后一下子到达谷值。原子序数为原子幻数的元素即为稀有气体。
稀有气体非常稳定是因为它有电子的最高的能级都是$n$p支壳层,且全满,而$n$p与$(n+1)$s有较大能隙,p支壳层全满的原子不容易激发;此外内满壳层电子云的电荷分布球对称,对价电子吸引强。
为什么碱金属容易电离是因为原子实中电荷均匀球对称分布,几乎就是一个单独的正电荷,因此价电子受到的束缚非常弱。
同理为什么卤素容易接受电子是因为容易失去一个空穴。

\[
    \sum_{m=-l}^l \abs{Y_{lm}(\theta, \varphi)}^2 = \frac{2l+1}{4\pi},
\]
从而
\[
    \rho(\vb*{r}) = -2(2l+1) \frac{e}{4\pi} \chi_{nl}(\vb*{r})
\]

\subsubsection{化学反应}

吸能:失去电子

放能:得到电子、正负电荷中心接近从而降低库伦能

\subsubsection{原子态和光谱}

由于能量和$m_l$、$m_s$无关,如果一个支壳层非空而非全满,那么就有能量简并。$C_{2(2l+1)}^N$

\subsection{相互作用带来的能量修正}

中心场近似中各个电子之间没有相互作用。实际上,各个电子之间有两种主要的相互作用。
其一是\textbf{剩余相互作用}$\hat{H}_1$,即中心场以外的电子间剩余库伦作用(交换能、关联能等全部被收入这一项),其二是自旋轨道耦合$\hat{H}_2$。
这两者分别可以写成
\begin{equation}
    \hat{H}_1 = \frac{1}{2} \sum_{\vb*{r}_1, \vb*{r}_2} \int \dd[3]{\vb*{r}_1} \dd[3]{\vb*{r}_2} \frac{1}{4 \pi \epsilon_0} \frac{e^2}{\abs{\vb*{r}_1 - \vb*{r}_2}} - S(r),
\end{equation}
以及
\begin{equation}
    \hat{H}_2 = \sum_{i=1}^N \xi(\vb*{r}_i) \vb*{l}_i \cdot \vb*{s}_i.
    \label{eq:spin-orbital-many}
\end{equation}
可以估计出
\[
    H_1 \sim Z, \quad H_2 \sim Z^4,
\]
且电子相距越近,剩余相互作用越明显。下面我们会发现,如果剩余相互作用远大于自旋轨道相互作用,则会导致L-S耦合,反之会导致j-j耦合,因此大部分元素的基态、轻元素的低激发态适用L-S耦合,重元素的激发态适用j-j耦合。
在$\hat{H}_1$和$\hat{H}_2$同阶时,会出现中间耦合,这适用于轻元素的高激发态和中等元素的激发态。

\subsubsection{L-S耦合}

如果剩余相互作用的修正远大于自旋-轨道相互作用,就首先考虑$\hat{H}_1$的作用。此时两个电子的自旋角动量之间存在耦合,从而不同电子的轨道角动量之间也存在耦合,但是自旋角动量和轨道角动量之间尚无耦合,这称为\textbf{L-S耦合}。
这样,在加入了$\hat{H}_1$之后的好量子数由$\hat{H}_0, \hat{L}^2, \hat{L}_z, \hat{S}^2, \hat{S}_z$给出,它们具有共同本征函数系。
于是把$\{n_i l_i m_{li} m_{si}\}$表象线性变换成$\{n_i l_i\} L M_L S M_S$表象,$\hat{H}_1$带来的修正在这一组表象下应当是对角的。%
\footnote{由于多个电子角动量合成之后的角动量代数是可约的,即使每个电子的$l_i$都知道了,还是必须明确给出$L$和$S$,从经典图像上说,这是因为我们不知道各个电子的角动量的指向如何。}%
这样,把$\hat{H}_1$当成微扰项,那么能量的一阶修正为
\begin{equation}
    E^{\text{rem}, (1)}_{\{n_i l_i\} LS} = \sum_{s_{z1}, s_{z2}, \ldots, s_{zN}} \int \prod \dd[3]{\vb*{r}_i} \Psi^{(0) *}_{\{n_i l_i\} L M_L S M_S}(\{\vb*{r}_i\}) \hat{H}_1 \Psi^{(0)}_{\{n_i l_i\} L M_L S M_S}(\{\vb*{r}_i\}).
    \label{eq:ls-1}
\end{equation}
中心场近似的能量$E^{(0)}_{\{n_i l_i\}}$仅仅和原子组态有关;实际上,$E^{(1)}_{LS}$只和原子组态以及$L,S$有关。
这是系统的对称性决定的,加入$\hat{H}_1$之后系统同时具有自旋旋转不变性(剩余相互作用中的交换能部分)和轨道旋转不变性(剩余相互作用中的库伦能部分),因此$M_L$和$M_S$对能量没有影响。
这样在只考虑剩余相互作用时,能级简并度为
\begin{equation}
    g_{\{n_i l_i\}LS} = (2L+1)(2S+1).
\end{equation}
从经典图景的角度说,这是因为剩余库伦能和剩余交换能分别由电子的相对角分布和自旋角动量的夹角决定,至于它们绝对地指向什么方向,并不重要。
在经典图景下各个$\{\vb*{l}_i\}$绕着$\vb*{L}$进动,为了把所有量确定下来要知道$\{\vb*{l}_i\}$的大小、$\vb*{L}$的大小和方向;自旋角动量同理。

现在在L-S耦合的系统中再引入$\hat{H}_2$,此时$L_z$和$S_z$也不再是守恒的了,但由于$\hat{H}_2$只是让角动量在自旋和轨道两种形式之间转换,总角动量在$z$轴的分量$J_z$还是守恒的,且$J^2$也是守恒的。同样$L^2$和$S^2$也是守恒的,这样好量子数是$\hat{H}_0, \hat{L}^2, \hat{S}^2, \hat{J}^2, \hat{J}_z$。
这样需要把$\{n_i l_i\} L M_L S M_S$表象再次变换成$\{n_i l_i\} L S J M_J$表象(这组表象由于$L,S$确定,\eqref{eq:ls-1}还是适用的),$\hat{H}_2$引入的修正在这一组表象下是对角的,为
\begin{equation}
    \begin{aligned}
        E^{\text{soi}, (1)}_{\{n_i l_i\} LSJ} &= \sum_{s_{z1}, s_{z2}, \ldots, s_{zN}} \int \prod \dd[3]{\vb*{r}_i} \Psi^{(0) *}_{\{n_i l_i\} L S J M_J}(\{\vb*{r}_i\}) \hat{H}_2 \Psi^{(0)}_{\{n_i l_i\} L S J M_J}(\{\vb*{r}_i\}) \\
        &= \frac{\zeta_{\{n_i l_i\}LS}}{2} \hbar^2 (J(J+1) - L(L+1) - S(S+1)).
        \label{eq:ls-2}
    \end{aligned}
\end{equation}
第二个等号看起来很奇怪不过下面马上会解释它。
由于空间旋转对称性,上式和$M_J$无关。这样,完整地考虑了两种相互作用带来的修正之后,能量修正为
\begin{equation}
    E^{\text{LS}, (1)}_{\{n_i l_i\} LSJ} = E^{\text{rem}, (1)}_{\{n_i l_i\} LS} + E^{\text{soi}, (1)}_{\{n_i l_i\} LSJ},
\end{equation}
它只和$L,S,J$有关,和$M_J$无关,于是能级简并度为
\begin{equation}
    g_{\{n_i l_i\}LSJ} = 2J + 1.
\end{equation}
从经典图景看,各个电子的$\vb*{l}_i$绕着$\vb*{L}$进动,$\vb*{s}_i$绕着$\vb*{S}$进动,而$\vb*{L}$和$\vb*{S}$又绕着$\vb*{J}$进动,由于$\hat{H}_1$远大于$\hat{H}_2$,前者远远快于后者。
这个经典图景有助于理解\eqref{eq:ls-2}的形式:由于$\vb*{l}_i$和$\vb*{s}_i$的进动非常快,可以做近似
\[
    H_2 \approx \bar{H}_2,
\]
从而
\[
    H_2 = \expval{\sum_{i=1}^N \left(\vb*{l}_i \cdot \frac{\vb*{L}}{L} \right) \frac{\vb*{L}}{L} \cdot \left(\vb*{s}_i \cdot \frac{\vb*{S}}{S} \right) \frac{\vb*{S}}{S} } \propto \vb*{L} \cdot \vb*{S} = \frac{1}{2} (J^2 - L^2 - S^2),
\]
这就导致了\eqref{eq:ls-2}的形式。$\zeta_{\{n_i l_i\}LS}$是$\vb*{l}_i$和$\vb*{s}_i$导致的比例系数。

总之,L-S耦合中,能级分裂情况如下面的列表所示:
\begin{enumerate}
    \item 忽略电子间库仑相互作用,得到类氢原子近似,能量完全由主量子数$n$确定;
    \item 引入中心场近似,能量由$n, l$决定,出现能级交错和洪特规则,可以解释稀有气体、碱金属和卤素的性质;
    \item 考虑剩余相互作用,相同组态的能量发生分裂,不同的$L$和$S$之间能量不同,例如各个电子自旋同向的状态能量会低一些;
    \item 考虑轨道自旋耦合,不同的$J$会带来不同能量,这就是类氢原子的精细结构在多电子原子中的对应。
\end{enumerate}
这样,对一个组态已知的原子系统,$L, S, J$完全决定其能量,记这样的能级(或者说\textbf{谱项})为$^{2S+1} L_J$,$2S+1$称为\textbf{自旋多重度},因为它给出了自旋量子数$M_S$的取值个数。
将所有的$\{l_i\}$合成可以求出$L$的取值范围,将所有的$\{s_i\}$合成可以求出$S$的取值范围,将$L$和$S$合成又可以得到$J$的取值范围。
需要注意的是不是所有的$\{s_i\}$都是可能的,因为必须保持多电子波函数反对称。
精细结构的能级差为
\begin{equation}
    E_{\{n_i l_i\} LSJ} - E_{\{n_i l_i\} LS(J-1)} = J \zeta_{\{n_i l_i\}} \hbar^2.
\end{equation}
这就是\textbf{朗德间隔定则}:如果原子遵循L-S耦合则它成立,如果原子不遵循L-S耦合,它通常会被违反。

使用所有$\{l_i\}$合成出$L$是非常繁琐的。
在合成时,我们总是可以首先把一个满支壳层中的所有电子的角动量代数合成,而单个电子无论是磁量子数还是自旋量子数都可以跑遍$\pm l$或$\pm \frac{1}{2}$中的全部整数或半整数值,它们求和会得到零,因此一个满支壳层中所有电子的角动量合成之后,$M_L$和$M_S$一定是零,从而一个满支壳层的$L$和$S$只能是零。因此实际上满支壳层对整个原子的$L$和$S$是没有贡献的。

\subsubsection{j-j耦合}

如果自旋-轨道相互作用的修正远大于剩余相互作用,就需要首先考虑$\hat{H}_2$,此时每个电子的自旋角动量和轨道角动量有很强的耦合,但是不同电子的总角动量之间并没有很强的耦合,这称为\textbf{j-j耦合}。
在加入$\hat{H}_2$之后,电子之间还是没有相互作用,因此我们可以只讨论单个电子的运动情况。
单个电子的哈密顿量包括两项,其一是中心场近似下的
\[
    \hat{h}_0 = \frac{\hat{p}^2}{2m_\text{e}} - \frac{Z e^2}{4\pi \epsilon_0 r} + S(\vb*{r}),
\]
其二是\eqref{eq:spin-orbital-many}导致的
\[
    \hat{h}_2 = \xi(\vb*{r}) \hat{\vb*{l}} \cdot \hat{\vb*{s}}.
\]
和上一节中讨论$\hat{H}_2$的作用时类似,一组好量子数由$\hat{h}_0, \hat{l}^2, \hat{s}^2, \hat{j}^2, \hat{j}_z$给出,于是从$nl m_l m_s$表象切换到$nlj m_j$表象,且记径向部分为$R_{nl}(\vb*{r})$(在中心场近似中,给定组态,各个电子的径向分布就给定了),则$\hat{h}_2$引入的能量修正为
\begin{equation}
    \begin{aligned}
        \epsilon_{nlj}^{\text{soi}, (1)} &= \sum_{s_z} \int \dd[3]{\vb*{r}} \psi^{(0)*}_{nlj m_j} \hat{h}_2 \psi^{(0)}_{nlj m_j} \\
        &= \frac{\xi_{nl}}{2} \hbar^2 (j(j+1) - l(l+1) - s(s+1)),
    \end{aligned}
\end{equation}
其中
\begin{equation}
    \xi_{nl} = \int_0^\infty \dd{r} (r R_{nl}(r))^2 \xi(r).
\end{equation}
% TODO:\xi(r)为什么没有角向部分?
由于电子间没有相互作用,这就意味着原子总能量为
\[
    E_{\{n_i l_i j_i\}} = \sum_{n, l, j} N_{nlj} \epsilon_{nlj},
\]
其中$N_{nlj}$为处于状态$(n, l, j)$的电子数目,由电子组态$\{n_i l_i\}$给定。这样$\hat{H}_2$对原子总能量的一阶修正就是
\begin{equation}
    E^{\text{soi}, (1)}_{\{n_i l_i j_i\}} = \sum_{n, l, j} N_{nlj} \epsilon_{nlj}^{\text{soi}, (1)}.
\end{equation}
能量和$m_j$没有任何关系,因此有能级简并。能级简并数为
\begin{equation}
    g_{\{n_i l_i j_i\}} = \prod_{n, l, j} C_{2j+1}^{N_{nlj}}.
\end{equation}
从经典图景看,每个电子的$\vb*{l}$和$\vb*{s}$绕着$\vb*{j}$进动。

现在再讨论剩余相互作用带来的修正。加入剩余相互作用之后,诸$\{m_{ji}\}$不再是好量子数,因为不同电子的角动量会有耦合,不过$\vb*{J}$还是守恒的,且$j^2$也还是守恒的,于是好量子数由$\hat{H}_0, \hat{J}^2, \hat{J}_z$给出。
从$\{n_i l_i j_i m_{ji}\}$表象(每个$i$对应一个单电子的$nl j m_j$表象)切换到$\{n_i l_i j_i\} J M_J$表象,$\hat{H}_1$带来的一阶修正为
\begin{equation}
    E^{\text{rem}, (1)}_{\{n_i l_i j_i\} J} = \sum_{s_{z1}, s_{z2}, \ldots, s_{zN}} \int \prod \dd[3]{\vb*{r}_i} \Psi^{(0) *}_{\{n_i l_i j_i\} J M_J}(\{\vb*{r}_i\}) \hat{H}_1 \Psi^{(0)}_{\{n_i l_i j_i\} J M_J}(\{\vb*{r}_i\}).
\end{equation}
同样空间旋转对称性意味着$M_J$对能量修正没有影响。于是同时考虑了两种相互作用带来的修正,我们有
\begin{equation}
    E^{\text{jj}, (1)}_{\{n_i l_i j_i\} J} = E^{\text{soi}, (1)}_{\{n_i l_i j_i\}} + E^{\text{rem}, (1)}_{\{n_i l_i j_i\} J},
\end{equation}
能量简并度为
\begin{equation}
    g_{\{n_i l_i j_i\}J} = 2J+1.
\end{equation}
从经典图景看,此时每个电子的$\vb*{l}$和$\vb*{s}$绕着$\vb*{j}$进动,各个$\vb*{j}_i$绕着$\vb*{J}$进动。前者明显快于后者。

j-j耦合中,能级分裂情况由以下列表所示:
\begin{enumerate}
    \item 忽略电子间库仑相互作用,得到类氢原子近似,能量完全由主量子数$n$确定;
    \item 引入中心场近似,能量由$n, l$决定,出现能级交错和洪特规则,可以解释稀有气体、碱金属和卤素的性质;
    \item 引入轨道自旋耦合,能量由$\{n_i l_i j_i\}$确定,这是类氢原子的剩余相互作用在多电子原子中的对应,这一步造成的能级分裂的能量大小顺序由$\{j_i\}$决定,$j_i$越大,说明方向相同的自旋角动量和轨道角动量越多,因此能量越高;
    \item 引入剩余相互作用,能量由$\{n_i l_i j_i\} J$确定,这一步中的能级分裂次序没有特别的规律。
\end{enumerate}

j-j耦合中原子组态已知时谱项表示为$(j_1, j_2, \ldots, j_N)_J$。
获得谱线的方式是先将$l$和$s$合成出$j$,然后再把各个$\{j_i\}$合成成$J$。
第一步是非常显然的,第二步则比较繁琐。实际上,此时满支壳层对$J$同样没有贡献。与L-S耦合类似,我们将一个满支壳层中的电子首先来做合成,则
\[
    M = \sum_{j=\abs{l-1/2}}^{l+1/2} \sum_{m_j=-j}^j m_j = 0,
\]
因此满支壳层的角动量代数的磁量子数唯一的取值是零,因此$J=0$,即这个角动量代数对整个原子的总角动量代数没有贡献。
实际上,使用类似的方法,可以证明满支壳层带来的自旋-轨道修正也是零,因为能量修正有的为正有的为负,加起来等于没修正。

\subsection{多电子原子的偶极跃迁}

\subsubsection{选择定则}

设电偶极跃迁光子的总角动量为$\vb*{j}_\gamma$,显然有
\[
    \vb*{J} + \vb*{j}_\gamma = \vb*{J}'.
\]
对光子$j_\gamma=1$,则
\[
    \Delta J = \pm 1, 0, 
\]
而且$J$和$J'$不能同时为零,否则角动量不可能守恒。
同样,磁量子数的变化为%
\footnote{这里可能会遇到一个疑难:光子的自旋角动量只在其前进方向上有投影,且只有$\pm 1$两种取值。
然而,光子的前进方向和我们选取的电子$z$方向未必相同,因此光子的自旋角动量投影在$z$方向上还是会有$0, \pm 1$三种取值。
}%
\[
    \Delta M_J = \pm 1, 0.
\]
同时我们还有宇称守恒,而光子具有奇宇称,而原子的宇称为
\[
    \Pi = (-1)^{\sum_{j} l_j},
\]
因此
\[
    \Delta L = \pm 1, \pm 3, \ldots,
\]
对低激发态,这意味着%
\footnote{通常只考虑低激发态的原因是,如果一份能量足够让原子的多个电子被激发,那也足够让单个电子被电离。
后者是更为常见的现象。}%
\[
    \Delta l_\text{trans} = \pm 1, \quad \Delta l_\text{other} = 0.
\]
无论如何,总角量子数发生变化意味着电偶极跃迁发生在不同组态的谱项之间。
同一组态中的谱项之间不能发生电偶极跃迁,因为轨道角动量没有发生变化,从而不满足宇称守恒。
这样就得到了任何一个多电子原子的偶极跃迁应遵循的选择定则:
\begin{equation}
    \Delta J = 0, \pm 1, \quad \Delta L = \pm 1, \pm 3, \ldots, \quad \Delta M_J = \pm 1, 0.
    \label{eq:many-electron-selective}
\end{equation}

除了以上规则,通过计算跃迁矩阵元,还可以发现一些选择定则(称为附加定则)。对L-S耦合,有
\begin{equation}
    \Delta S = 0, \quad \Delta L = \pm 1.
    \label{eq:l-s-selective}
\end{equation}
对j-j耦合,有
\begin{equation}
    \Delta j_\text{trans} = \pm 1, 0, \quad \Delta j_\text{other} = 0.
    \label{eq:j-j-selective}
\end{equation}

\subsubsection{类氢光谱}

考虑一个类氢原子(即氢原子或者碱金属原子),其基态价壳组态为$n$s。
考虑低激发态,即只有价电子跃迁。这样,角动量——无论是自旋还是轨道——完全来自价电子。

首先采用L-S耦合,则
\[
    L = l, \quad S = s = \frac{1}{2}, 
\]
而
\[
    J = j = \begin{cases}
        l \pm 1/2, &\quad l \neq 0, \\
        1/2, &\quad l = 0.
    \end{cases}
\]
应用L-S耦合的选择定则\eqref{eq:l-s-selective}和\eqref{eq:many-electron-selective},我们有
\[
    \Delta j = 0, \pm 1, \quad \Delta m_j = 0, \pm 1, \quad \Delta s = 0, 
\]
容易看出这正是单电子的选择定则。使用j-j耦合也可以得到同样的结果。

如果是L-S耦合,我们可以根据$nl$分析会有哪些谱项。
\begin{enumerate}
    \item 如果$l=0$,即价电子占据支壳层$n$s,那么$J=1/2$,于是谱项为\lsterm{2}{S}{1/2};
    \item 如果$l=1$,即价电子占据支壳层$n$l,那么$J=1/2, 3/2$,谱项为\lsterm{2}{P}{1/2}和\lsterm{2}{P}{3/2},后者自旋和轨道角动量平行,由\eqref{eq:spin-ortibal-coupling}可以看出后者能量高于前者,这和单电子的精细结构来自同样的物理机制;
    \item $l=3$,按照以上步骤可以得到两个谱项\lsterm{2}{D}{3/2}和\lsterm{2}{D}{5/2},后者能量高于前者;
    \item $l=4$,得到\lsterm{2}{F}{5/2}和\lsterm{2}{F}{7/2}。更高能量的谱项暂不考虑。
\end{enumerate}

现在以钠为例分析可以有哪些跃迁。首先同一$l$的谱项肯定不能相互跃迁。
下面列举了一些常见的跃迁,同一类型的跃迁导致的谱线称为一个\textbf{线系}。

\begin{itemize}
    \item 会跃迁到基态3s上的只有$l=1$的谱项,即从3p,4p,5p等跃迁到3s上,这一组谱线都是双线(因为$l=1$有精细结构谱项分裂),其中3p到3s就是钠双黄线。
    这一线系称为\textbf{主线系(principal series)}。随着波长变短,精细结构导致的波长分裂也会变小。
    这就是将主线系的出发态$l=1$命名为p态的原因。
    \item 4s,5s,6s等谱线可以跃迁到3p,从一个单能级跳到双能级,从而导致一组双线,这组双线不同波长的两条谱线距离相等(就是\lsterm{2}{P}{1/2}和\lsterm{2}{P}{3/2}的差距),非常清晰,称为\textbf{锐线系(sharp series)}。这就是锐线系的出发态$l=0$名为s态的原因。
    \item d态也可以跃迁到p态。d和p都有精细结构分裂,但由于$\Delta j$最大取到1,这实际上是一个三线系,称为\textbf{漫线系(diffuse series)}。这就是漫线系的出发态$l=2$称为d态的原因。
    \item 与漫线系类似,f态可以跃迁到d态,产生一个三线系,称为\textbf{基线系(fundamental series)}。这就是基线系的出发态$l=3$称为f态的原因。
\end{itemize}

其余的谱线都发生在比较高的能级之间,不容易观察到。

以上提到的都是发射光谱,但显然也可以把产生它们的过程倒转过来而得到吸收光谱。
由于基态为3s,吸收光谱中通常只能看到主线系。

\subsubsection{类氦光谱}

下面讨论类氦原子,即基态价壳组态为$n$s$^2$的原子,包括氦原子和碱土金属原子。
低激发态只有一个价电子发生跃迁。

使用L-S耦合。设跃迁价电子的角量子数为$l$,则$L=l$。两个电子给出$S=0, 1$,即有自旋单态和三重态。
附加选择定则\eqref{eq:l-s-selective}要求$\Delta S=0$,因此三重态和单态之间的跃迁是禁戒的——这样一来,如果只考虑偶极辐射,三重态的原子永远处于三重态,单态的原子永远处于单态。
我们将三重态的原子称为\textbf{正氦},将单态的原子称为\textbf{仲氦}。
三重态原子的自旋波函数是对称的,因此轨道波函数必须反对称,因此两个价电子不能出现在同一个轨道上。
由于仅考虑低激发态即只有一个价电子,两个价电子出现在同一轨道上只可能意味着电子组态为基态组态,即$n$s$^2$。
由于自旋平行会让能量降低,同一电子组态的正氦能量低于仲氦。
总之,正氦没有$n$s$^2$价电子组态,且同一电子组态的能力低于仲氦。

基于L-S耦合,仲氦的谱项列举如下:(以下$n=1$指的是价壳层)
\begin{enumerate}
    \item \lsterm{1}{S}{0},$n=1, 2, ,\ldots$;
    \item \lsterm{1}{P}{1},$n=2, 3, \ldots$;
    \item \lsterm{1}{D}{2},$n=3, 4, \ldots$;
    \item \lsterm{1}{F}{3},$n=4, 5, ,\ldots$。
\end{enumerate}
而正氦的谱项列举如下:
\begin{enumerate}
    \item \lsterm{3}{S}{1},$n=2, 3,\ldots$;
    \item \lsterm{3}{P}{0,1,2},$n=2, 3, \ldots$;
    \item \lsterm{3}{D}{1,2,3},$n=3, 4, \ldots$;
    \item \lsterm{3}{F}{2,3,4},$n=4, 5, ,\ldots$。
\end{enumerate}
请注意正氦没有$n=1$的态;此外除了S态以外,正氦的每个电子组态均存在关于$J$的能级三重分裂,$J$越大能量越高。

下面分别分析正氦和仲氦的光谱,由选择定则还是会有主线系、漫线系、锐线系、基线系。
对仲氦,所有这些线系都是单线系。
对正氦,主线系、锐线系是三线系,基线系、漫线系是三线系。

正氦和仲氦还有亚稳态。对仲氦,\lsterm{1}{S}{0}是亚稳态,因为低于它的能级只有基态,但从它到基态的过程$\Delta L = 0$;同样对正氦,\lsterm{3}{S}{1}是亚稳态,它要跃迁到基态必须发生正氦到仲氦的转变。
换而言之,1s2s电子组态是亚稳态,无论是仲氦还是正氦。它要跃迁到基态只能通过原子碰撞、双光子过程、电四极子跃迁、磁偶极子跃迁等微弱得多的过程。

\subsection{磁场中的多电子原子}

现在将\eqref{eq:magnetic-hamiltonian}引入。外加磁场破坏了空间各向同性,从而让磁量子数不再造成简并。
记由此产生的哈密顿量为
\begin{equation}
    \hat{H}_3 = - \hat{\vb*{\mu}} \cdot \vb*{B}.
\end{equation}

\subsubsection{弱磁场近似}

首先假定磁场很弱,比剩余相互作用和自旋-轨道耦合都弱,从而只计算一阶微扰。
计算结果是,
\begin{equation}
    E^{\text{mag}, (1)}_{JM_J} = \frac{J_z}{\hbar} g_J \mu_\text{B} B = - \vb*{\mu}_J \cdot \vb*{B},
\end{equation}
其中
\begin{equation}
    \vb*{\mu}_J = - g_J \frac{\mu_\text{B}}{\hbar} \vb*{J}
\end{equation}
称为\textbf{原子平均磁矩}。这个形式和单电子自旋角动量或轨道角动量对磁矩的贡献非常相似:要乘以一个无量纲修正因子$g_J$(称为\textbf{朗德g因子})。
无论如何,能量关于$M_J$的简并就解出了。

可以计算出对L-S耦合有
\begin{equation}
    g_J = \frac{3}{2} + \frac{S(S+1) - L(L+1)}{2J(J+1)},
    \label{eq:g-factor-ls}
\end{equation}
对j-j耦合$g_J$还和$\{j_i\}$有关。特别的,对L-S耦合下的双电子组态,我们有
\begin{equation}
    g_{j} = \frac{3}{2} + \frac{s(s+1) - l(l+1)}{2j(j+1)},
\end{equation}
这就是所谓的\textbf{单电子朗德g因子};而对j-j耦合下的双电子组态,有
\begin{equation}
    g_{J j_1 j_2} = \frac{g_{j_1} + g_{j_2}}{2} + \frac{(g_{j_2} - g_{j_1})(j_2(j_2+1) - j_1(j_1+1))}{2J(J+1)},
\end{equation}
其中$g_{j_1}$和$g_{j_2}$指的是单电子的磁矩和总角动量相差的因子。

\subsubsection{半经典图像}

上面的表达式的导出需要使用严格的量子力学计算,不过实际上它们还是可以使用角动量矢量的经典图像推导并解释。

对L-S耦合,由强到弱的三种微扰分别造成以下影响:

\begin{enumerate}
    \item 剩余相互作用$\hat{H}_1$使得不同电子的轨道磁矩绕总轨道角动量$\vb*{L}$快速进动,自旋磁矩绕总自旋角动量$\vb*{S}$快速进动;
    \item 自旋-轨道耦合$\hat{H}_2$让磁矩$\vb*{\mu}_L$,$\vb*{\mu}_S$和总磁矩$\vb*{\mu}$绕着$\vb*{J}$缓慢进动(由于自旋和轨道角动量和对应磁矩之间的比例关系差一个因子2,$\vb*{\mu}$和$\vb*{J}$并不平行);
    \item 磁场作用$\hat{H}_3$让$\vb*{J}$围绕$\vb*{B}$做拉莫尔进动,这个进动的幅度又远小于上述两种进动。
\end{enumerate}

这就意味着,在拉莫尔进动的时间尺度上,总磁矩平均而言只在$\vb*{J}$的方向上有分量,它垂直于$\vb*{J}$的分量一直在不停地变化,无法产生明显效应。%
\footnote{系统不能对过快的外界扰动产生反应,正如简谐振子展示的那样。}%
因此有效的$\vb*{\mu}$为
\[
    \bar{\vb*{\mu}} = \vb*{\mu}_J = \left(\vb*{\mu} \cdot \frac{\vb*{J}}{\abs{\vb*{J}}}\right) \frac{\vb*{J}}{\abs{\vb*{J}}} = - \underbrace{\frac{(\vb*{L} + 2\vb*{S}) \cdot (\vb*{L} + \vb*{S})}{J^2}}_{g_J} \frac{\mu_\text{B}}{\hbar} \vb*{J}.
\]
这就找到了$g_J$的表达式。通过
\[
    \vb*{J} = \vb*{L} + \vb*{S},
\]
就得到了\eqref{eq:g-factor-ls}。

j-j耦合由于有大量的$j$,处理起来稍微复杂一些。三种由强到弱的微扰分别造成以下影响:
\begin{enumerate}
    \item 自旋-轨道耦合$\hat{H}_2$让每个电子的角动量合成成$\vb*{j}$,单电子磁矩$\vb*{\mu}$绕着$\vb*{j}$快速进动;
    \item 剩余相互作用$\hat{H}_1$让每个电子的单电子磁矩绕着$\vb*{J}$缓慢进动;
    \item 磁场作用$\hat{H}_3$让$\vb*{J}$绕$\vb*{B}$做最慢的拉莫尔进动。
\end{enumerate}
$\vb*{\mu}_i$绕$\vb*{j}_i$的进动在拉莫尔进动的时间尺度下非常快,因此真正有效的只有$\vb*{\mu}_i$在$\vb*{j}_i$方向上的投影$\vb*{\mu}_{j i}$。
相应的,原子总磁矩$\vb*{\mu}$绕$\vb*{J}$进动也很快,有效的只有$\vb*{\mu}$在$\vb*{J}$方向上的投影,从而
\[
    \bar{\vb*{\mu}} = \vb*{\mu}_J = \sum_i \left( \vb*{\mu}_{j i} \cdot \frac{\vb*{J}}{\abs{\vb*{J}}} \right) \frac{\vb*{J}}{\abs{\vb*{J}}}.
\]
使用和L-S耦合非常相似的方法,我们有
\begin{equation}
    \vb*{\mu}_{j i} = - g_{j i} \frac{\mu_\text{B}}{\hbar} \vb*{j}_i, \quad g_{j i} = \frac{3}{2} + \frac{s_i(s_i + 1) - l_i (l_i + 1)}{2 j_i (j_1 + 1)},
\end{equation}
代入上式即可。

以上两个推导都建立在拉莫尔进动非常慢这一假设上。
\[
    \vb*{M}_J = \vb*{\mu}_J \times \vb*{B},
\]
再设$\vb*{\omega}_\text{L}$为$\vb*{J}$的进动角速度,即
\[
    \dv{\vb*{J}}{t} = \vb*{\omega}_\text{L} \times \vb*{J},
\]
从而推导出\textbf{拉莫尔进动角速度}
\begin{equation}
    \vb*{\omega}_\text{L} = \frac{g_J \mu_\text{B} \vb*{B}}{\hbar}.
\end{equation}
这就是总角动量绕着磁场进动的角速度。相应的,可以计算出 % TODO: $M_J$是什么?磁量子数还是别的什么东西?
\begin{equation}
    E^{\text{mag}, (1)}_{JM_J} = M_J g_J \mu_\text{B} B = M_J \hbar \omega_\text{L},
\end{equation}
和量子力学计算出的结果一致。在拉莫尔进动中总角动量不守恒,但是磁量子数和角量子数——总角动量的进动锥体的母线和高度——都还是好量子数。
可以看到,磁矩和磁场的夹角越大,能量越高。

\subsubsection{塞曼效应}

既然磁场在能量中引入了$M_J$的依赖,原本L-S耦合和j-j耦合的$2J+1$重简并会发生等间距的分裂。
$M_J$越大意味着角动量和磁场的夹角越小(磁场在$z$方向上)。
分裂产生的$2J+1$个子能级称为\textbf{塞曼能级},它会导致弱磁场中的原子光谱出现分裂,即\textbf{塞曼效应}。

塞曼效应可以分成两种,一种是\textbf{正常塞曼效应},谱线等间距分裂成三根,另一种是\textbf{反常塞曼效应},即不满足以上条件的光谱分裂。
塞曼效应发现时量子力学尚未建立,但是塞曼的老师——洛伦兹——通过经典电偶极振子模型计算出来正常塞曼效应。这也就是“正常”和“反常”这两个概念的来源。

正常塞曼效应的模型大体上是这样的:原子价电子受到正离子线性回复力,做简谐运动,

\end{document}