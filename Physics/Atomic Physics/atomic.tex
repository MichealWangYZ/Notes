\documentclass[UTF8, a4paper]{ctexart}

\usepackage{geometry}
\usepackage{titling}
\usepackage{titlesec}
\usepackage{paralist}
\usepackage{footnote}
\usepackage{enumerate}
\usepackage{amsmath, amssymb, amsthm}
\usepackage{cite}
\usepackage{graphicx}
\usepackage{subfigure}
\usepackage{physics}
\usepackage[colorlinks, linkcolor=black, anchorcolor=black, citecolor=black]{hyperref}

\geometry{left=3.28cm,right=3.28cm,top=2.54cm,bottom=2.54cm}
\titlespacing{\paragraph}{0pt}{1pt}{10pt}[20pt]
\setlength{\droptitle}{-5em}
\preauthor{\vspace{-10pt}\begin{center}}
\postauthor{\par\end{center}}

\newcommand*{\ee}{\mathrm{e}}
\newcommand*{\ii}{\mathrm{i}}
\renewcommand*{\dd}{\mathop{}\!\mathrm{d}}
\newcommand*{\st}{\quad \text{s.t.} \quad}
\newcommand*{\const}{\mathrm{const}}
\newcommand*{\natnums}{\mathbb{N}}
\newcommand*{\reals}{\mathbb{R}}
\newcommand*{\complexes}{\mathbb{C}}
\DeclareMathOperator{\timeorder}{T}
\newcommand*{\ogroup}[1]{\mathrm{O}(#1)}
\newcommand*{\sogroup}[1]{\mathrm{SO}(#1)}

\title{原子物理}
\author{wujinq}

\begin{document}

\maketitle

% TODO:黑体温度和辐射等

\section{粒子性描述的必要性}

\subsection{半经典问题}

本节总结一些原子物理中常见的不能够仅仅使用经典力学加上经典电磁场理论讨论的问题。本节给出的解答仍然是半经典的,也即,并不直接使用第一性的量子理论。

\subsubsection{黑体辐射}

\subsubsection{光电效应}

\subsubsection{康普顿散射}

非相干散射
\[
    \vb*{p} = \vb*{p}' + \vb*{p}'_e,
\]
\[
    pc + m_e c^2 = p' c + \sqrt{{p_e'}^2 c^2 + m_e^2 c^4},
\]
设散射角为$\theta$,计算得到
\begin{equation}
    \Delta \lambda = \frac{h}{m_e c} (1 - \cos \theta).
\end{equation}
波长移动仅仅和散射角有关,而和其它任何因素,如靶是什么原子等,完全无关。

散射光子的能量为
\begin{equation}
    E' = p'c = \frac{E}{1 + \dfrac{E}{m_e c^2}(1 - \cos \theta)} \geq \frac{E}{1 + \dfrac{2 E}{m_e c^2}}.
\end{equation}

我们也可以看到为什么经典理论下散射光的波长不应该变化:经典理论下电磁波完全是连续的,这等价于单个光子的能量充分小,因此光子的能量相比于与之碰撞的电子的束缚能很小,因此只能够发生相干散射,因此散射出的光子的波长并未发生变化。

原子序数增大,则原子周围的电子大部分都是被束缚的内层电子,因此相干散射随着原子序数增大而增强。

\subsubsection{原子结构}

如果原子是一个完全经典的体系,那么由于电子绕着原子核做周期性运动,它会向外发射电磁波,由此带来的电磁阻尼会导致电子失去能量而落入原子核。但实际上原子是非常稳定的,因此描述原子不能只使用经典力学。

\[
    v_n = \frac{\alpha c}{n},
\]
\[
    E_n = \frac{E_1}{n^2},
\]
\begin{equation}
    \alpha = \frac{e^2}{4\pi \epsilon_0 h c} \approx \frac{1}{137}
\end{equation}
称为\textbf{精细结构常数}。这是一个无量纲的常数,
\[
    R = \frac{1}{2} \frac{m_e (\alpha c)^2}{hc}.
\]
\[
    a_0 = \frac{4\pi\epsilon_0 \hbar^2}{m_e e^2}
\]
量子化条件
\begin{equation}
    m_e r_n v_n = n \hbar
\end{equation}
% TODO:波尔-索莫非量子化

通过经典的轨道运动方程和角动量量子化条件,我们就得到了完整的原子模型。

$n\to\infty$时原子几乎原理了原子核的束缚,能量趋于零,

\[
    m_e \longrightarrow m_\mu = \frac{m_e}{1 + m_e/m_A}
\]

两体修正:里德伯常数$R$和原子核的质量是有关系的,质量较大的类氢离子的$R$更加接近理论计算值

能谱展宽的原因:
\begin{itemize}
    \item \textbf{自然展宽},即原子发出的波列的长度有限(该长度与原子的激发态的寿命正相关),而光谱是波列做傅里叶变换得到的结果,因此谱线展宽,这种展宽完全来自不确定性原理;
    \item \textbf{多普勒展宽},即波列传播方向不同(光源中的粒子会无规则运动,被测原子也会)导致它们被测量到的频率不一,而导致谱线展宽;
    \item \textbf{洛伦兹展宽},即被测原子和其它粒子发生碰撞,能量交换而改变了释放出来的波列的频率,从而谱线展宽。
\end{itemize}

\section{氢原子}

\end{document}