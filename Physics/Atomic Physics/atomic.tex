\documentclass[UTF8, a4paper]{ctexart}

\usepackage{geometry}
\usepackage{titling}
\usepackage{titlesec}
\usepackage{paralist}
\usepackage{footnote}
\usepackage{enumerate}
\usepackage{amsmath, amssymb, amsthm}
\usepackage{cite}
\usepackage{graphicx}
\usepackage{subfigure}
\usepackage{physics}
\usepackage[colorlinks, linkcolor=black, anchorcolor=black, citecolor=black]{hyperref}

\geometry{left=3.28cm,right=3.28cm,top=2.54cm,bottom=2.54cm}
\titlespacing{\paragraph}{0pt}{1pt}{10pt}[20pt]
\setlength{\droptitle}{-5em}
\preauthor{\vspace{-10pt}\begin{center}}
\postauthor{\par\end{center}}

\newcommand*{\ee}{\mathrm{e}}
\newcommand*{\ii}{\mathrm{i}}
\renewcommand*{\dd}{\mathop{}\!\mathrm{d}}
\newcommand*{\st}{\quad \text{s.t.} \quad}
\newcommand*{\const}{\mathrm{const}}
\newcommand*{\natnums}{\mathbb{N}}
\newcommand*{\reals}{\mathbb{R}}
\newcommand*{\complexes}{\mathbb{C}}
\DeclareMathOperator{\timeorder}{T}
\newcommand*{\ogroup}[1]{\mathrm{O}(#1)}
\newcommand*{\sogroup}[1]{\mathrm{SO}(#1)}

\title{原子物理}
\author{wujinq}

\begin{document}

\maketitle

% TODO:黑体温度和辐射等

\section{量子理论}

\subsection{半经典问题}

本节总结一些原子物理中常见的不能够仅仅使用经典力学加上经典电磁场理论讨论的问题。本节给出的解答仍然是半经典的,也即,并不直接使用第一性的量子理论。

\subsubsection{黑体辐射}

\subsubsection{光电效应}

\subsubsection{康普顿散射}

非相干散射
\[
    \vb*{p} = \vb*{p}' + \vb*{p}'_e,
\]
\[
    pc + m_e c^2 = p' c + \sqrt{{p_e'}^2 c^2 + m_e^2 c^4},
\]
设散射角为$\theta$,计算得到
\begin{equation}
    \Delta \lambda = \frac{h}{m_e c} (1 - \cos \theta).
\end{equation}
波长移动仅仅和散射角有关,而和其它任何因素,如靶是什么原子等,完全无关。

散射光子的能量为
\begin{equation}
    E' = p'c = \frac{E}{1 + \dfrac{E}{m_e c^2}(1 - \cos \theta)} \geq \frac{E}{1 + \dfrac{2 E}{m_e c^2}}.
\end{equation}

我们也可以看到为什么经典理论下散射光的波长不应该变化:经典理论下电磁波完全是连续的,这等价于单个光子的能量充分小,因此光子的能量相比于与之碰撞的电子的束缚能很小,因此只能够发生相干散射,因此散射出的光子的波长并未发生变化。

原子序数增大,则原子周围的电子大部分都是被束缚的内层电子,因此相干散射随着原子序数增大而增强。

\subsubsection{原子结构}

如果原子是一个完全经典的体系,那么由于电子绕着原子核做周期性运动,它会向外发射电磁波,由此带来的电磁阻尼会导致电子失去能量而落入原子核。但实际上原子是非常稳定的,因此描述原子不能只使用经典力学。

\[
    v_n = \frac{\alpha c}{n},
\]
\[
    E_n = \frac{E_1}{n^2},
\]
\begin{equation}
    \alpha = \frac{e^2}{4\pi \epsilon_0 h c} \approx \frac{1}{137}
\end{equation}
称为\textbf{精细结构常数}。这是一个无量纲的常数,
\[
    R = \frac{1}{2} \frac{m_e (\alpha c)^2}{hc}.
\]
\[
    a_0 = \frac{4\pi\epsilon_0 \hbar^2}{m_e e^2}
\]
量子化条件
\begin{equation}
    m_e r_n v_n = n \hbar
\end{equation}
% TODO:波尔-索莫非量子化

通过经典的轨道运动方程和角动量量子化条件,我们就得到了完整的原子模型。

$n\to\infty$时原子几乎原理了原子核的束缚,能量趋于零,

\[
    m_e \longrightarrow m_\mu = \frac{m_e}{1 + m_e/m_A}
\]

两体修正:里德伯常数$R$和原子核的质量是有关系的,质量较大的类氢离子的$R$更加接近理论计算值

能谱展宽的原因:
\begin{itemize}
    \item \textbf{自然展宽},即原子发出的波列的长度有限(该长度与原子的激发态的寿命正相关),而光谱是波列做傅里叶变换得到的结果,因此谱线展宽,这种展宽完全来自不确定性原理;
    \item \textbf{多普勒展宽},即波列传播方向不同(光源中的粒子会无规则运动,被测原子也会)导致它们被测量到的频率不一,而导致谱线展宽;
    \item \textbf{洛伦兹展宽},即被测原子和其它粒子发生碰撞,能量交换而改变了释放出来的波列的频率,从而谱线展宽。
\end{itemize}

\subsection{薛定谔方程}

由于传统上认为是波的对象实际上具有粒子性,传统上认为是粒子的对象实际上具有波动性,需要寻找一个波动方程,被它描述的波同时也能够被赋予粒子性的意义。
这种波称为\textbf{物质波}或\textbf{德布罗意波}。

\begin{equation}
    \ii \hbar \pdv{\psi}{t} = - \frac{\laplacian}{2m} \psi + V \psi.
\end{equation}

\section{被束缚的电子系统}

\subsection{势阱}

有限高,无限厚的势阱:粒子可以有隧穿,但是在无穷远处衰减为零。

无限高,无限厚的势阱:没有任何隧穿。

\subsection{库伦势场中的单电子}

库伦势场中的单个电子的哈密顿量为
\begin{equation}
    \hat{H} \psi = \frac{\hat{p}^2}{2m} \psi - \frac{Z}{4\pi \epsilon_0} \frac{e^2}{\abs{\vb*{r}}} \psi.
    \label{eq:columb-electron-hamiltonian}
\end{equation}
求解此问题等价于求解定态方程
\begin{equation}
    \frac{\hat{p}^2}{2m} \psi - \frac{1}{4\pi \epsilon_0} \frac{e^2}{\abs{\vb*{r}}} \psi = E \psi,
\end{equation}
可以直接分离变量,但让我们首先采取一种物理意义比较明显的变形。
\[
    \hat{H} = - \frac{\hbar^2}{2 m r} \pdv[2]{r} r + \frac{\hat{L}^2}{2 m r^2} - \frac{1}{4\pi \epsilon_0} \frac{e^2}{\abs{\vb*{r}}}.
\]
算符$\hat{L}^2$仅仅和$\theta$和$\varphi$有关,因此我们可以将径向部分和角向部分做分离变量。
能够这么容易地分离变量当然是因为库伦势场中的单电子非常对称,因此可以找到很多守恒量。

标记电子的量子数:
\begin{enumerate}
    \item 主量子数$n=1, 2, \ldots$,它标记不同的能量,它是分立的,因为电子陷在一个势阱中,从而是离散谱;
    \item 角量子数$l = 0, 1, \ldots, n$,它是分立的,因为波函数在角方向上是单值的;
    \item 磁量子数$m_l = 0, \pm 1, \ldots, \pm l$,它
\end{enumerate}

可以依稀从量子力学中的氢原子看出一些经典的图像。在经典的原子模型中,粒子可以做椭圆运动,但是运动的能量仅仅关乎一个参数即椭圆的半长轴$a$,而和半短轴$b$无关;半短轴$b$则决定角动量等。

求解出来的解是$\hat{H}, \hat{L}^2, \hat{L}_z$的共同本征态,存在这样的共同本征态是因为这三个量对易。

能级简并度
\[
    g_n = n^2
\]

\begin{equation}
    L^2 = l(l+1) \hbar^2
\end{equation}

磁量子数不改变径向概率密度。

径向运动是有效势阱中的一维运动
\[
    \frac{p_r^2}{2m} + \frac{L^2}{2mr^2} - \frac{Ze^2}{4\pi \epsilon_0 r} = E
\]

纯量子的理论也展现出了和经典理论很不同的一些特性。请注意量子理论中电子可以完全没有角动量,这在经典理论下是不可能的——电子会直接落入原子核。
然而,哈密顿量\eqref{eq:columb-electron-hamiltonian}有量子涨落,因此如果角动量确定为零,那么电子的位置就不能够确定,因此电子并不会落入原子核。

\subsection{跃迁和发光}

跃迁的两种机制:受激跃迁、自发跃迁

受激跃迁只需要量子化原子+经典电动力学即可解决,自发跃迁必须使用QED

爱因斯坦有唯象理论

考虑温度为$T$的空腔中有大量相同的原子,显然处于定态$i$和$j$的原子需要满足玻尔兹曼分布率
\[
    N_i \propto \ee^{-\frac{E_i}{kT}},
\]
\[
    \frac{N_j}{N_i} = \ee^{-\hbar \omega_{ji} / kT},
\]
平衡态时从$E_j > E_i$向上跃迁跃迁率为
\[
    \lambda_{ij} = B_{ij} u(\omega_{ji}, T),
\]
\[
    \lambda_{ji} = B_{ji} u(\omega_{ji}, T) + A_{ji},
\]
$B$为受激发射系数,$A$为自发发射系数。平衡时两者相等,即有
\[
    N_i B_{ij} u(\omega_{ji}, T) = N_j (B_{ji} u(\omega_{ji}, T) + A_{ji})
\]
$T \to \infty$,不同能级上原子分布的个数差别变得很小,$u \to \infty$,$B_{ij} = B_{ji}$,
\[
    u = \frac{\hbar \omega^3}{\pi^2 c^3} \frac{1}{\ee^{\hbar \omega / kT} - 1}
\]
自发发射跃迁率为
\[
    A_{ji} = \frac{\omega_{ji}^3}{3\pi \epsilon_0 \hbar c^3}
\]

\end{document}