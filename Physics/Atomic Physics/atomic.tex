\documentclass[UTF8, a4paper]{ctexart}

\usepackage{geometry}
\usepackage{titling}
\usepackage{titlesec}
\usepackage{paralist}
\usepackage{footnote}
\usepackage{enumerate}
\usepackage{amsmath, amssymb, amsthm}
\usepackage{cite}
\usepackage{graphicx}
\usepackage{subfigure}
\usepackage{physics}
\usepackage[colorlinks, linkcolor=black, anchorcolor=black, citecolor=black]{hyperref}

\geometry{left=3.28cm,right=3.28cm,top=2.54cm,bottom=2.54cm}
\titlespacing{\paragraph}{0pt}{1pt}{10pt}[20pt]
\setlength{\droptitle}{-5em}
\preauthor{\vspace{-10pt}\begin{center}}
\postauthor{\par\end{center}}

\newcommand*{\ee}{\mathrm{e}}
\newcommand*{\ii}{\mathrm{i}}
\newcommand*{\st}{\quad \text{s.t.} \quad}
\newcommand*{\const}{\mathrm{const}}
\newcommand*{\natnums}{\mathbb{N}}
\newcommand*{\reals}{\mathbb{R}}
\newcommand*{\complexes}{\mathbb{C}}
\DeclareMathOperator{\timeorder}{T}
\newcommand*{\ogroup}[1]{\mathrm{O}(#1)}
\newcommand*{\sogroup}[1]{\mathrm{SO}(#1)}
\DeclareMathOperator{\legpoly}{P}

\title{原子物理}
\author{吴何友}

\begin{document}

\maketitle

% TODO:黑体温度和辐射等

\section{量子理论的必要性}

\subsection{半经典问题}

本节总结一些原子物理中常见的不能够仅仅使用经典力学加上经典电磁场理论讨论的问题。本节给出的解答仍然是半经典的,也即,并不直接使用第一性的量子理论。

\subsubsection{黑体辐射}

% 韦恩定律,金斯公式,等等

\subsubsection{光电效应}

\subsubsection{康普顿散射}

非相干散射
\[
    \vb*{p} = \vb*{p}' + \vb*{p}'_e,
\]
\[
    pc + m_e c^2 = p' c + \sqrt{{p_e'}^2 c^2 + m_e^2 c^4},
\]
设散射角为$\varphi$,计算得到
\begin{equation}
    \Delta \lambda = \frac{h}{m_e c} (1 - \cos \varphi).
\end{equation}
波长移动仅仅和散射角有关,而和其它任何因素,如靶是什么原子等,完全无关。

散射光子的能量为
\begin{equation}
    E' = p'c = \frac{E}{1 + \dfrac{E}{m_e c^2}(1 - \cos \varphi)} \geq \frac{E}{1 + \dfrac{2 E}{m_e c^2}}.
\end{equation}

我们也可以看到为什么经典理论下散射光的波长不应该变化:经典理论下电磁波完全是连续的,这等价于单个光子的能量充分小,因此光子的能量相比于与之碰撞的电子的束缚能很小,因此只能够发生相干散射,因此散射出的光子的波长并未发生变化。

原子序数增大,则原子周围的电子大部分都是被束缚的内层电子,因此相干散射随着原子序数增大而增强。

\subsubsection{原子结构}

如果原子是一个完全经典的体系,那么由于电子绕着原子核做周期性运动,它会向外发射电磁波,由此带来的电磁阻尼会导致电子失去能量而落入原子核。但实际上原子是非常稳定的,因此描述原子不能只使用经典力学。

\[
    v_n = \frac{\alpha c}{n},
\]
\[
    E_n = \frac{E_1}{n^2},
\]
\begin{equation}
    \alpha = \frac{e^2}{4\pi \epsilon_0 h c} \approx \frac{1}{137}
\end{equation}
称为\textbf{精细结构常数}。这是一个无量纲的常数,
\[
    R = \frac{1}{2} \frac{m_e (\alpha c)^2}{hc}.
\]
\[
    a_0 = \frac{4\pi\epsilon_0 \hbar^2}{m_e e^2}
\]
量子化条件
\begin{equation}
    m_e r_n v_n = n \hbar
\end{equation}
% TODO:波尔-索莫非量子化

通过经典的轨道运动方程和角动量量子化条件,我们就得到了完整的原子模型。

$n\to\infty$时原子几乎原理了原子核的束缚,能量趋于零,

\[
    m_e \longrightarrow m_\mu = \frac{m_e}{1 + m_e/m_A}
\]

两体修正:里德伯常数$R$和原子核的质量是有关系的,质量较大的类氢离子的$R$更加接近理论计算值

能谱展宽的原因:
\begin{itemize}
    \item \textbf{自然展宽},即原子发出的波列的长度有限(该长度与原子的激发态的寿命正相关),而光谱是波列做傅里叶变换得到的结果,因此谱线展宽,这种展宽完全来自不确定性原理;
    \item \textbf{多普勒展宽},即波列传播方向不同(光源中的粒子会无规则运动,被测原子也会)导致它们被测量到的频率不一,而导致谱线展宽;
    \item \textbf{洛伦兹展宽},即被测原子和其它粒子发生碰撞,能量交换而改变了释放出来的波列的频率,从而谱线展宽。
\end{itemize}

\subsection{薛定谔方程}

由于传统上认为是波的对象实际上具有粒子性,传统上认为是粒子的对象实际上具有波动性,需要寻找一个波动方程,被它描述的波同时也能够被赋予粒子性的意义。
这种波称为\textbf{物质波}或\textbf{德布罗意波}。

量子力学给出以下薛定谔方程:
\begin{equation}
    \ii \hbar \pdv{\psi}{t} = - \frac{\laplacian}{2m} \psi + V \psi.
\end{equation}
如果考虑磁场,那么就有
\begin{equation}
    \ii \hbar \pdv{\psi}{t} = - \frac{(\nabla - \ii e \vb*{A}/c)^2}{2m} \psi + V \psi.
\end{equation}
这当然只是“磁场让动理动量和正则动量不同”的量子版本。

自旋的表达式:
\begin{equation}
    \vb*{S} = \frac{1}{2} sum_{\alpha, \beta} \vb*{\sigma}_{\alpha \beta} \ket{\alpha} \bra{\beta}.
\end{equation}

\section{束缚态单电子系统}

\subsection{势阱}

有限高,无限厚的势阱:粒子可以有隧穿,但是在无穷远处衰减为零。

无限高,无限厚的势阱:没有任何隧穿。

\subsection{有心力场中的单电子}

\subsubsection{哈密顿量和薛定谔方程}

有心力场中的单个电子的哈密顿量为
\begin{equation}
    \hat{H} \psi = \frac{\hat{p}^2}{2m} \psi + V(r) \psi.
\end{equation}
求解此问题等价于求解坐标表象下的定态方程
\begin{equation}
    \frac{\hat{p}^2}{2m} \psi + V(r) \psi = - \frac{\hbar^2}{2m} \laplacian \psi + V(r) \psi = E \psi,
    \label{eq:centered-force-eq}
\end{equation}
使用球坐标系,以$\theta$为和$z$轴的夹角,$\varphi$为$x$-$y$平面上的转角,则
\[
    \hat{p}^2 = -\hbar^2 \laplacian = - \hbar^2 \left( \frac{1}{r^2} \pdv{r} r^2 \pdv{r} + \frac{1}{r^2 \sin \theta} \pdv{\theta} \sin \theta \pdv{\theta} + \frac{1}{r^2 \sin^2 \theta} \pdv[2]{\varphi} \right) .
\]
可以直接分离变量,但让我们首先采取一种物理意义比较明显的变形。注意到轨道角动量算符的平方为
\[
    \hat{L}^2 = \hat{\vb*{r}}^2 \hat{\vb*{p}}^2 - (\hat{\vb*{r}} \cdot \hat{\vb*{p}}) (\hat{\vb*{p}} \cdot \hat{\vb*{r}}),
\]
在球坐标系下$\vb*{r}$只在$r$方向上有分量,于是上式变成
\[
    \hat{L}^2 = r^2 \hat{p}^2 - r \left( - \hbar^2 \pdv[2]{r} \right) r,
\]
注意到(实际上这是动量-坐标对易关系的自然推论)
\[
    r \pdv[2]{r} r = \pdv{r} r^2 \pdv{r},
\]
这就给出了角动量长度平方的一个简洁的表达式:
\begin{equation}
    \hat{L}^2 = - \hbar^2 \left( \frac{1}{\sin \theta} \pdv{\theta} \sin \theta \pdv{\theta} + \frac{1}{\sin^2 \theta} \pdv[2]{\varphi} \right),
\end{equation}
当然也可以按照定义直接计算出这个表达式。相应的,哈密顿量在球坐标系下就是
\[
    \hat{H} = - \frac{\hbar^2}{2 m r} \pdv[2]{r} r + \frac{\hat{L}^2}{2 m r^2} + V(r).
\]
以上对哈密顿量的改写和经典情况可以类比。经典情况下,径向运动是有效势阱中的一维运动,且满足
\[
    \frac{p_r^2}{2m} + \frac{L^2}{2mr^2} - \frac{Ze^2}{4\pi \epsilon_0 r} = E,
\]
正好是将所有算符替换成实数之后的结果——本该如此。

现在,定态薛定谔方程成为
\[
    - \frac{\hbar^2}{2 m r} \pdv[2]{r} (r \psi) + \frac{\hat{L}^2}{2 m r^2} \psi + V(r) \psi = E \psi,
\]
算符$\hat{L}^2$仅仅和$\varphi$和$\theta$有关,因此我们可以将径向部分和角向部分做分离变量。
设
\[
    \psi = R(r) Y(\theta, \varphi),
\]
则径向部分的方程是
\begin{equation}
    \left( - \frac{\hbar^2}{2m r^2} \dv{r} r^2 \dv{r} + \frac{\hbar^2}{2m r^2} \alpha + V(r) \right) R = E R,
    \label{eq:original-r-equation}
\end{equation}
角向部分的方程为
\begin{equation}
    \hat{L}^2 Y = \alpha \hbar^2 Y.
    \label{eq:angle-equation}
\end{equation}
其中$\alpha$为常数。

\subsubsection{角向部分的方程的求解}

径向部分的方程包含一个可变的$V(r)$项,而角向部分的方程可以直接解出。
请注意角向部分实际上是一个拉普拉斯方程的角向部分,因此其解为球谐函数。
下面简单地展示其求解过程。设
\[
    Y(\theta, \varphi) = \Theta(\theta) \Phi(\varphi),
\]
代入
\[
    - \left( \frac{1}{\sin \theta} \pdv{\theta} \sin \theta \pdv{\theta} + \frac{1}{\sin^2 \theta} \pdv[2]{\varphi} \right) Y = \alpha Y,
\]
得到两个方程
\[
    \frac{1}{\Phi} \dv[2]{\Phi}{\varphi} = c,
\]
以及
\[
    - \left( \frac{1}{\sin \theta} \dv{\theta} \sin \theta \dv{\theta} + \frac{1}{\sin^2 \theta} c \right) \Theta = \alpha \Theta.
\]
$\Phi$必须是单值的,因为波函数应该是单值的,于是
\[
    c = m^2, \quad m = 0, 1, 2, \ldots,
\]
而
\[
    - \left( \dv{\cos \theta} (1 - \cos^2 \theta) \dv{\cos \theta} + \frac{1}{1 - \cos^2 \theta} m^2 \right) \Theta = \alpha \Theta.
\]
这是一个连带勒让德方程,为了保证$\Theta(\theta)$单值且有界,应有
\[
    \alpha = l(l+1), \quad l \geq \abs{m}, \quad l = 0, 1, 2, \ldots,
\]
于是\eqref{eq:angle-equation}的一组正交解为
\[
    Y(\theta, \varphi) = \ee^{\ii m \varphi} \legpoly_l^m(\cos \theta),
\]
其中$\legpoly_l^m$表示缔合勒让德多项式。归一化使用球坐标系的角向积分
\[
    \int \sin \theta \dd{\theta} \dd{\varphi},
\]
最后得到正交归一化的球谐函数
% TODO:(-1)^m因子?
\begin{equation}
    Y_{lm}(\theta, \varphi) = (-1)^m \sqrt{\frac{(2l+1)(l-\abs{m})!}{4\pi (l+\abs{m})!}} \legpoly_l^\abs{m} (\cos \theta) \ee^{\ii m \varphi}, \quad l = 0, 1, \ldots, m = \pm l, \pm (l-1), \ldots, 0 .
\end{equation}

球谐函数是$\hat{L}^2$的本征函数,但它带有两个量子数$l$和$m$,这意味着$\hat{L}^2$的本征函数存在简并;$l,m$分别标记了$\hat{L}^2$的本征值和某个不确定的可观察量的本征值。
注意到球坐标系中
\[
    \hat{L}_z = - \ii \hbar \pdv{\varphi},
\]
它正是$\Phi(\varphi)$满足的本征方程中的那个算符,因此$m$标记的是$\hat{L}_z$的本征值,$l$标记的是$\hat{L}^2$的本征值。
这是正确的,因为
\[
    m = \pm l, \pm (l-1), \ldots, 0
\]
正是角动量代数的重要性质。我们下面记此处的$m$为$m_l$,与自旋角动量在$z$方向上的取值$m_s = \pm \frac{1}{2}$相区分。
球谐函数$Y_{lm}(\theta, \varphi)$同时是$\hat{L}^2$和$L_z$的本征函数,相应的本征值为
\begin{equation}
    \hat{L}^2 Y_{lm} = l(l+1) \hbar^2 Y_{lm}, \quad \hat{L}_z Y_{lm} = m \hbar Y_{lm}.
\end{equation}

\subsubsection{量子数}

现在径向方程\eqref{eq:original-r-equation}成为了
\begin{equation}
    \left( - \frac{\hbar^2}{2m r^2} \dv{r} r^2 \dv{r} + \frac{\hbar^2}{2m r^2} l(l+1) + V(r) \right) R = E R.
    \label{eq:r-equation}
\end{equation}
这是一个单变量的本征值问题,它还会产生一个(而且也只有一个,因为一旦$E$确定了,$R$就确定了,不存在简并)量子数$n$,它标记不同的能量。
请注意由于角动量部分被分离变量出去了,实际上多出来了一个有效势$l(l+1)$项。

这样,\eqref{eq:centered-force-eq}有一组由$n, l, m_l$标记的正交归一化解,再考虑到自旋自由度,电子的状态就完全确定了。
通过求解过程可以看出前三个量子数标记了$\hat{H}, \hat{L}^2, \hat{L}_z$的本征值;
它们是通过分离变量求解坐标空间中的薛定谔方程得到的,因此随着时间变化,它们对应的物理量是守恒的且彼此对易,能够找到这样三个守恒且彼此对易的物理量当然是因为有心力系统的对称性。
目前尚未引入任何涉及自旋的机制,因此自旋也是恒定不变的。
于是我们找到了四个标记电子的好量子数:
\begin{enumerate}
    \item 主量子数$n$,它标记不同的能量,它是分立的,因为电子陷在一个势阱中,从而是离散谱;
    \item 角量子数$l$,它标记不同的角动量大小,它是分立的,因为波函数在角方向上是单值的;
    \item 磁量子数$m_l = 0, \pm 1, \ldots, \pm l$,它标记$z$轴上的角动量分量,同样也是分立的;
    \item 自旋量子数$m_s = \pm \frac{1}{2}$,它来自电子的内禀旋转自由度;自旋量子数和我们刚才讨论的轨道空间无关。
\end{enumerate}

这四个量子数直接决定了波函数的形状。主量子数决定了径向概率分布,角量子数和磁量子数决定了波函数的角向概率分布。
这四个量子数还可以确定其它一些量子数。例如,做宇称变换
\[
    (r, \theta, \varphi) \longrightarrow (r, \pi - \theta, \pi + \varphi),
\]
径向部分始终不变,若$l$为奇数则球谐函数会差一个负号,从而波函数为奇宇称,若$l$为偶数则球谐函数不变,从而波函数为偶宇称。

纯量子的理论展现出了和经典理论很不同的一些特性。请注意量子理论中电子可以完全没有角动量,这在经典理论下是不可能的——电子会直接落入有心力的力心,比如说原子核。
然而,哈密顿量\eqref{eq:columb-electron-hamiltonian}中各项不对易从而有量子涨落,因此如果角动量确定为零,那么电子的位置就不能够确定,因此电子并不会落入力心。

\subsubsection{库伦势场}

库伦势场中的单个电子的哈密顿量为
\begin{equation}
    \hat{H} \psi = \frac{\hat{p}^2}{2m} \psi - \frac{Z}{4\pi \epsilon_0} \frac{e^2}{\abs{\vb*{r}}} \psi.
    \label{eq:columb-electron-hamiltonian}
\end{equation}
我们常常将这样的体系称为类氢原子,因为它和氢原子的结构除了$Z$可能不一样以外完全一致。
此时\eqref{eq:r-equation}为
\[
    \left( - \frac{\hbar^2}{2m r^2} \dv{r} r^2 \dv{r} + \frac{\hbar^2}{2m r^2} l(l+1) - \frac{1}{4\pi \epsilon_0} \frac{e^2}{r} \right) R = E R,
\]
显然这是一个束缚在势阱中的电子的方程,它必定有束缚态解,从而可以提供我们需要的主量子数。

% TODO:拉盖尔方程

\[
    l = 0, 1, 2, \ldots, n-1.
\]

可以依稀从量子力学中的氢原子看出一些经典的图像。在经典的原子模型中,粒子可以做椭圆运动,但是运动的能量仅仅关乎一个参数即椭圆的半长轴$a$,而和半短轴$b$无关;半短轴$b$则决定角动量等。
因此能量和角动量是分开的。角动量可以有不同的指向,因此角动量长度和它在$z$轴上的投影也没有必然的关系(当然,角动量在$z$轴上的投影不可能超过总的角动量长度)

求解出氢原子的薛定谔方程的完整解之后,可以发现主量子数为$n$的能级上有$n+1$个可能的角量子数,每个角量子数又允许$2l+1$个磁量子数,而最后还有两个自旋量子数,因此主量子数为$n$的能级上有
\[
    \frac{1}{2} (1 + (n-1)) n = \frac{1}{2}n^2
\]
个轨道,有$n^2$个电子。

$n-l$决定了径向峰值的数目

\subsection{跃迁和发光}

电子和电磁场耦合,因此可以在不同能级之间跃迁而发射或吸收光子。
跃迁包括受激跃迁(电子首先吸收光子,然后发生跃迁)以及自发跃迁(电子直接发生跃迁)。
对这一过程的完整计算涉及量子电动力学的束缚态,但通常对能标不是非常高的过程,使用量子化的原子和经典电动力学就足够计算一些问题。
受激跃迁只需要量子化原子加上经典电动力学即可完全解释,而自发跃迁不能使用经典电动力学解释而必须将光场量子化,因为“自发”意味着光场的真空涨落,这要使用量子理论处理。

\subsubsection{跃迁的费米黄金法则}

在讨论跃迁之前需要先引入一个计算跃迁概率的规则:费米黄金法则。
设系统的自由哈密顿量为$\hat{H}_0$,它的一组基矢量为$\{\ket{m}\}$,本征值记为$\{E_m\}$,相互作用哈密顿量为$\hat{H}'$。
假定相互作用哈密顿量不显含时间。设系统初态为$\ket{m}$,态随时间的演化为
\[
    \ket{\psi(t)} = \sum_n a_n(t) \ket{n} \ee^{- \ii E_n t / \hbar},
\]
显然$t=0$时除了$a_m=1$以外其它$a$均为零。使用Dyson级数并只计算到一阶,有
\[
    \ii \hbar a^{(1)}_k(t) = \sum_n \int \dd{t'} \mel{k}{\hat{H}'}{n} \ee^{\ii \omega_{kn} t'} a_n^{(0)},
\]
其中
\[
    \omega_{mn} = \frac{E_m - E_n}{\hbar}.
\]
$a_n^{(0)}$只在$n=m$时有非零值,且时间演化是从$t'=0$演化到$t'=t$,于是
\[
    \begin{aligned}
        \ii \hbar a^{(1)}_k(t) &= \sum_n \int \dd{t'} \mel{k}{\hat{H}'}{n} \ee^{\ii \omega_{kn} t'} a_n^{(0)} \\
        &= \mel{k}{\hat{H}'}{m} \frac{\sin \omega_{km} t / 2}{\omega_{km} / 2} \ee^{\ii \omega_{km} t / 2}. 
    \end{aligned}
\]
注意到$m \neq k$时$a_k = a_k^{(1)}$,而$a_k(t)$的模长平方正是$t'=0$时系统状态为$\ket{m}$而$t'=t$时系统经过观测状态为$\ket{k}$的概率,此概率就是所谓的\textbf{跃迁概率},于是跃迁概率的表达式就是
\begin{equation}
    P_k(t) = \frac{4 \abs*{\mel{k}{\hat{H}'}{m}}^2}{\hbar^2} \frac{\sin^2 \omega_{km} t / 2}{\omega_{km}^2}.
\end{equation}
现在假如系统实际上是一个开放系统,且诸$\ket{m}$构成一组偏好基,则可以使用一个经典马尔可夫过程来描述系统的演化,而使用经典的态(各个态出现的概率就是$\ket{m}$的振幅的模长平方)描述系统的状态。
系统每个时刻都有一定概率跃迁,也有一定概率不跃迁而等待下一时刻,此时跃迁速率为
\begin{equation}
    \Gamma_k(t) = \dv{P_k}{t} = \frac{2 \abs*{\mel{k}{\hat{H}'}{m}^2}}{\hbar^2} \frac{\sin \omega_{km} t}{\omega_{km}}.
\end{equation}
如果我们并不关心系统跃迁到了哪一个态而只关心系统事实上发生了跃迁,那么总跃迁速率为
\[
    \Gamma(t) = \sum_k \Gamma_k(t) = \sum_{E_k} \frac{2 \abs*{\mel{k}{\hat{H}'}{m}^2}}{\hbar^2} \frac{\sin \omega_{km} t}{\omega_{km}},
\]
如果跃迁之后的能级是连续谱,那么就有
\[
    \begin{aligned}
        \Gamma(t) &= \int \dd{\omega} \hbar \rho(E) \frac{2 \abs*{\mel{k}{\hat{H}'}{m}^2}}{\hbar^2} \frac{\sin \omega t}{\omega} \\
        &= \frac{2 \abs*{\mel{k}{\hat{H}'}{m}^2}}{\hbar} \int \dd{\omega} \rho(E) \frac{\sin \omega t}{\omega}, 
    \end{aligned}
\]
其中$E = \hbar \omega + \const$,而实际上$\sin \omega t / \omega$是一个非常尖锐的峰,因此可以把态密度$\rho(E)$提到积分号外面,最后计算得到
\begin{equation}
    \Gamma(t) = \frac{2\pi}{\hbar} \rho(E) \abs*{\mel{k}{\hat{H}'}{m}^2}.
\end{equation}
这样,如果我们有数目巨大的一系列完全相同的系统,总数为$N$,它们之间相互影响很小,那么在$\dd{t}$时间内,发生跃迁的系统的数目几乎确定为$N \Gamma(t) \dd{t}$。
因此可以列出某个状态的系统的数目服从的微分方程。

\subsubsection{爱因斯坦的唯象理论}

考虑温度为$T$的空腔中有大量相同的原子,显然处于定态$i$和$j$的原子需要满足玻尔兹曼分布率
\[
    N_i \propto \ee^{-\frac{E_i}{k_\text{B} T}},
\]
或者写成
\[
    \frac{N_j}{N_i} = \ee^{-\hbar \omega_{ji} / kT}.
\]
能够达到热力学平衡意味着电子需要在不同能级之间跃迁。
电子和电磁场有耦合,因此电子在不同能级上跃迁确实是可以的。跃迁发生的机制可能有这么几种:
\begin{enumerate}
    \item 自发发射,也就是电子放出一个光子,跃迁到较低的能级;
    \item 受激发射,即电子先吸收一个光子再放出一个光子,然后发生跃迁;
    \item 吸收,即电子吸收一个光子然后跃迁到较高的轨道上。
\end{enumerate}
请注意受激发射和吸收可以使用量子化的原子和一个经典电磁场耦合来解释,但是自发发射在这个框架下是很难解释的,因为一个激发态的原子放在完全没有电磁场的空间内照样会有自发发射。
对这一现象的完整解释显然涉及真空涨落,因此需要量子电动力学。

设温度为$T$的光场中频率为$\omega$附近的能量密度为$u(\omega, T)$。设有两个能级$i$和$j$,且$E_j > E_i$。
我们假定(之后会通过量子力学严格证明)自发发射的跃迁率和$u$无关,而受激发射和吸收的跃迁率正比于$u(\omega_{ji}, T)$,其中
\begin{equation}
    \hbar \omega_{ji} = E_j - E_i,
    \label{eq:photon-energy}
\end{equation}
这样在这两个能级之间的自发发射、受激发射、吸收的跃迁率分别是
\[
    A_{ji} N_j, \quad B_{ji} N_j u(\omega_{ji}, T), \quad C_{ij} N_i u(\omega_{ji}, T).
\]
能级$i$向上跃迁到$j$的跃迁率为
\[
    \lambda_{ij} = C_{ij} u(\omega_{ji}, T),
\]
能级$j$向下跃迁到$i$的跃迁率为
\[
    \lambda_{ji} = B_{ji} u(\omega_{ji}, T) + A_{ji}.
\]
平衡时两者相等,即有
\[
    N_i C_{ij} u(\omega_{ji}, T) = N_j (B_{ji} u(\omega_{ji}, T) + A_{ji})
\]
$T \to \infty$,不同能级上原子分布的个数差别变得很小,$u \to \infty$,而上式仍然成立,因此$C_{ij} = B_{ji}$%
\footnote{请注意对温度的依赖被完全归入$u(\omega, T)$中,系数$C$和$B$由电子和光场耦合的方式决定,因此不依赖温度。}%
,这样就有
\[
    u(\omega_{ji}) = \frac{A_{ji} / B_{ji}}{\ee^{\omega_{ji} \hbar / k T} - 1}.
\]
由于原子能级可以随意调整,我们有
\[
    u(\omega) = \frac{A_{ji} / B_{ji}}{\ee^{\omega \hbar / k T} - 1}.
\]

\[
    u = \frac{\hbar \omega^3}{\pi^2 c^3} \frac{1}{\ee^{\hbar \omega / kT} - 1}
\]
自发发射跃迁率为
\[
    A_{ji} = \frac{\omega_{ji}^3}{3\pi \epsilon_0 \hbar c^3}
\]

\subsubsection{跃迁系数和选择定则}

\begin{equation}
    B_{ji} = \frac{4\pi^2 e^2}{3 \hbar^2} \abs{\expval*{\vb*{r}_{ji}}}^2,
\end{equation}
其中
\begin{equation}
    e \expval*{\vb*{r}_{ji}} = e \int \dd[3]{\vb*{r}} \psi_i^*(\vb*{r}) \vb*{r} \psi_j(\vb*{r})
    \label{eq:electro-dipole}
\end{equation}
为电偶极矩。也即,受激发射系数主要由电偶极矩确定。对波长非常短的光需要考虑电四极辐射。

\eqref{eq:electro-dipole}中的波函数对$(r, \theta, \varphi)$是分离变量的,而
\[
    \begin{aligned}
        r_x &= r \sin \theta \cos \varphi, \\
        r_y &= r \sin \theta \sin \varphi, \\
        r_z &= r \cos \theta 
    \end{aligned}
\]
也是分离变量的。
记$\psi_1$和$\psi_2$的量子数分别是$n_1, l_1, m_1$和$n_2, l_2, m_2$。
\eqref{eq:electro-dipole}给出非零结果的必要条件是其角部分均不为零。
在$\varphi$方向上,积分是
\[
    \int_0^{2\pi} \dd{\varphi} \ee^{ - \ii m_1 \varphi} \cos \varphi \ee^{\ii m_2 \varphi} \vb*{e}_x + \int_0^{2\pi} \dd{\varphi} \ee^{ - \ii m_1 \varphi} \sin \varphi \ee^{\ii m_2 \varphi} \vb*{e}_y + \int_0^{2\pi} \dd{\varphi} \ee^{ - \ii m_1 \varphi} \ee^{\ii m_2 \varphi} \vb*{e}_z,
\]
让三个分量不全为零的可能取值是:
\[
    m_2 - m_1 = \pm 1, 0.
\]
在$\theta$方向上,积分是
\[
    \int_0^\pi \dd{\theta} \sin \theta \legpoly_{l_1}^{m_1} (\cos \theta) \legpoly_{l_2}^{m_2} (\cos \theta) (\vb*{e}_x + \vb*{e}_y) + \int_0^\pi \dd{\theta} \cos \theta \legpoly_{l_1}^{m_1} (\cos \theta) \legpoly_{l_2}^{m_2} (\cos \theta) \vb*{e}_z,
\]
使用勒让德多项式的性质,可以证明让三个分量不全为零的可能取值为
\[
    l_2 - l_1 = \pm 1.
\]
总之,要让受激发射系数不为零,需要
\begin{equation}
    \Delta m = 0, \pm 1, \quad \Delta l = \pm 1.
\end{equation}
这就是\textbf{单电子原子跃迁的选择定则}。

实际上,也可以通过对称性分析得到这个结论。
% TODO:这个过程到底涉及多少光子??

不满足选择定则的跃迁称为\textbf{禁戒跃迁}。通过磁偶极跃迁、电四极子跃迁、双光子跃迁等方法,禁戒跃迁也是可以发生的,但是相对来说发生概率不大,从而对应的能级为亚稳态。

\subsection{磁性}

\subsubsection{磁矩}

一些系统在外加静磁场时能量会增加一项
\[
    E_\text{M} = - \vb*{\mu} \cdot \vb*{B},
\]
其中的矢量$\vb*{\mu}$就称为磁矩。
磁矩和电荷的周期性运动具有非常密切的关系。一个没有内部结构的电荷做周期性运动相当于产生了一个环状电流,因此会产生一个磁矩,称为\textbf{轨道磁矩}。
首先采用经典理论分析轨道磁矩。电子轨道角动量的公式为
\[
    \vb*{L} = m_\text{e} \vb*{r} \times \dv{\vb*{r}}{t} = 2 m_\text{e} \dv{\vb*{S}}{t},
\]
电子被束缚在原子核周围时做平面周期性运动,这样它就产生了一个大小为
\[
    I = - \frac{e}{\tau}
\]
的电流,其中$\tau$是运动周期。
在电磁学中,一个电流为$I$,围绕的面积为$\vb*{S}$的平面线圈的磁矩为
\[
    \vb*{\mu} = I \vb*{S} = - \frac{e \vb*{S}}{\tau},
\]
而由于电子做周期性运动,由角动量守恒我们有
\[
    \dv{\vb*{S}}{t} = \frac{\vb*{S}}{\tau},
\]
这样轨道磁矩就是
\[
    \vb*{\mu} = - \frac{e}{2m_\text{e}} \vb*{L}.
\]
为了与原子物理的背景保持一致,引入下标$l$表示这是来自轨道角动量的磁矩,并且设
\begin{equation}
    \mu_\text{B} = \frac{e\hbar}{2m_\text{e}}
\end{equation}
称为\textbf{玻尔磁子},于是
\begin{equation}
    \vb*{\mu}_l = - \frac{\mu_\text{B}}{\hbar} \vb*{L}.
    \label{eq:orbit-magnetic-moment}
\end{equation}
虽然\eqref{eq:orbit-magnetic-moment}是在经典力学中导出的,但它也适用于量子理论。

量子理论中还有自旋角动量,这是不是会引入自旋磁矩?确实会,不过自旋磁矩的值和把电子当成带电小球计算出来的值并不相同。实际上,自旋磁矩是
\begin{equation}
    \vb*{\mu}_s = - \frac{e}{m_\text{e}} \vb*{S}.
\end{equation}
当然这也不奇怪,因为自旋在粒子图像中并没有经典对应。实际上自旋磁矩的严格计算直接来自QED。

\subsubsection{半经典理论}

轨道角动量有明确的经典意义,于是可以使用半经典理论描述它。\textbf{拉莫尔进动},角动量在一个锥面上运动

如果磁场是不均匀的,那么磁矩不仅受到力矩还受力。

\subsubsection{精细结构}

自旋-轨道耦合。

\section{多电子系统}

多电子处于同一系统时会产生更多有趣的结果。由于电子是费米子,体系的波函数(同时包括轨道部分和自旋部分)一定是交换反对称的。
如果不考虑轨道-自旋耦合,为了保证反对称性,轨道部分对称则自旋部分反对称;轨道部分反对称则自旋部分对称。

\subsection{双电子原子}

\subsubsection{角动量代数}

首先考虑一个双电子原子。如果两个电子的轨道运动相同,那么轨道部分的波函数一定是对称的(如果是反对称的就变成零了),那么自旋部分的波函数一定是反对称的,并且两个电子的自旋一定不相同(否则所有状态都相同,违反泡利不相容原理)。
这样,自旋部分的波函数就是
\begin{equation}
    \chi = \frac{1}{\sqrt{2}} (\chi_{\uparrow 1} \chi_{\downarrow 2} - \chi_{\uparrow 2} \chi_{\downarrow 1}).
    \label{eq:asym-spin}
\end{equation}
不需要其它任何条件,自旋部分的波函数就完全确定了(可以差一个因子但这无关紧要)。
因此轨道部分相同的两个电子的自旋角动量代数是单态的,即$l=0, m=0$。

如果两个电子的轨道运动不同,那么轨道部分的波函数可以是对称的也可以是反对称的。
假定它是对称的,那么自旋部分的波函数一定反对称,因此两个电子的自旋不可能相等。这就意味着自旋部分的波函数还是\eqref{eq:asym-spin}。
而如果轨道部分的波函数是反对称的,那么自旋部分的波函数是对称的。下面我们考虑自旋本征态。
两个电子的自旋如果相等,那么自旋波函数就是以下二者之一:
\[
    \chi_{\uparrow 1} \chi_{\uparrow 2}, \quad \chi_{\downarrow 1} \chi_{\downarrow 2}.
\]
如果两个电子的自旋不相等,那么自旋波函数就应该是$\chi_{\uparrow 1} \chi_{\downarrow 2}$及其交换的线性组合,并且满足对称条件,从而为
\[
    \chi = \frac{1}{\sqrt{2}} (\chi_{\uparrow 1} \chi_{\downarrow 2} + \chi_{\uparrow 2} \chi_{\downarrow 1}).
\]
因此自旋本征态为
\begin{equation}
    \chi = \chi_{\uparrow 1} \chi_{\uparrow 2}, \quad \chi_{\downarrow 1} \chi_{\downarrow 2}, \quad \frac{1}{\sqrt{2}} (\chi_{\uparrow 1} \chi_{\downarrow 2} + \chi_{\uparrow 2} \chi_{\downarrow 1}).
    \label{eq:sym-spin}
\end{equation}
\eqref{eq:sym-spin}是一个三态,$l=1, m=0, \pm 1$。

从角动量代数的角度,两个电子放在一起,它们的角动量代数的复合是两个$l=1/2$的角动量代数的复合,所得结果的$l$取值范围为$0, 1$,和刚才推导得到的一致。
具体$l$取多少由轨道波函数的情况决定。轨道部分如果对称,那么自旋部分必须反对称%
\footnote{注意这是交换对称不是空间对称,空间对称由宇称描述。}%
,这对应$l=0$;反之,轨道部分反对称,则自旋部分必须对称,对应$l=1$。

\subsubsection{交换能}

双电子原子的波函数必须满足交换对称或者反对称条件还意味着,电子之间的库伦能也会发生改变。
两个电子之间的库伦能由
\begin{equation}
    E = \int \dd[3]{\vb*{r}_1} \dd[3]{\vb*{r}_2} \psi^*(\vb*{r}_1, \vb*{r}_2) \frac{1}{4\pi \epsilon_0} \frac{e^2}{\abs{\vb*{r}_1 - \vb*{r}_2}} \psi(\vb*{r}_1, \vb*{r}_2)
\end{equation}
给出。对对称态或者反对称态
\[
    \psi(\vb*{r}_1, \vb*{r}_2) = \frac{1}{\sqrt{2}} (\psi_1(\vb*{r}_1) \psi_2(\vb*{r}_2) \pm \psi_2(\vb*{r}_1) \psi_1(\vb*{r}_2)),
\]
我们有
\begin{equation}
    \begin{aligned}
        E &= \int \dd[3]{\vb*{r}_1} \dd[3]{\vb*{r}_2} \psi_1^*(\vb*{r}_1) \psi_1(\vb*{r}_1) \frac{1}{4\pi \epsilon_0} \frac{e^2}{\abs{\vb*{r}_1 - \vb*{r}_2}} \psi_2^*(\vb*{r}_2) \psi_2(\vb*{r}_2) \\
        &\pm \int \dd[3]{\vb*{r}_1} \dd[3]{\vb*{r}_2} \psi_2^*(\vb*{r}_1) \psi_1(\vb*{r}_1) \frac{1}{4\pi \epsilon_0} \frac{e^2}{\abs{\vb*{r}_1 - \vb*{r}_2}} \psi_1^*(\vb*{r}_2) \psi_2(\vb*{r}_2).
    \end{aligned}
\end{equation}
等式右边第一项就是将电子云密度看成电荷密度计算出来的库伦能,第二项则是一个没有经典对应的项,称为\textbf{交换能}。
可以看到在两个电子的波函数没有很大重叠时交换能可以略去,这也是合理的。

交换能导致两个电子之间同时出现自旋和轨道耦合

\subsection{多电子原子}

\subsubsection{中心场近似}

将一个电子受到其它电子的作用看成一个平均场,即考虑屏蔽作用,

发现能量和角量子数有关,这是因为$n$相同$l$不同的原子径向分布不同,因此受到的屏蔽也不同。角动量大的电子近核概率小,屏蔽效应强,能量高。

原子中所有电子在单电子能级上的分布情况称为\textbf{电子组态},它给出了全部电子的能级的组合,也即,给出了$n$和$l$的组合。
不需要知道完整的电子状态就可以得到电子组态。
电子组态可以使用标准的spdf记号给出,

原子幻数:$Z=2, 10, 18, 36, 54, 86, \ldots$时第一电离能位于峰值,然后一下子到达谷值

原子的壳层结构:$n=1, 2, 3, \ldots$:KLMNOP壳层,但是现在很少用这些符号

$n$:壳层;$l$:支壳层;$m_l$:轨道;$m_s$:至此唯一确定一个态

$l$支壳层有$2(2l+1)$个电子,即$2l+1$个轨道,$2$个自旋
$n$壳层,有$2n^2$个电子

外层电子的能量主要由$l$决定;这就导致了所谓的能级交错现象。

为什么稀有气体非常稳定:$n$p与$(n+1)$s有较大能隙,p支壳层全满的原子不容易激发;内满壳层电子云的电荷分布球对称,对价电子吸引强
\[
    \sum_{m=-l}^l \abs{Y_{lm}(\theta, \varphi)}^2 = \frac{2l+1}{4\pi},
\]
从而
\[
    \rho(\vb*{r}) = -2(2l+1) \frac{e}{4\pi} \chi_{nl}(\vb*{r})
\]

为什么碱金属容易电离:因为原子实中电荷均匀球对称分布,几乎就是一个单独的正电荷,因此价电子收到的束缚非常弱。

为什么卤素容易接受电子:因为容易失去一个空穴

\subsubsection{化学反应}

吸能:失去电子

放能:得到电子、正负电荷中心接近从而降低库伦能

\end{document}