\documentclass[hyperref, UTF8, a4paper]{ctexart}

\usepackage{geometry}
\usepackage{titling}
\usepackage{titlesec}
\usepackage{paralist}
\usepackage{footnote}
\usepackage{enumerate}
\usepackage{amsmath, amssymb, amsthm}
\usepackage{cite}
\usepackage{graphicx}
\usepackage{subfigure}
\usepackage{physics}
\usepackage{tikz}
\usepackage[colorlinks, linkcolor=black, anchorcolor=black, citecolor=black]{hyperref}
\usepackage{prettyref}

\geometry{left=3.18cm,right=3.18cm,top=2.54cm,bottom=2.54cm}
\titlespacing{\paragraph}{0pt}{1pt}{10pt}[20pt]
\setlength{\droptitle}{-5em}
\preauthor{\vspace{-10pt}\begin{center}}
\postauthor{\par\end{center}}

\DeclareMathOperator{\timeorder}{T}
\DeclareMathOperator{\diag}{diag}
\newcommand*{\ii}{\mathrm{i}}
\newcommand*{\ee}{\mathrm{e}}
\newcommand*{\const}{\mathrm{const}}
\newcommand*{\comment}{\paragraph{注记}}

\newrefformat{sec}{第\ref{#1}节}
\newrefformat{note}{注\ref{#1}}
\renewcommand{\autoref}{\prettyref}

\newenvironment{bigcase}{\left\{\quad\begin{aligned}}{\end{aligned}\right.}

\title{统计物理}
\author{wujinq}

\begin{document}

\maketitle

% TODO:各态遍历假设、复合系统、从量子到经典

\section{统计力学的理论框架}

我们将讨论量子系统的统计力学。本文中我们将采用标准的关于“测量”的理论而不分析其背后的原理。
使用$\{\ket{n}\}$表示系统态空间的一组完备正交基。

\subsection{混合态和密度算符}

\subsubsection{引入密度算符}\label{sec:introduction-of-density-operator}

很多量子系统——即使简单如一个单粒子——的态空间都可以分解成一些态空间的直积。
一些时候我们只是关心整个系统的一部分。因此,接下来称我们关心的这部分为\textbf{系统},称我们不关心的部分为\textbf{环境}。
我们设环境完全被算符$\hat{B}$描述而$\hat{A}$是关于系统的一个算符,则系统-环境对的态可以写成这样:
\begin{equation}
    \ket{\text{sys-env}} = \sum_i c_i \ket{\psi_i} \ket{B_i}.
    \label{eq:sys-env-state}
\end{equation}
也就是说我们把系统-环境对的态中含有的所有项都整理成以上形式。
我们还假定系统的演化独立于环境,
这或者是因为环境对系统的作用并不强以至于可以忽略,或者是因为环境对系统的作用如此之强以至于其结果可以很容易地知道而不需要考虑环境的内部状态(例如,考虑原子核对电子的作用)。
然而,虽然系统和环境之间没有耦合,但在制备系统的时候\eqref{eq:sys-env-state}中的$\ket{\psi_i}$和$\ket{B_i}$之间有某种关系,
例如,如果通过裂变的方式制备具有某种自旋的粒子束,那么我们需要的粒子和我们丢弃的粒子放在一起的态就是%
\footnote{可以看到,产生纠缠态还是需要系统和环境发生相互作用。如果系统和环境在$t=-\infty$时就没有发生相互作用,并且它们的动力学彼此不相关,那么它们的态就永远不会有纠缠。但纠缠态产生之后,就算系统和环境不再发生相互作用,纠缠还是会一直存在。}
\[
    \ket{\text{all}} = \frac{1}{\sqrt{2}} \ket{\uparrow} \ket{\uparrow} + \frac{1}{\sqrt{2}} \ket{\downarrow} \ket{\downarrow},
\]
两个粒子一旦被制备就不再有相互作用,但是显然不可能使用$\ket{\uparrow}$和$\ket{\downarrow}$的叠加写出其中任何一个粒子的态——不能够写出
\[
    \ket{\text{all}} = \left( a_1 \ket{\uparrow} + b_1 \ket{\downarrow} \right) \otimes \left( a_2 \ket{\uparrow} + b_2 \ket{\downarrow} \right)
\]
这样的表达式。
把第一个粒子看成系统而第二个粒子看成环境的一部分我们就得到了\eqref{eq:sys-env-state}形式的态。这种“总系统的态不能够写成各个部分的态的直积”情况称为\textbf{量子纠缠}。
这意味着,系统的观察结果不可能完全由诸$\ket{\psi_i}$确定,而必须考虑环境;但是通常环境是什么样的我们并不知道。
因此不能够简单地通过“求解系统-环境对的运动方程”来计算我们关心的结果,而必须通过某种手段把环境“积掉”。
具体怎么做,接下来很快会看到。

可以制备大量的这种系统-环境对,这些系统-环境对之间并没有相互作用,它们的集合称为\textbf{系综}。
系综很自然地导致一个使用古典概型的\textbf{概率}。
通过系综计算出来的概率和真的动手做变量控制得足够好的实验时的概率是一致的:动手做实验时实验结果依赖于某些环境参数$\theta$,
只要环境是足够杂乱,以至于$\theta$的分布完全随机(环境通常足够大,因此总是这样),
那么重复做实验就是采集了大量$\theta$样本,每个$\theta$确定了一个系统,从而建立了一个系综。%
\footnote{当然,系综中系统-环境对的数目总是很大的,而真实的实验做不了那么多次。这里只是原理性地说明系综这一概念的合理性。}%
% TODO:以上说法和退相干等的关系
% 还有这其实有点循环论证的意味:多次实验做出来的概率和系综算出来的概率一致是因为环境提供了很好的随机数,
% 环境为什么能够提供经典意义上的随机数要严格论证还是要用到系综的概念。
% 当然如果假定了多重宇宙那就没有任何问题了——这样我们有了一个“原始系综”,因此可以严格地定义“某个事件发生的概率”:
% 它就是这个事件发生的宇宙的数目除以宇宙总数目。
另一种会自然地要求我们讨论系综的情景是,系统的初态依赖某些参数,而我们并不知道这些参数是什么,于是不得不列出所有可能的参数然后平行地让这些可能的态往前演化,看看具有不同性质的态各占多少。
很容易看出这和“多次做实验”根本就是一回事。

现在我们从这个系综当中随机取出一个系统-环境对,然后对它做一次测量,会得到什么样的结果?
算符$\hat{A}$在系统的态空间和环境的态空间的直积上显然是简并的(无论$\hat{A}$在系统的态空间上是不是简并的)。
测量态\eqref{eq:sys-env-state}的$\hat{A}$值。设$\hat{A}$的本征值$A_i$对应着系统的本征态$\ket{A_i^{(j)}}$,$j=1, 2, \ldots$。
测量结果为$A_i$的概率是
\[
    \begin{aligned}
        P(A_i) &= \sum_{j,k} \abs{\bra{A_i^{(j)}} \bra{B_k} \ket{\text{sys-env}}}^2 \\
        &= \sum_{j, k} \abs{ \bra{A_i^{(j)}} \sum_l c_l \ket{\psi_l} \braket{B_k}{B_l} }^2 \\
        &= \sum_{j, k} \abs{c_k}^2 \abs{\braket{A_i^{(j)}}{{\psi_k}}}^2 \\
        &= \sum_j \ev{ \sum_k \abs{c_k}^2 \dyad{\psi_k} }{A_i^{(j)}}
    \end{aligned}
\]
从系综中随机抽取一个系统-环境对,用$\hat{A}$做测量,得到结果为$A_i$的概率是:
\[
    \begin{aligned}
        P(A_i) &= \sum_j P(\ket{\text{sys-env}_j}) P(\text{$\ket{\text{sys-env}_j}$ gives $A_i$}) \\
        &= \sum_l P(\ket{\text{sys-env}_l}) \sum_j \ev{ \sum_k \abs{c_{l, k}}^2 \dyad{\psi_{l, k}} }{A_i^{(j)}} \\
        &= \sum_j \ev{ \sum_{l, k} P(\ket{\text{sys-env}_l}) \abs{c_{l, k}}^2 \dyad{\psi_{l, k}} }{A_i^{(j)}},
    \end{aligned}
\]
其中下标$l, k$指的是系综中第$l$个系统的第$k$个$\ket{\psi}$(见\eqref{eq:sys-env-state})。
我们定义\textbf{密度算符}
\begin{equation}
    \hat{\rho} = \sum_{l, k} P(\ket{\text{sys-env}_l}) \abs{c_{l, k}}^2 \dyad{\psi_{l, k}},
    \label{eq:density-operator-def}
\end{equation}
就有
\begin{equation}
    P(A_i) = \sum_j \ev{\hat{\rho}}{A_i^{(j)}}.
    \label{eq:prop-of-quantity}
\end{equation}
相应的,期望值为
\[
    \expval*{\hat{A}} = \sum_{i, j} A_i \ev{\hat{\rho}}{A_i^{(j)}}.
\]
注意到
\[
    \sum_{i, j} \ev{\hat{\rho} A_i }{A_i^{(j)}} = \sum_{i, j} \ev{\hat{\rho} \hat{A} }{A_i^{(j)}} = \trace \left(\hat{\rho} \hat{A}\right),
\]
我们得到
\begin{equation}
    \expval*{\hat{A}} = \sum_{i, j} A_i \ev{\hat{\rho}}{A_i^{(j)}} = \trace \left(\hat{\rho} \hat{A}\right).
    \label{eq:expectation}
\end{equation}

在\eqref{eq:density-operator-def}中$\abs{c_{l, k}}^2$相当于是$P(\ket{\psi_{l, k}}|\ket{\text{sys-env}_l})$(归一化性质是显然的),
于是\eqref{eq:density-operator-def}写成
\begin{equation}
    \hat{\rho} = \sum_{i} P(\ket{\psi_i}) \dyad{\psi_i},
    \label{eq:density-operator}
\end{equation}
其中$P(\ket{\psi_i})$指的是从系综中随机取出一个态,经过测量发现它处于$\ket{\psi_i}$的概率。%
\footnote{更准确地说,这是“经过测量之后发现它在各个$\ket{\psi}$态之中处于$\ket{\psi_i}$”的概率。
测量永远是针对一个算符的而不是针对一个单独的态的,对系统做一次测量,观察它会落到诸$\ket{\psi}$中的哪一个的方法是,构造算符
\[
    \hat{A} = \sum_i A_i \dyad{\psi_i},
\]
其中不同的$i$对应不同的$A_i$,然后使用这个算符对系统做测量,若测量结果是某个$A_i$,那么系统就落在了态$\ket{\psi_i}$上。}%
需要注意的是,即使诸$\ket{\psi_i}$相互并不正交,\eqref{eq:density-operator}也是成立的。
\eqref{eq:density-operator}中的每一项的系数都是正的,因此$\hat{\rho}$是正定的。
而又由于\eqref{eq:density-operator}中每一项的系数都小于等于$1$,数学上可以证明,$\hat{\rho}$的本征值全部在$0$和$1$之间,可以取到$1$。
当然,如果\eqref{eq:density-operator}中的某一个$P(\ket{\psi_i})$真的取到了$1$,那么按照概率的性质,此时其余的$P(\ket{\psi_j})$都是零,从而
\[
    \hat{\rho} = \dyad{\psi_i},
\]
因此系统处于纯态。类似地也可以说明,$\hat{\rho}$的本征值取到$1$,当且仅当系统处于纯态;此时$\hat{\rho}$的本征态就是系统的态矢量。

如果可能的$\psi_i$只有一个,那么称此时的系综是\textbf{纯}的,或者说系统处于纯态,因为此时根本不需要引入系综的概念:直接对这个仅有的$\ket{\psi}$解运动方程就可以得到想要的一切信息。
否则,称此时的系综为\textbf{混合}的,或者说系统处于混合态。
需要注意的是即使是纯态也会引入随机性,因为测量所用的算符的本征态未必和$\ket{\psi_i}$一致。
但混合态引入了另一种随机性:我们甚至不知道系统(从系综中随便选取的某一个)具体处于什么态!
这种随机性是由于我们缺乏某些信息而产生的:我们或者不知道和我们关心的系统纠缠的态是什么样的,或者不知道我们关心的系统到底在什么态上面。

通常,对一个系综我们只关心特定的物理量取某些值的概率,以及物理量的期望,后者又可以从前者推出来。
从\eqref{eq:prop-of-quantity}和\eqref{eq:expectation}可以看出,密度算符给出了所有这些信息。
因此我们认为密度算符完整描述了系综。
除了这两项信息以外的信息则不能从密度算符中提取。例如,请注意从\eqref{eq:density-operator}中不能读取出$\ket{\psi_i}$分别都是什么,因为可以找到多组$\ket{\psi_i}$,使用不同的$P$,而得到同样的$\hat{\rho}$,也就是说不同构造的系综可以有同样的密度算符。
通常称诸$\ket{\psi_i}$,也就是有非零系数的态,为\textbf{参与态}。显然密度矩阵提供不了参与态具体是什么的信息,不过一般我们也不需要这些信息。
实际上,\eqref{eq:density-operator}本身就体现了这一点:我们并不关心混合态是因为系统和环境的纠缠还是因为别的什么引起的,因此使用统一的\eqref{eq:density-operator}处理两种情况。

一般来说,对实际的、通常规模很大的系统,我们不可能知道它的所有信息。或者我们不知道它的某些参数,或者我们不知道它是不是和环境纠缠在一起。无论哪种情况,描述系统都需要使用混合态。
因此接下来在不至于引起混淆时我们不严格区分“系统”和“系综”,因为我们根本就不知道“实际上的系统”是什么样子的,而只能讨论系综。于是称纯的系综处于\textbf{纯态},混合系综处于\textbf{混合态}。
相应的,凡是不能够从密度算符中读取得到的信息,我们一概不讨论,因为这些信息不能从系综中读出来。

\subsubsection{时间演化}

下面我们分析密度矩阵的时间演化。我们将只讨论不显含时间的物理量。为了一般性,首先在相互作用绘景下分析问题。此时
\[
    \ii \hbar \dv{t} \ket{\psi^I} = \hat{H}_i^I \ket{\psi^I},
\]
由于系统和环境的演化可认为是彼此独立的,于是系统-环境的时间演化算符是系统的时间演化算符和环境的时间演化算符的直积,两者均为幺正算符,从而随着时间演化,$c_i$不会变化。
另一方面,如果两个态在某一个时刻不同,那么它们不会在某一个连续的时间区间内处处相同;
既然$P(\ket{\text{sys-env}_l})$是通过系综中相同的态的个数除以总个数算出来的,显然我们有
\[
    P_{t_1} (\ket{\text{sys-env}_l (t_1)}) = P_{t_2} (\ket{\text{sys-env}_l (t_2)}).
\]
于是以下我们略去$P$的时间下标以及其括号内的时间标记,因为这个参数对$P$而言没有意义。
因此$P(\ket{\psi_i})$恒定不变。
这样可以推导出
\begin{equation}
    \dv{\hat{\rho}^I}{t} = \frac{1}{\ii \hbar} \comm*{\hat{H}_i^I}{\hat{\rho}^I}.
\end{equation}
请注意这个方程的对易子和算符运动方程的对易子是反的。
由此,我们得出薛定谔绘景中的密度算符演化方程
\begin{equation}
    \dv{\hat{\rho}^S}{t} = \frac{1}{\ii \hbar} \comm*{\hat{H}^S}{\hat{\rho}^S},
\end{equation}
以及海森堡绘景中的密度算符演化方程
\begin{equation}
    \hat{\rho}^H = \const.
\end{equation}
请注意这些方程在$\hbar \to 0$时退化为经典统计力学中的刘维尔方程,因此称其为\textbf{量子刘维尔方程}。

系综达到平衡,也就是说,各个物理量出现的概率都不再发生任何变化的时候,意味着密度算符不变,这又等价于
\begin{equation}
    [\hat{\rho}, \hat{H}] = 0.
    \label{eq:equilibrium-case}
\end{equation}

\subsubsection{密度算符的性质}

现在来分析密度算符的性质。为方便起见以下记
\[
    P(\ket{\psi_i}) = p_i.
\]
首先,
\[
    \trace \hat{\rho} = \sum_n \mel{n}{\hat{\rho}}{n} = \sum_n \mel{n}{\sum_i p_i \dyad{\psi_i}}{n} = \sum_{n, i} p_i \braket{n}{\psi_i} \braket{\psi_i}{n},
\]
于是
\begin{equation}
    \trace \hat{\rho} = 1.
    \label{eq:trace-of-density-operator}
\end{equation}
容易看出导出\eqref{eq:trace-of-density-operator}的论证也可以反过来用。在已知\eqref{eq:trace-of-density-operator}的情况下,可以推知,若$\hat{\rho}$可以被展开为一系列归一化态的叠加
\[
    \hat{\rho} = \sum_i \rho_i \dyad{\psi_i},
\]
则
\[
    \sum_i \rho_i = 1,
\]
无论诸$\ket{\psi_i}$是否正交。通常称$\rho_i$为\textbf{分布函数}。

\eqref{eq:trace-of-density-operator}无论是对纯态还是混合态都是成立的。
然而,$\hat{\rho}^2$的迹却并非如此。对纯态而言
\[
    \hat{\rho}^2 = \dyad{\psi} \dyad{\psi} = \dyad{\psi} = \hat{\rho},
\]
而对混合态,
\[
    \hat{\rho}^2 = \sum_{i, j} p_i p_j \braket{\psi_i}{\psi_j} \dyad{\psi_i}{\psi_j},
\]
从而
\[
    \begin{aligned}
        \trace \hat{\rho}^2 &= \sum_n \mel{n}{\sum_{i, j} p_i p_j \braket{\psi_i}{\psi_j} \dyad{\psi_i}{\psi_j}}{n} \\
        &= \sum_{n, i, j} p_i p_j \braket{\psi_i}{\psi_j} \braket{\psi_j}{n} \braket{n}{\psi_i} \\
        &= \sum_{i, j} p_i p_j \braket{\psi_i}{\psi_j} \braket{\psi_j}{\psi_i} \\
        &=  \sum_{i, j} p_i p_j \abs{\braket{\psi_i}{\psi_j}}^2 \\
        &< \sum_{i, j} p_i p_j = 1 = \trace \hat{\rho}.
    \end{aligned}
\]
上式中我们取小于号而不是小于等于号是因为混合态中诸态不可能全部相互平行。
总之,$\hat{\rho}$幂等的充要条件是它描述了一个纯态,且
\begin{equation}
    \trace \hat{\rho}^2 \begin{cases}
        = 1, \quad & \text{for pure states}, \\
        < 1, \quad & \text{for mixed states}.
    \end{cases}
    \label{eq:inequality-of-mixed-state}
\end{equation}
也就是说密度算符能够提供“纯态还是混合态”的信息。于是可以定义一个密度算符的\textbf{纯度}为
\begin{equation}
    \varsigma = \trace \hat{\rho}^2,
\end{equation}
它越接近$1$说明系统越接近纯态。

此外很容易看出密度算符是厄米的。如果各个参与态相互正交,那么密度算符的本征值就是对应的本征态出现的概率。
当然,各个参与态完全可以不正交,但因为我们从密度算符中并不能判断出哪些是参与态,因此总是可以将密度算符使用它自身的本征态展开,不失一般性地假定各个参与态就是密度算符的本征态。
在各个参与态正交时,可以具体地写出任何一个物理量的期望的公式。我们有
\[
    \begin{aligned}
        \hat{\rho} &= \sum_n P(\ket{n}) \dyad{n}, \\
        \expval*{\hat{A}} &= \trace \hat{\rho} \hat{A} \\
        &= \sum_m \mel{m}{\left(\sum_n P(\ket{n}) \dyad{n} \hat{A} \right)}{m} \\
        &= \sum_{m, n} P(\ket{n}) \braket{m}{n} \mel{n}{\hat{A}}{m}, 
    \end{aligned}
\]
从而
\begin{equation}
    \expval*{\hat{A}} = \sum_n P(\ket{n}) \mel{n}{\hat{A}}{n}.
\end{equation}

\subsubsection{复合系统}

本节将讨论,如果我们已有一个大系统的密度算符,而实际上我们只想讨论其中的一部分的行为,那么要如何写出这个部分的密度算符。
将系统分成两部分,其中一部分称为系统1,另一部分称为系统2。
设$\hat{A}$是只和系统1有关的一个算符。记描述系统2的一组基态为$\ket{\chi_i}$;$\ket{\phi_i}$是系统1的一组态,但它们未必满足正交归一化条件。
则系统的任何一个态均形如
\[
    \ket{\psi} = \sum_{i, j} c_{ij} \ket{\phi_i} \ket{\chi_j},
\]
也就是说我们使用系统2的基矢量展开整个系统的态。
从而整个系统的密度算符形如
\[
    \hat{\rho} = \sum_k p_k \sum_{i,j} \abs{c_{k,ij}}^2 \ket{\phi_i} \ket{\chi_j} \bra{\phi_i} \bra{\chi_j}
\]

现在使用$\hat{A}$对系统1做一次测量,得到$A_i$的概率为
\[
    \begin{aligned}
        P(A_i) &= \sum_{j, k} \bra*{A^{(j)}_i} \bra{\chi_k} \hat{\rho} \ket*{A^{(j)}_i} \ket{\chi_k} \\
        &= \sum_{j, k, l, m, n} p_l \abs{c_{l, mn}}^2 \braket*{A_i^{(j)}}{\phi_m} \braket*{\phi_m}{A_i^{(j)}} \braket{\chi_k}{\chi_n} \braket{\chi_n}{\chi_k} \\
        &= \sum_j \mel*{A_i^{(j)}}{\sum_m \left(\sum_{l, n} p_l \abs{c_{l, mn}}^2 \right) \dyad{\phi_m}}{A_i^{(j)}} 
    \end{aligned}.
\]
记
\[
    \hat{\rho}_1 = \sum_m \left(\sum_{l, n} p_l \abs{c_{l, mn}}^2 \right) \dyad{\phi_m}.
\]
每一项的系数看起来有些复杂,不过请注意
\[
    \abs{c_{l, mn}}^2 = P(\ket{\phi_m} \ket{\chi_n} | \ket{\psi_l}),
\]
有
\[
    \sum_{l, n} p_l \abs{c_{l, mn}}^2 = \sum_{l, n} P(\ket{\psi_l}) P(\ket{\phi_m} \ket{\chi_n} | \ket{\psi_l}) = P(\ket{\phi_m}),
\]
也就是说这个系数就是“从系综中随便取一个态做测量结果发现系统1正好就在$\ket{\phi_m}$上”的概率。
从而我们导出
\[
    \hat{\rho}_1 = \sum_m P(\ket{\phi_m}) \dyad{\phi_m}.
\]
这个表达式的形式和\eqref{eq:density-operator}一模一样。
而系统1经过测量得到$A_i$的概率则是
\[
    P(A_i) = \sum_j \mel{A^{(j)}_i}{\hat{\rho}_1}{A^{(j)}_i},
\]
相应的$\hat{A}$的期望值就是
\[
    \begin{aligned}
        \sum_i A_i P(A_i) &= \sum_{i, j} \mel{A^{(j)}_i}{\hat{\rho}_1 A_i}{A^{(j)}_i} \\
        &= \sum_{i, j} \mel{A^{(j)}_i}{\hat{\rho}_1 \hat{A}}{A^{(j)}_i} = \trace_1 \left(\hat{\rho}\hat{A}\right),
    \end{aligned}
\]
其中$\trace$的下标1表示我们是在系统1的希尔伯特空间上做迹运算。
所有这些结果都和\eqref{eq:prop-of-quantity}和\eqref{eq:expectation}完全一致。
因此我们称$\hat{\rho}_1$为\textbf{约化密度算符}。
容易验证,它可以由
\begin{equation}
    \hat{\rho}_1 = \trace_2 \hat{\rho}
\end{equation}
得到。

很容易就可以看出,以上推导和\autoref{sec:introduction-of-density-operator}中从纠缠态导出密度算符的方式完全一样。
这是当然的,因为系统1可以和系统2有纠缠,因此人为把系统1孤立出来必然导致\autoref{sec:introduction-of-density-operator}节中的操作。

另一方面,设我们有两个相互独立的系统,称为系统1和系统2。
所谓相互独立指的是对其中一个系统做某些操作(或者说,让其中一个系统和另一些东西产生相互作用)不影响另一个系统的状态。例如,对其中一个系统做测量不会影响另一个系统的状态。
设两个系统的密度算符分别为
\[
    \hat{\rho}_1 = \sum_i P(\ket*{\psi_i^{(1)}}) \dyad*{\psi_i^{(1)}}, \quad \hat{\rho}_2 = \sum_i P(\ket*{\psi_i^{(2)}}) \dyad*{\psi_i^{(2)}}.
\]
现在把系统1和系统2看成同一个系统。实际上,我们是把描述系统1的系综和描述系统2的系综拼成了一个大系综。这个大系综中的态可以写成$\ket*{\psi_i^{(1)}} \otimes \ket*{\psi_j^{(2)}}$的形式。
现在使用这一组态对大系统做一次测量,由于系统1和系统2无关,有
\[
    P(\ket*{\psi_i^{(1)}} \otimes \ket*{\psi_j^{(2)}}) = P(\ket*{\psi_i^{(1)}}) P(\ket*{\psi_j^{(2)}}),
\]
从而,大系统的密度矩阵就是
\begin{equation}
    \hat{\rho} = \hat{\rho}_1 \otimes \hat{\rho}_2.
    \label{eq:independent-systems-combinition}
\end{equation}
反之也容易验证,如果\eqref{eq:independent-systems-combinition}成立,那么设$\hat{H}_1$仅仅作用在系统1上,则
\[
    \begin{aligned}
        \dv{t} \hat{\rho}_1 \otimes \hat{\rho}_2 &= \frac{1}{\ii \hbar} \comm*{\hat{H}_1}{\hat{\rho}_1 \otimes \hat{\rho}_2} \\
        &= \frac{1}{\ii \hbar} \comm*{\hat{H}_1}{\hat{\rho}_1} \otimes \hat{\rho}_2,
    \end{aligned}
\]
因此对系统1做的操作不影响系统2,反之亦然。
因此,两个系统独立,当且仅当\eqref{eq:independent-systems-combinition}成立。
这又等价于,
\begin{equation}
    (\trace_2 \hat{\rho}) \otimes (\trace_1 \hat{\rho}) = \hat{\rho}.
\end{equation}

\subsubsection{未归一化的密度算符}

以上讨论的密度算符在定义时保证了其系数真的就是对应的态出现的概率。有时我们能够比较容易地计算出某个态出现的概率正比于某个值,即只知道
\begin{equation}
    P(\ket{\psi_i}) \propto f(\psi_i),
\end{equation}
而不容易将它归一化。此时可以定义未归一化的密度算符为
\begin{equation}
    \hat{\rho} = \sum_i f(\psi_i) \dyad{\psi_i},
\end{equation}
定义\textbf{配分函数}
\begin{equation}
    Z = \sum_i f(\psi_i) = \trace \hat{\rho},
\end{equation}
则$\hat{\rho} / Z$就是归一化的密度算符。使用这个关系,我们得到期望值公式为
\begin{equation}
    \expval*{\hat{A}} = \frac{1}{Z} \trace \left(\hat{\rho} \hat{A}\right) = \frac{\trace \left(\hat{\rho} \hat{A}\right)}{\trace \hat{\rho}},
\end{equation}
在参与态为正交归一化基时这就是
\begin{equation}
    \expval*{\hat{A}} = \frac{1}{Z} \sum_n P(\ket{n}) \mel{n}{\hat{A}}{n}.
\end{equation}
纯度公式为
\begin{equation}
    \varsigma = \frac{\trace \hat{\rho}^2}{\trace \hat{\rho}},
\end{equation}
越接近1说明态越纯。

\subsection{热力学量}

现在我们引入几个量的严格定义。仅仅从这些量的性质出发对系统的能量状况做的分析叫做\textbf{热力学}。

设$\hat{\rho}$是归一化的密度算符。首先定义
\begin{equation}
    S = - \trace (\hat{\rho} \ln \hat{\rho})
    \label{eq:neumann-entropy}
\end{equation}
为\textbf{熵},或称为\textbf{冯诺依曼熵}来和信息熵和经典熵区分。设密度算符被谱展开为
\[
    \hat{\rho} = \sum_n \rho_n \dyad{n},
\]
我们只取其中非零的项。那么熵就可以写成分布函数的函数:
\begin{equation}
    S = - \sum_n \rho_n \ln \rho_n.
\end{equation}
这意味着如果把诸$\ket{n}$一起相同的幺正变换,$S$不变。这就是说,$S$在密度算符做幺正变换时不变,也即
\begin{equation}
    S(\hat{\rho}) = S(\hat{U} \hat{\rho} \hat{U}^{-1}).
\end{equation}
如前所述,$0 < \rho_n \leq 1$,从而$S \geq 0$。

如果系统处于纯态,那么总是有一个态$\ket{\psi}$使密度算符可以写成
\[
    \hat{\rho} = \dyad{\psi},
\]
此时$\rho_n$只有一个,且它的值为$1$,从而$S=0$。反之,如果$S=0$,那么所有的$\rho_n$都是1,因此只有一个$\rho_n$且它是1,因此系统处于纯态。
这意味着熵为$0$是系统处于纯态的充要条件。因此熵可以看成系统偏离纯态的量度,或者说看成“我们对系统有多无知”的量度。

我们已经发现了熵取最小值意味着什么。顺带而来的问题:熵取极大值又意味着什么?我们会看到,这意味着系统达到了平衡态。

设有两个彼此独立的系统,它们各自的密度算符被谱展开为
\[
    \hat{\rho}_1 = \sum_i \rho_i^{(1)} \dyad*{i^{(1)}}, \quad \hat{\rho}_2 = \sum_j \rho_j^{(2)} \dyad*{j^{(2)}},
\]
从而
\[
    \hat{\rho} = \sum_{i,j} \rho_i^{(1)} \rho_j^{(2)} \dyad*{i^{(1)}, j^{(2)}}.
\]
组成的大系统的熵为
\[
    \begin{aligned}
        S(\hat{\rho}) &= - \sum_{i, j} \rho_i^{(1)} \rho_j^{(2)} \ln (\rho_i^{(1)} \rho_j^{(2)}) \\
        &= - \sum_{i, j} \rho_i^{(1)} \rho_j^{(2)} \ln \rho_i^{(1)} - \sum_{i, j} \rho_i^{(1)} \rho_j^{(2)} \ln \rho_j^{(2)} \\
        &= - \sum_i \rho_i^{(1)} \ln \rho_i^{(1)} - \sum_j \rho_j^{(2)} \ln \rho_j^{(2)} \\
        &= S(\hat{\rho}_1) + S(\hat{\rho}_2).
    \end{aligned}
\]
也就是说,彼此独立的系统组成的大系统的熵就是组成它的各个系统的熵之和。

\subsection{平衡态多粒子系统}

\subsubsection{微正则系综}

接下来我们来讨论,当一个系统达到平衡态时它会具有什么性质。在系统很大(从而它不是可积的),并且和外界有小但确实有的相互作用(从而它的演化轨迹会时不时从一条偏移到另外一条上,但每条轨迹又不会有很大偏离)时,我们有\textbf{各态遍历性}:在一段时间内,系统会经过所有可能的态。%
\footnote{这些条件都是必要的:如果系统很简单,比如说,就是理想的二体问题,那么就算系统和环境有纠缠或者持续的小的相互作用也不会各态遍历——可积系统不会热化。}%
所谓可能指的是和系统已知的各个参数一致,例如如果系统和外界无能量交换,那么所有可能的态就是指能量和初始能量相等的态。%
\footnote{类似于“系统有硬边界”这样的条件,如“系统装在一个盒子里”,可以看成是系统受到一个外加势的作用,这个外加势在盒子内部为零,在盒子外部为无穷大,从而由系统能量有限可以知道,系统中的粒子绝对不会跑到盒子外部去。}
系统遍历所有可能的态的时间,也就是\textbf{遍历时间},通常远远小于我们观察的时间尺度。
显然,这就意味着在我们观察的时间尺度上均匀取样地对系统做观察,得到的结果是随机的。%
\footnote{这是又一个虽然没有实际上的随机性,但信息的缺乏意味着我们必须引入概率测度来分析问题的例子。}%
相应的,系统取各个态的概率是相同的。
% TODO:为什么是相同的?
从而,系统具有某个宏观性质的概率就正比于满足这个性质的正交态的数目。

许多系统如果和外界毫无接触,那么并不会有遍历性;但是几乎我们关心的所有系统都或多或少地和外界有小的相互作用。
这种系统和外界虽有小的相互作用但相互作用对系统能量影响不大,以至于系统总能量可以看成是给定的的情况称为\textbf{微正则系综}。

\subsubsection{正则系综}

接下来我们讨论系统和环境有较多能量交换的情况。设系统和环境中的一部分之间有能量传递,且这一部分远大于系统,称其为\textbf{热库}。%
\footnote{由于环境远大于系统,如果热库不远大于系统,总是可以将环境中的另外一些部分加入热库使之远大于系统。}%
较小的系统和热库组成了一个大系统,这个大系统宏观上是封闭的,微观上则可以受到环境中其它部分的微小作用,因此它满足各态遍历假设。我们关心的小系统却未必能够满足这个假设。
我们还要求系统和环境不交换物质,这实际上只是一个人为的约定,因为我们总是可以把一群总量不变的粒子当成一个系统。%TODO:这个假设在哪里被使用了?
总能量为
\begin{equation}
    E_T = E_s + E_r,
    \label{eq:total-energy}
\end{equation}
其中$E_s$指系统能量,$E_r$指热库能量。记态密度为 % TODO:能级实际上总是离散的
\begin{equation}
    d = \dv{\Omega}{E}.
\end{equation}
由于\eqref{eq:total-energy},我们有
\[
    \dd{E_s} \dd{E_r} = \dd{E_s} (\dd{E} - \dd{E_s}) = \dd{E_s} \dd{E},
\]
于是
\[
    \begin{aligned}
        d_T (E_T) &= \int_{E_s + E_r = E_T} \dd{E_s} \dd{E_r} d_r (E_r) d_s (E_s) \\
        &= \int \dd{E_s} \dd{E} d_r (E_T - E_s) d_s (E_s),
    \end{aligned}
\]
从而
\[
    d_T (E_T) = \int \dd{E_s} d_r (E_T - E_s) d_s (E_s).
\]
由于热库远大于系统,$d_s$只有在$E_s$远小于$E_T$时才有比较大的值,因此
\begin{equation}
    d_T (E_T) = d_r (E_T).
\end{equation}
% 这个公式有什么用吗?

我们来分析系统取特定能量的概率。%
\footnote{请注意由于系统可以和热库有相互作用,系统的能量并不是固定的。当然,到平衡态时,系统的能量的期望是固定的。平衡态系统的能量在其期望附加有上下涨落。}
我们称系统处于某个能量上的所有态的总和为\textbf{能级}。
首先假定系统的能级没有简并,或者各个能级的简并情况是相同的(也就是说,每个能级上的简并数相同)。%
\footnote{等价地说就是,系统的态空间可以写成能量空间直积上别的某些自由度。另一种等价的说法是,系统的哈密顿算符$\hat{H}$和别的某几个算符可以构成系统的一组CSCO。}%
由于系统和热库总能量恒定,我们有
\[
    P(E_s) \propto d_r (E_T - E_s),
\]
从而
\[
    \frac{P(E_s)}{P(E_{s'})} = \frac{d_r (E_T - E_s)}{d_r (E_T - E_{s'})}.
\]
等式右边是一个关于$E_T, E_s, E_{s'}$的表达式。因此我们成功地确定了一个事实:$E_s$的变动导致的概率的变动仅仅依赖于$E_s, E_{s'}$和$E_T$。
但实际上,$E_T$并不会改变概率比值。只要满足热库远大于系统,将环境中的什么部分算作热库是完全任意的,因此完全可以将环境中和原热库和系统没有相互作用的一部分算作热库,这增大了$E_T$,却对系统的行为没有任何影响。这表明我们有
\[
    \frac{P(E_s)}{P(E_{s'})} \propto f(E_s, E_{s'}).
\]
最后,注意到系统的能量可以整体地加上一个常数而不影响其行为%
\footnote{我们就是在这里用到了系统无简并或者各个能级的简并情况相同这一假设。如果系统不同能级上的简并数不同,那当我们整体地将能量加上一个常数时还需要调整对应能级的简并数才能够让系统行为保持不变。
例如,设能级$E_0$上的简并数为$n_1$,$E_0 + \Delta E$上的简并数为$n_2$,现在如果将能量整体地加上常数$\Delta E$,那需要重新规定能级$E_0 + \Delta E$上的简并数为$n_1$。}%
,因此上式右边只应该关于$E_s$和$E_{s'}$的差值,从而
\[
    \frac{P(E_s)}{P(E_{s'})} \propto f(E_s - E_{s'}).
\]
满足这个关系的唯一可能就是
\begin{equation}
    P(E_s) \propto \ee^{- \beta E_s}.
    \label{eq:gibbs-distribution-without-degenerate}
\end{equation}
这就是所谓的\textbf{玻尔兹曼分布}或者\textbf{吉布斯分布}。与环境交换物质能量的系统在平衡时一定会落在这个分布上。

进一步,假定系统能级无简并。这样可以很容易地对\eqref{eq:gibbs-distribution-without-degenerate}做归一化,就有
\begin{equation}
    P(E_s) = \frac{1}{Z} \ee^{ - \beta E_s},
    \label{eq:distribution-of-energy-without-degenerate}
\end{equation}
其中
\begin{equation}
    Z = \sum_{E_s} \ee^{ - \beta E_s}
    \label{eq:partition-function-without-degenerate}
\end{equation}
就是在引入未归一化的密度算符时提到的配分函数。相应的,密度算符为
\begin{equation}
    \hat{\rho} = \sum_{E_s} P(E_s) \dyad{E_s} = \sum_{E_s} \ee^{-\beta E_s} \dyad{E_s} = \sum_n \ee^{-\beta E_n} \dyad{n}.
    \label{eq:density-operator-without-degenerate}
\end{equation}
其中$n$是能量本征态的标记,$E_n$指的是$\ket{n}$的本征值。

如果系统的能级有简并,这说明能量$\hat{H}$不足以成为描述系统的CSCO;还需要一些其它的算符来完整地描述系统。我们把它们打包设为$\hat{A}$。
于是,可以先在哈密顿量中引入微小的关于$\hat{A}$的扰动,即
\[
    \hat{H}' = \hat{H} + \lambda \hat{H}_1 (\hat{A}), \quad \lambda \ll 1,
\]
让简并消失,但与此同时不对态空间产生很大影响,此时\eqref{eq:density-operator-without-degenerate}就适用了。现在让$\lambda$趋于零,我们注意到\eqref{eq:density-operator-without-degenerate}中关于$E_s$的所有表达式在$\lambda$取0时会出现突变,因为此时本来能够分开的$E_s$突然变得不能分开了,但是关于$\ket{n}$的表达式的变化却是可微的。
因此有简并时的密度算符为%
\footnote{设算符$\hat{A}$可被谱展开为
\[
    \hat{A} = \sum_i A_i \dyad{i},
\]
则可以验证,一个解析函数作用在$\hat{A}$上的结果为
\[
    f(\hat{A}) = \sum_i f(A_i) \dyad{i}.
\]
}
\begin{equation}
    \hat{\rho} = \sum_n \ee^{-\beta E_n} \dyad{n} = \ee^{-\beta \hat{H}}.
    \label{eq:gibbs-density-operator}
\end{equation}
相应的,可以读出
\begin{equation}
    P(E_i) = \frac{1}{Z} \sum_{E_n = E_i} \mel{n}{\hat{\rho}}{n} = \frac{1}{Z} g_i \ee^{-\beta E_i},
\end{equation}
其中$E_i$指的是第$i$个能级的能量,$g_i$指的是第$i$个能级的简并度。相应的配分函数为
\begin{equation}
    Z = \sum_i g_i \ee^{- \beta E_i} = \sum_\text{eigenstate $\ket{n}$} \ee^{- \beta E_n} .
\end{equation}
第二个等号表示对所有能量本征态求和,当然,此时没有必要使用简并数因子,因为所有态,无论简并不简并,都已被考虑了。

下面计算各物理量的期望值。能量的期望可以直接从配分函数中读出来。注意到
\[
    \expval*{\hat{E}} = \sum_i E_i P(E_i) = \frac{1}{Z} \sum_i E_i g_i \ee^{- \beta E_i},
\]
可以得到
\begin{equation}
    \expval*{\hat{E}} = - \frac{1}{Z} \pdv{Z}{\beta} = - \pdv{\ln Z}{\beta}.
\end{equation}
其余物理量的计算略微麻烦一些。为方便起见,先考虑一个受到外部扰动的哈密顿量
\begin{equation}
    \hat{H}' = \hat{H} + \lambda \hat{A},
\end{equation}
记它的配分函数为
\begin{equation}
    Z(\beta, \lambda) = \trace \ee^{-\beta (\hat{H} + \lambda \hat{A})}
    \label{eq:partition-function-with-disturbance}
\end{equation}
显然,取$\lambda = 0$我们就回退到了没有扰动的系统的配分函数。
我们尝试写出\eqref{eq:partition-function-with-disturbance}在$\lambda$很小时的表达式。
在$\lambda$很小时,$\hat{H} + \lambda \hat{A}$的各个本征态仍然满足正交归一化条件(实际上不管$\lambda$多大都是如此),且相对诸$\ket{n}$只有微小的偏移,由微扰论我们知道,略微偏离态$\ket{n}$的$\hat{H} + \lambda \hat{A}$的本征值约为
\[
    E'_n = E_n + \lambda \mel{n}{\hat{A}}{n},
\]
从而我们有
\[
    \begin{aligned}
        Z(\beta, \lambda) &= \sum_{\text{eigenstate $\ket{n'}$}} \ee^{- \beta E'_n} \\
        &= \sum_{\text{eigenstate $\ket{n}$}} \ee^{- \beta (E_n + \lambda \mel{n}{\hat{A}}{n})}.
    \end{aligned}
\]
第二个等号要求$\lambda$充分小。
因此在$\lambda$很小时,
\[
    \begin{aligned}
        \pdv{Z(\beta, \lambda)}{\lambda} &= \sum_{\text{eigenstate $\ket{n}$}} (-\beta \mel{n}{\hat{A}}{n}) \ee^{-\beta (E_n + \lambda \mel{n}{\hat{A}}{n})} \\
        &= - \beta \sum_{\text{eigenstate $\ket{n}$}} \mel{n}{\hat{A}}{n} \ee^{- \beta E_n} \quad \text{as $\lambda \to 0$},
    \end{aligned}
\]
而
\[
    \expval*{\hat{A}} = \frac{1}{Z} \sum_\text{eigenstate $\ket{n}$} \mel{n}{\hat{A}}{n} \ee^{- \beta E_n},
\]
因此
\begin{equation}
    \expval*{\hat{A}} = - \frac{1}{\beta Z} \eval{\pdv{Z(\beta, \lambda)}{\lambda}}_{\lambda = 0} = - \frac{1}{\beta} \eval{\pdv{\ln Z(\beta, \lambda)}{\lambda}}_{\lambda=0}.
\end{equation}
于是我们从配分函数得到了所有物理量的期望。

在我们有一系列$\{\hat{A}_i\}_i$时,可以定义
\begin{equation}
    Z(\beta, J) = \trace \ee^{-\beta \hat{H} + \sum_i J_i \hat{A}_i },
\end{equation}
仿照上面的论证,可以得到
\begin{equation}
    \expval*{\hat{A}_{k_1} \hat{A}_{k_2} \cdots \hat{A}_{k_n}} = \frac{1}{Z(\beta,0)} \eval{\frac{\partial^n Z}{\partial J_{k_1} \partial J_{k_2} \cdots \partial J_{k_n}}}_{J=0}.
\end{equation}
在连续极限下,$\hat{A}_i$变成了量子场$\hat{A}(\vb*{x})$,

\subsubsection{平衡态热力学}

首先指出一个事实:玻尔兹曼分布\eqref{eq:gibbs-density-operator}是让熵取极大值的分布。可以使用拉格朗日乘子法导出这一点。记$\hat{\rho}$为归一化的密度算符,则%
\footnote{这只是一个条件,但我们只需要说明“所有可能的熵极大值都对应吉布斯平衡态”即可,如果只使用这个条件就能够推出这个结论,那就没有问题。}
\[
    \trace \hat{\rho} = 1.
\]
平衡态时系统的能量的期望恒定,于是
\[
    \trace (\hat{\rho} \hat{H}) = E = \const.
\]
从而,只需要最大化
\[
    S = - \trace (\hat{\rho} \ln \hat{\rho}) \quad \text{s.t.} \; \begin{bigcase}
        \trace \hat{\rho} &= 1, \\
        \trace (\hat{\rho} \hat{H}) &= E
    \end{bigcase}
\]
即可。取目标函数为
\[
    u = - \trace (\hat{\rho} \ln \hat{\rho}) + \gamma (E - \trace (\hat{\rho} \hat{H})) + \gamma' (1 - \trace \hat{\rho}),
\]
对$\hat{\rho}$优化得到
\[
    \ln \hat{\rho} + 1 + \gamma \hat{H} + \gamma' = 0,
\]
从而得到
\[
    \hat{\rho} = \exp (-(1+\gamma')) \exp ( - \gamma \hat{H}).
\]
重新定义常数,得到
\[
    \hat{\rho} = \frac{1}{Z} \ee^{-\beta \hat{H}}.
\]
由于$\hat{\rho}$的迹为1,自然的就有
\[
    Z = \sum_\text{eigenstate $\ket{n}$} \ee^{-\beta E_n}.
\]
这正是玻尔兹曼分布。这表明,任何系统到达平衡态时熵都取极大值。

\subsection{从量子统计退化到经典统计}

\subsection{对单位制的说明}

% 玻尔兹曼常数,等等

\end{document}