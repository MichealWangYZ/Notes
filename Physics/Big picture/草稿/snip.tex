空间平移群是李群,它在$\mathcal{H}_{1\text{d}}$上的幺正表示$\hat{Q}(a)$满足以下特征:
\begin{itemize}
    \item $\hat{Q}(a) \ket{x} = \ket{x+a}$;
    \item $\hat{Q}(a+b) = \hat{Q}(a) \hat{Q}(b)$;
\end{itemize}
$\hat{Q}(a)$是幺正的意味着$\ii \dd{\hat{Q}} / \dd{a}$是厄米的,因此对应一个可观察量。
$\hat{Q}(a+b)=\hat{Q}(a) \hat{Q}(b)$则意味着$\dd{\hat{Q}} / \dd{a}$与群参数$a$无关,即变换是均匀的。
于是设
\begin{equation}
    \hat{Q}(\dd{x}) = \hat{I} + \frac{1}{\ii} \dd{x} \hat{p},
\end{equation}
其中$\hat{p}$是一个不显含任何参量的厄米算符。容易看出它具有动量量纲。
注意到
\[
    \begin{split}
        \hat{x} \hat{Q}(\dd{x'}) \ket{x'} = \hat{x} \ket{x' + \dd{x'}} = (x' + \dd{x'}) \ket{x' + \dd{x'}}, \\
        \hat{Q}(\dd{x'}) \hat{x} \ket{x'} = \hat{Q}(\dd{x'}) x' \ket{x'} = x' \hat{Q}(\dd{x'}) \ket{x'} = x' \ket{x' + \dd{x'}},
    \end{split}
\]
就有
\[
    [\hat{x}, \hat{Q}(\dd{x'})] \ket{x'} = \dd{x'} \ket{x' + \dd{x'}} \approx \dd{x'} \ket{x'}.
\]
考虑到$\ket{x'}$的任意性,我们得到
\[
    [\hat{x}, \hat{Q}(\dd{x'})] = \left[\hat{x}, \hat{I} + \frac{1}{\ii} \dd{x} \hat{p}\right] = \hat{I},
\]
从而
\begin{equation}
    [\hat{x}, \hat{p}] = \ii . 
    \label{eq:x-p-commutator-1d}   
\end{equation}
对易关系\eqref{eq:x-p-commutator-1d}完全确定了$\hat{x}$和$\hat{p}$的李代数结构。

下面推导$\hat{x}, \hat{p}$和任意物理量的对易关系。
设能够将物理量$\hat{F}$展开为$\hat{x}, \hat{p}$的多项式$\hat{F} = F(\hat{x}, \hat{p})$。
对其中的每一项,都可以使用对易关系\eqref{eq:x-p-commutator-1d}把$\hat{x}$挪到最前面而把$\hat{p}$挪到后面,
因此展开式最后就可以写成若干个$a \hat{x}^m \hat{p}^n$形式的项之和。
现在分析其中的一项:
\[
    [\hat{x}, \hat{x}^m \hat{p}^n] = \hat{x}^m [\hat{x}, \hat{p}^n] + [\hat{x}, \hat{x}^m] \hat{p}^n = \hat{x}^m [\hat{x}, \hat{p}^n],
\]
而
\[
    [\hat{x}, \hat{p}^n] = [\hat{x}, \hat{p} \hat{p}^{n-1}] = 
    \hat{p} [\hat{x}, \hat{p}^{n-1}] + [\hat{x}, \hat{p}] \hat{p}^{n-1} = \hat{p} [\hat{x}, \hat{p}^{n-1}] + \ii \hat{p}^{n-1}
\]
于是递推得到
\[
    [\hat{x}, \hat{p}^n] = \ii n \hat{p}^{n-1},
\]
因此
\[
    [\hat{x}, \hat{x}^m \hat{p}^n] = \ii n \hat{x}^m \hat{p}^{n-1}.
\]
这样就可以写出
\begin{equation}
    [\hat{x}, \hat{F}(\hat{x}, \hat{p})] = \ii \pdv{p} \hat{F}(\hat{x}, \hat{p}),
\end{equation}
在作用偏微分符号之前需要先把$F$中的每一项都变形成$\hat{x}$在前$\hat{p}$在后的形式。
使用同样的方法还可以导出
\begin{equation}
    [\hat{p}, \hat{F}(\hat{x}, \hat{p})] = - \ii \pdv{x} \hat{F}(\hat{x}, \hat{p}),
\end{equation}
同样,作用偏微分符号之前需要先把$F$中的每一项都变形成$\hat{x}$在前$\hat{p}$在后的形式。

在海森堡绘景下
\[
    \dv{\hat{A}}{t} = \frac{1}{\ii} [\hat{A}, H] + \pdv{\hat{A}}{t},
\]
于是
\[
    \dv{\hat{x}}{t} = \frac{1}{\ii} [\hat{x}, H] = \pdv{p} \hat{H}(\hat{x}, \hat{p}), \quad
    \dv{\hat{p}}{t} = \frac{1}{\ii} [\hat{p}, H] = -\pdv{x} \hat{H}(\hat{x}, \hat{p})
\]
当$\hbar \to 0$时,上式仍然成立,而此时$\hat{x}$和$\hat{p}$已经是对易的了,因此它们退化为了可以直接使用实数表示的情况,我们也就过渡到了经典力学。
这就是要求$\hat{p}$具有动量量纲的原因——它的确是经典动量的推广。

三维空间平移群在态空间上的幺正表示为$\hat{Q}(\vb*{a})$,同样还是满足与一维情况类似的几个条件:
\begin{itemize}
    \item 平移将位置算符的一个本征态变换为另一个:
    \begin{equation}
        \hat{Q}(\vb*{a}) \ket{\vb*{x}} = \ket{\vb*{x} + \vb*{a}}.
    \end{equation}
    \item 一次完成平移和多次完成是一样的:
    \begin{equation}
        \hat{Q}(\vb*{a}) \hat{Q}(\vb*{b}) = \hat{Q}(\vb*{a} + \vb*{b}).
        \label{eq:movement-plusable}
    \end{equation}
\end{itemize}
同样,\eqref{eq:movement-plusable}意味着生成元不显含群参数。
三维空间平移群本身是三维的,也就是说有三个生成元。不妨设这三个生成元是满足
\begin{equation}
    \begin{split}
        \hat{Q}(\vb*{e}_1 \dd{x_1}) = \hat{I} + \frac{1}{\ii \hbar} \hat{p}_1, \\
        \hat{Q}(\vb*{e}_2 \dd{x_2}) = \hat{I} + \frac{1}{\ii \hbar} \hat{p}_2, \\
        \hat{Q}(\vb*{e}_3 \dd{x_3}) = \hat{I} + \frac{1}{\ii \hbar} \hat{p}_3
    \end{split} 
\end{equation}
的不显含群参量的算符。显然它们是厄米的。
我们可以重复一维的操作步骤来获得三维空间平移群的生成元及其对易关系。