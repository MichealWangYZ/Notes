\documentclass[UTF8]{ctexart}
\usepackage{enumerate}
\usepackage{amsmath, amssymb}
\usepackage{physics}

\newcommand*{\natnums}{\mathbb{N}}
\newcommand*{\reals}{\mathbb{R}}
\newcommand*{\complexes}{\mathbb{C}}

\title{物理中的波动}
\author{wujinq}

\begin{document}

\maketitle

\begin{abstract}
    关于波是什么、波动的性质。
\end{abstract}

\section{传播}
设有一元非常函数$f$,以及函数$\phi(t, \vb*{r}) \in \reals$。现在合理选取$\vb*{r}(t)$使
\[
    f(\phi(t, \vb*{r})) = f(\phi(t + \dd t, \vb*{r} + \dd \vb*{r}))
\]
恒成立。考虑到$f$不是常函数,且$t$变化小量之后$\phi(t, \vb*{r})$只应该发生局部的变化,应当有
\[
    \phi(t, \vb*{r}) = \phi(t + \dd t, \vb*{r} + \dd \vb*{r}) = 
    \phi(t, \vb*{r}) + \pdv{\phi}{t} \dd t + \pdv{\phi}{\vb*{r}} \cdot \dd \vb*{r}
\]
\[
    \pdv{\phi}{\vb*{r}} \cdot \dv{\vb*{r}}{t} = - \pdv{\phi}{t}
\]
考虑到$\vb*{r}$位置的任意性,函数$f(\phi(t, \vb*{r}))$对应一个以
\[
    \vb*{v} = - \pdv{(\partial \phi / \partial t)}{(\partial \phi / \partial \vb*{r})}
\]
速度传播的波前。

\section{经典波动方程}
\begin{equation}
    \nabla^2 \phi - \frac{1}{c^2} \frac{\partial^2 \phi}{\partial t^2} = 0.
    \label{eq:classical-wave-equation}
\end{equation}

\begin{equation}
    \nabla^2 \phi + k^2 \phi = 0
    \label{eq:helmholtz-equation}
\end{equation}

\section{稳态波}

亥姆霍兹方程\eqref{eq:helmholtz-equation}有格林函数
\begin{equation}
    \phi (\boldsymbol{r}, \boldsymbol{r}') = \frac{e^{\mathrm{i}kR}}{4\pi R}, \; R = |\boldsymbol{r} - \boldsymbol{r}'|
\end{equation}
使得
\[
    \nabla^2 \phi + k^2 \phi = - \delta(\boldsymbol{r} - \boldsymbol{r}')
\]
现在我们假设$\psi$是亥姆霍兹方程的解,

\section{散射}

\end{document}