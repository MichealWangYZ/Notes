\documentclass[hyperref, UTF8, a4paper]{ctexart}

\usepackage{geometry}
\usepackage{titling}
\usepackage{titlesec}
\usepackage{paralist}
\usepackage{footnote}
\usepackage{enumerate}
\usepackage{amsmath, amssymb, amsthm}
\usepackage{bbm}
\usepackage{cite}
\usepackage{graphicx}
\usepackage{subfigure}
\usepackage{physics}
\usepackage{tikz}
\usepackage{autobreak}
\usepackage[ruled, vlined, linesnumbered, noend]{algorithm2e}
\usepackage[colorlinks, linkcolor=black, anchorcolor=black, citecolor=black]{hyperref}
\usepackage{prettyref}

% Page style
\geometry{left=3.18cm,right=3.18cm,top=2.54cm,bottom=2.54cm}
\titlespacing{\paragraph}{0pt}{1pt}{10pt}[20pt]
\setlength{\droptitle}{-5em}
\preauthor{\vspace{-10pt}\begin{center}}
\postauthor{\par\end{center}}

% Math operators
\DeclareMathOperator{\timeorder}{T}
\DeclareMathOperator{\diag}{diag}
\DeclareMathOperator{\legpoly}{P}
\DeclareMathOperator{\primevalue}{P}
\DeclareMathOperator{\sgn}{sgn}
\newcommand*{\ii}{\mathrm{i}}
\newcommand*{\ee}{\mathrm{e}}
\newcommand*{\const}{\mathrm{const}}
\newcommand*{\comment}{\paragraph{注记}}
\newcommand*{\suchthat}{\quad \text{s.t.} \quad}
\newcommand*{\argmin}{\arg\min}
\newcommand*{\argmax}{\arg\max}
\newcommand*{\normalorder}[1]{: #1 :}
\newcommand*{\pair}[1]{\langle #1 \rangle}
\newcommand*{\fd}[1]{\mathcal{D} #1}
\DeclareMathOperator{\bigO}{\mathcal{O}}

% prettyref setting
\newrefformat{sec}{第\ref{#1}节}
\newrefformat{note}{注\ref{#1}}
\newrefformat{fig}{图\ref{#1}}
\newrefformat{alg}{算法\ref{#1}}
\renewcommand{\autoref}{\prettyref}

% TikZ setting
\usetikzlibrary{arrows,shapes,positioning}
\usetikzlibrary{arrows.meta}
\usetikzlibrary{decorations.markings}
\tikzstyle arrowstyle=[scale=1]
\tikzstyle directed=[postaction={decorate,decoration={markings,
    mark=at position .5 with {\arrow[arrowstyle]{stealth}}}}]
\tikzstyle ray=[directed, thick]
\tikzstyle dot=[anchor=base,fill,circle,inner sep=1pt]

% Algorithm setting
\renewcommand{\algorithmcfname}{算法}
% Python-style code
\SetKwIF{If}{ElseIf}{Else}{if}{:}{elif:}{else:}{}
\SetKwFor{For}{for}{:}{}
\SetKwFor{While}{while}{:}{}
\SetKwInput{KwData}{输入}
\SetKwInput{KwResult}{输出}
\SetArgSty{textnormal}

\renewcommand{\emph}[1]{\textbf{#1}}
\newcommand*{\concept}[1]{\underline{\textbf{#1}}}
\newcommand*{\Ztwo}{$\mathbb{Z}_2$}

\title{常见格点模型}
\author{吴何友}

\begin{document}

\maketitle

\section{相互作用体系}

\subsection{Hubbard模型}

\begin{equation}
    \hat{H} = - t \sum_{\pair{i, j}, \sigma} \hat{c}^\dagger_{i \sigma} \hat{c}_{j \sigma} + U \sum_i \hat{n}_\uparrow \hat{n}_\downarrow,
\end{equation}

\section{自旋模型}

\subsection{自旋自由度}

我们将主要讨论自旋$1/2$的情况;不过,为简化公式形式,我们将认为$z$方向自旋可以取$\pm 1$而不是$\pm 1/2$。
在自旋$1/2$的系统中有很多非常方便的性质,如$\hat{\sigma}^x$可以将$\hat{\sigma}^z$的本征态翻转,即将$\ket{\sigma^z=1}$变为$\ket{\sigma^z=-1}$,反之亦然。
此外,我们有
\begin{equation}
    \braket{\sigma^x}{\sigma^z} = \frac{1}{\sqrt{2}} \exp(\ii \pi \frac{1 - \sigma^x}{2} \frac{1 - \sigma^z_l}{2}),
\end{equation}
并且实际上上式是一个实数。

\subsection{伊辛模型}

\subsubsection{经典伊辛模型}

所谓\concept{经典伊辛模型}指的是
\begin{equation}
    \hat{H} = - \sum_{\pair{i, j}} J_{ij} \hat{S}_i^z \hat{S}_j^z + \sum_{i} B_i \hat{S}_i^z,
\end{equation}
它被称为是\emph{经典}的是因为其能量本征态就是($z$方向上的)自旋本征态,因此一个伊辛模型的统计配分函数就是简单的
\begin{equation}
    Z = \sum_{\{s_i\}} \exp(\beta \sum_{\pair{i, j}} J_{ij} s_i s_j - \beta \sum_i B_i s_i),
\end{equation}
即我们可以将$\hat{S}_i^z$当成经典变量。当然也可以使用量子的路径积分来写出配分函数,但由于这样明显麻烦,通常不会这么做。

\subsubsection{横场伊辛模型}

\concept{横场伊辛模型}定义为
\begin{equation}
    \hat{H} = - \sum_{\pair{i, j}} J_{ij} \hat{S}_i^z \hat{S}_j^z + \sum_{i} B_i \hat{S}_i^x,
\end{equation}
唯一的区别就是磁场加在$x$方向而不是$z$方向(即\emph{横场}),但是这个区别意味着横场伊辛模型具有零温量子涨落。

一个$d$维量子横场伊辛模型对应一个$d+1$维的经典伊辛模型。要看出这是为什么,注意到
\[
    \begin{aligned}
        \mel{S^z(\tau + \Delta \tau)}{\ee^{-\Delta \tau \hat{H}}}{S^z(\tau)} &= \mel{S^z(\tau + \Delta \tau)}{\ee^{-\Delta \tau \sum_i B_i \hat{S}^x_i} \ee^{\Delta \tau \sum_{\pair{i, j}} J_{ij} \hat{S}_i^z \hat{S}_j^z}}{S^z(\tau)} \\
        &= \ee^{\Delta \tau \sum_{\pair{i, j}} J_{ij} S_i^z S_j^z} \mel{S^z(\tau + \Delta \tau)}{\ee^{-\Delta \tau \sum_i B_i \hat{S}^x_i}}{S^z(\tau)} \\
        &= \ee^{\Delta \tau \sum_{\pair{i, j}} J_{ij} S_i^z S_j^z} \sum_{\{S^x_i\}} \ee^{-\Delta \tau \sum_i B_i S^x_i} \braket{S^z(\tau + \Delta \tau)}{S^x} \braket{S^x}{S^z(\tau)} \\
        &= \ee^{\Delta \tau \sum_{\pair{i, j}} J_{ij} S_i^z S_j^z} \prod_{i} \sum_{S^x_i} \ee^{-\Delta \tau B_i S^x_i} \braket{S^z_i(\tau + \Delta \tau)}{S^x_i} \braket{S^x_i}{S^z_i(\tau)} \\
        &= \ee^{\Delta \tau \sum_{\pair{i, j}} J_{ij} S_i^z S_j^z} \prod_{i} \sum_{S^x_i = \pm 1} \ee^{-\Delta \tau B_i S^x_i} \frac{1}{2} \ee^{\ii \pi \frac{1 - S^x_i}{2} \left( \frac{1 - S^z_i(\tau)}{2} + \frac{1 - S^z_i(\tau + \Delta \tau)}{2} \right)} \\
        &= \frac{1}{2^N} \ee^{\Delta \tau \sum_{\pair{i, j}} J_{ij} S_i^z S_j^z} \prod_i \left( \ee^{- \Delta \tau B_i} + \ee^{\Delta \tau B_i} \ee^{\ii \pi \frac{1 - S^z_i(\tau)}{2}} \ee^{\ii \pi \frac{1 - S^z_i(\tau + \Delta \tau)}{2}} \right) \\
        &= \frac{1}{2^N} \ee^{\Delta \tau \sum_{\pair{i, j}} J_{ij} S_i^z S_j^z} \prod_i \left( \ee^{- \Delta \tau B_i} + \ee^{\Delta \tau B_i} S^z_i(\tau) S^z_i(\tau + \Delta \tau) \right),
    \end{aligned}
\]
请注意$S_i^z$只取$\pm 1$,于是我们有
\[
    \cosh J_\tau + \sinh J_\tau S^z_i(\tau) S^z_i(\tau + \Delta \tau) = \ee^{J_\tau S^z_i(\tau) S^z_i(\tau + \Delta \tau)},
\]
那么只需要定义
\begin{equation}
    \tanh J^\tau_i = \ee^{2 \Delta \tau B_i}, 
\end{equation}
就有
\[
    \begin{aligned}
        \mel{S^z(\tau + \Delta \tau)}{\ee^{-\Delta \tau \hat{H}}}{S^z(\tau)} &\propto \ee^{\Delta \tau \sum_{\pair{i, j}} J_{ij} S_i^z S_j^z} \prod_i \ee^{J_i^\tau S_i^z(\tau) S_i^z(\tau + \Delta \tau)} \\
        &= \ee^{\Delta \tau \sum_{\pair{i, j}} J_{ij} S_i^z S_j^z} \ee^{\sum_i J_i^\tau S_i^z(\tau) S_i^z(\tau + \Delta \tau)},
    \end{aligned}
\]
容易看出这最终导致一个三维各向异性(虚时间维的$J$和空间维的$J$无关)经典伊辛模型的配分函数。

\subsection{自旋链模型}

\subsubsection{XXZ模型}

一个\concept{XXZ模型}指的是这样的一个哈密顿量:
\begin{equation}
    \hat{H} = J_x \sum_i (\hat{S}_{i}^x \hat{S}_{i+1}^x + \hat{S}_{i}^y \hat{S}_{i+1}^y) + J_z \sum_{i} \hat{S}_i^z \hat{S}_{i+1}^z.
\end{equation}

\section{磁场}

将电子和一个满足库伦规范的磁矢势$\vb*{A}$耦合,那么会出现动量的一个修正,这个修在在波函数上引入如下的相位变化:
\begin{equation}
    \theta = \int \dd{\vb*{l}} \cdot \vb*{A}.
\end{equation}
在格点模型中,电子仅仅出现在格点上。我们知道紧束缚模型的哈密顿量(即跃迁项)实际上就是动能,因此加入磁场意味着紧束缚模型的$t_{ij}$出现变化,考虑相位变化,则磁场会导致以下修正:
\begin{equation}
    t_{ij} \longrightarrow \ee^{\ii e \int_j^i \dd{\vb*{l}} \cdot \vb*{A} } t_{ij}.
\end{equation}
相应的,设一个格点上的闭合路径为$C$,通过它的磁通量为$\Phi$,则
\begin{equation}
    \ee^{\ii \Phi} = \prod_{C} t_{ij}.
\end{equation}

\end{document}