\documentclass[hyperref, UTF8, a4paper]{ctexart}

\usepackage{geometry}
\usepackage{titling}
\usepackage{titlesec}
\usepackage{paralist}
\usepackage{footnote}
\usepackage{enumerate}
\usepackage{amsmath, amssymb, amsthm}
\usepackage{bbm}
\usepackage{cite}
\usepackage{graphicx}
\usepackage{subfigure}
\usepackage{physics}
\usepackage{tikz}
\usepackage[colorlinks, linkcolor=black, anchorcolor=black, citecolor=black]{hyperref}
\usepackage{prettyref}

\geometry{left=3.18cm,right=3.18cm,top=2.54cm,bottom=2.54cm}
\titlespacing{\paragraph}{0pt}{1pt}{10pt}[20pt]
\setlength{\droptitle}{-5em}
\preauthor{\vspace{-10pt}\begin{center}}
\postauthor{\par\end{center}}

\DeclareMathOperator{\timeorder}{T}
\DeclareMathOperator{\diag}{diag}
\DeclareMathOperator{\legpoly}{P}
\DeclareMathOperator{\primevalue}{P}
\DeclareMathOperator{\sgn}{sgn}
\newcommand*{\ii}{\mathrm{i}}
\newcommand*{\ee}{\mathrm{e}}
\newcommand*{\const}{\mathrm{const}}
\newcommand*{\comment}{\paragraph{注记}}
\newcommand*{\suchthat}{\quad \text{s.t.} \quad}
\newcommand*{\argmin}{\arg\min}
\newcommand*{\argmax}{\arg\max}
\newcommand*{\normalorder}[1]{: #1 :}
\newcommand*{\pair}[1]{\langle #1 \rangle}
\newcommand*{\fd}[1]{\mathcal{D} #1}

\newrefformat{sec}{第\ref{#1}节}
\newrefformat{note}{注\ref{#1}}
\newrefformat{fig}{图\ref{#1}}
\renewcommand{\autoref}{\prettyref}

\usetikzlibrary{arrows,shapes,positioning}
\usetikzlibrary{arrows.meta}
\usetikzlibrary{decorations.markings}
\tikzstyle arrowstyle=[scale=1]
\tikzstyle directed=[postaction={decorate,decoration={markings,
    mark=at position .5 with {\arrow[arrowstyle]{stealth}}}}]
\tikzstyle ray=[directed, thick]
\tikzstyle dot=[anchor=base,fill,circle,inner sep=1pt]

\title{伊辛模型和统计场论}
\author{吴何友}

\begin{document}

\maketitle

本文如无说明,采用自然单位制。

\section{伊辛模型的定义}

考虑一个$d$维空间中的晶格,共有$N$个格点。每个格点上有一个自旋。我们照惯例采用$z$方向的自旋来标记每个格点的状态,记之为$s_i$而省略$z$下标。
在\textbf{伊辛模型}中,假定:
\begin{enumerate}
    \item 格点$i$处的自选取值为$\pm 1$(具体值其实不重要,因为它们在尺度变换之下可以改变;关键是,自旋只有两个取值);
    \item 只有相邻格点会发生相互作用;
    \item 自旋-自旋相互作用仅仅含有关于最近邻两格点的$\hat{s}_i \hat{s}_j$形式的项,也就是说,自旋的$x$和$y$分量不参与相互作用;\footnote{需要注意在实际的材料中这样的相互作用并不容易制备,但我们姑且假定相互作用是这样的形式。}
    \item 外部磁场和自旋线性耦合;
\end{enumerate}
由于不同格点的自旋算符显然是对易的,$\{s_i\}$表象就是系统的能量本征态,且
\begin{equation}
    E = - B \sum_i s_i - J \sum_{\pair{i,j}} s_i s_j.
    \label{eq:ising-energy}
\end{equation}
这里我们略去了第一项中的常数,把它们全部归入磁场中。

显然,在$J>0$且绝对值很大时,自旋倾向于同向排列,即形成一个\textbf{铁磁序};而当$J<0$而绝对值很大时,自旋倾向于一上一下地排列,形成一个\textbf{反铁磁序}。

有限温下,我们写出\eqref{eq:ising-energy}对应的配分函数和热力学自由能:
\[
    Z(T, J, B) = \sum_{\{s_i\}} \ee^{- \beta E[s_i]}, \quad F_\text{thermo} = \expval{E} - TS  = - T \ln Z.
\]
原则上这就包含了关于系统的全部信息。

带有相互作用的配分函数通常很难计算。如果我们关心的只是比较大尺度上的磁化情况,那么其实并不需要完整的$\{s_i\}$。
定义$m(\vb*{x})$为在某个长度尺度$L$上平均化的$\{s_i\}$,$F[m(\vb*{x})]$是它对应的有效自由能。
也即,我们将$\{s_i\}$做空间傅里叶变换,把特征尺度小于$L$的成分全部积掉,就得到
\[
    F[m(\vb*{x})] = \sum_{\{s_i\} | m} E[s_i], \quad Z = \sum_m \ee^{-\beta F[m(\vb*{x})]}.
\]

严格从\eqref{eq:ising-energy}出发计算$F[m(\vb*{x})]$一般是非常困难的。
通常我们使用朗道-金斯堡方法,即根据\eqref{eq:ising-energy}的对称性写出$F[m(\vb*{x})]$的普遍形式。
这样写出的$F[m(\vb*{x})]$中的各个常数都是没有确定的,要确定这些常数肯定需要从\eqref{eq:ising-energy}出发严格计算。
这些使用统计场论处理问题的方式见\autoref{sec:ising-effective-field}和\autoref{sec:ising-rg}。

\section{平均场处理方法}

\subsection{均匀系统}

本节讨论一个非常简单的情况:我们把截断尺度$L$选取到整个体系,从而使用一个单独的$m$描述整个体系。
此时
\begin{equation}
    m = \frac{1}{N} \sum_i s_i,
\end{equation}
容易看出$-1 < m < 1$,它就是平均磁化强度。实际上它是一个序参量,因为如果系统处于顺磁相那么$m$应该接近$1$或者$-1$,而如果系统处于无序相那么$m$应该接近$0$。
$m$本质上仍然是离散的,它能够发生的最小变化是一个$-1$格点变成$+1$,即它能够发生的最小变化是$2/N$。
在大$N$极限下有
\begin{equation}
    Z = \frac{N}{2} \int_{-1}^1 \dd{m} \ee^{-\beta F(m)}.
\end{equation}
下面要做的是计算出$F(m)$。我们假定系统充分均匀,则将\eqref{eq:ising-energy}中的所有$s_i$和$s_j$都使用$m$代替,从而得到
\begin{equation}
    \frac{E}{N} = - B m - \frac{1}{2} J q m^2,
    \label{eq:ising-energy-m-approx}
\end{equation}
其中$q$是每个格点周围最近邻格点的数目,和维数有关。请注意这一步近似\eqref{eq:ising-energy-m-approx}的可靠性完全没有保障:我们并不知道$\{s_i\}$的涨落情况,自然不可能知道将它们全部替换成$m$之后是不是能够得到足够精确的结果。
正如通常的平均场理论一样,这只是一个近似的起步,需要在此基础上逐阶考虑涨落的影响。
不过我们还是先姑且按照这个假设算下去,仅仅因为它方便处理。
由定义可以得到
\[
    m = \frac{N_\uparrow - N_\downarrow}{N} = \frac{2N_\uparrow - N}{N}, \quad \Omega = \frac{N!}{N_\uparrow!(N-N_\uparrow)!},
\]
使用斯特林公式可以计算出以下近似值
\begin{equation}
    \ln\Omega(m) = - \frac{m+1}{2} N \ln\frac{m+1}{2} - \frac{1-m}{2} N \ln\frac{1-m}{2},
    \label{eq:lattice-entropy}
\end{equation}
设$f(m)$为平均每个格点的有效自由能,则根据
\[
    \ee^{-\beta F(m)} = \Omega(m) \ee^{-\beta E},
\]
即
\[
    -\beta N f(m) = \ln \Omega(m) - \beta E,
\]
可以计算出
\begin{equation}
    f(m) = - Bm - \frac{1}{2} J q m^2 + T \left( \frac{m+1}{2} \ln(m+1) + \frac{1-m}{2} \ln(1-m) - \ln 2 \right).
    \label{eq:free-energy-of-m}
\end{equation}
取鞍点近似,由于$F(m)$的极小值就是$f(m)$的极小值,我们有
\[
    \eval{\pdv{f}{m}}_{m=\expval*{m}} = 0,
\]
于是发现热平衡时的平均自旋$\expval*{m}$满足
\begin{equation}
    m = \tanh(\beta B + \beta J q m).
    \label{eq:self-consistency-m-approx}
\end{equation}
我们得到了一个自洽方程。

\subsection{平均场分解}

实际上,也可以显式地将相互作用项做平均场分解来得到\eqref{eq:self-consistency-m-approx}:做分解
\[
    \begin{aligned}
        s_i s_j &\approx \expval*{s_i} s_j + s_i \expval*{s_j} - \expval*{s_i} \expval*{s_j} \\
        &\approx \expval*{m} (s_i + s_j) - \expval*{m}^2,
    \end{aligned}
\]
其中第二个约等号表示系统的均匀性,即每个点的自选取值都和$m$相差不大,并注意到
\[
    \sum_i s_i = N m,
\]
\[
    \sum_{\pair{i,j}} \expval*{m} s_i = \frac{1}{2} \expval*{m} q \sum_i s_i = \frac{1}{2} \expval*{m} q N m,
\]
而
\[
    \sum_{\pair{i,j}} m^2 = \frac{1}{2} \expval*{m}^2 q N,
\]
则得到平均场哈密顿量
\begin{equation}
    E = - B N m - J q N \expval*{m} m + \frac{1}{2} J q N \expval*{m}^2.
    \label{eq:ising-mean-field-energy}
\end{equation}
可以看到除了一个常数项以外,相当于有一个大小为
\begin{equation}
    B_\text{eff} = B + J q \expval*{m}
\end{equation}
的场被作用在了$m$上。使用\eqref{eq:ising-mean-field-energy}连同\eqref{eq:lattice-entropy},根据热力学第一定律就可以推导出\eqref{eq:self-consistency-m-approx}。
这也就是我们称它为平均场解的原因。

\subsection{相变}

我们只是知道\eqref{eq:self-consistency-m-approx}给出了所有解,但是并不知道这些解有几个,是不是稳定等。
回顾\eqref{eq:free-energy-of-m},做泰勒展开得到
\begin{equation}
    f(m) = - T \ln 2 - Bm + \frac{1}{2} (T-Jq) m^2 + \frac{1}{12} T m^4 + \cdots,
\end{equation}
在$m$和$B$均不大时取开头四项已经足够确定自由能曲线的形状了。

\subsubsection{无磁场情况下的二级相变}

\begin{figure}
    \centering
    \subfigure[$T>T_\text{c}$的自由能曲线]{
        \begin{tikzpicture}
            
            % m横轴
            \draw[->] (-2.5,0) -- (2.5,0) node[right] {$m$};
            % 自由能纵轴
            \draw[->] (0,-0.5) -- (0,5.5) node[above] {$f(m)$};
            
            % 画出自由能
            \draw[samples=50, smooth, domain=-2.2:2.2] plot(\x,{0.5*(\x*\x)+0.11*(\x*\x*\x*\x)});
    
            % 标出极小值点
            \node[circle,fill,inner sep=1.5pt] at (0, 0) {};
    
        \end{tikzpicture}
        \label{fig:small-t-free-energy}
    }
    \subfigure[$T<T_\text{c}$的自由能曲线]{
        \begin{tikzpicture}
            
            % m横轴
            \draw[->] (-2.5,0) -- (2.5,0) node[right] {$m$};
            % 自由能纵轴
            \draw[->] (0,-1.5) -- (0,4.5) node[above] {$f(m)$};
            
            % 画出自由能
            \draw[samples=50, smooth, domain=-2.2:2.2] plot(\x,{-0.5*(\x*\x)+0.25*(\x*\x*\x*\x)});
    
            % 标出极小值点
            \node[circle,fill,inner sep=1.5pt] at (1, -0.25) {};
            \node[circle,fill,inner sep=1.5pt] at (-1, -0.25) {};
    
        \end{tikzpicture}
        \label{fig:big-t-free-energy}
    }
    \caption{不同形状的自由能曲线}
    \label{fig:chemical-potential}
\end{figure}

在$B=0$时$m=0$肯定是一个极小值点,稳定不稳定暂时不论。
略去自由能中无关紧要的常数项,$T-Jq$的正负决定了曲线在$m=0$附加是上凸还是下凹。
如果它是正的,那么曲线下凹,而由于四次方项也是正的,整个曲线都是下凹的,也就是说只有$m=0$一个极小值点,而且这是稳定的极小值点,如\autoref{fig:small-t-free-energy}。
如果它是负的那么曲线在$m=0$附近上凸,$m=0$不是稳定解,于是出现两个极小值点,如\autoref{fig:big-t-free-energy}。

这意味着在$B=0$时,当$T$经过
\begin{equation}
    T_\text{c} = Jq
\end{equation}
时出现了一个相变,因为随着$T$的增大,不为零的序参量$m$变为零且一直停留在零。
在\autoref{fig:big-t-free-energy}中出现了自发对称性破缺,原本具有的自旋翻转对称性被破缺了。
因此$T<T_\text{c}$时出现了一个铁磁序,这个铁磁序破缺了自旋翻转对称性;$T>T_\text{c}$对应一个无序相。
计算稳态时序参量$m$的值,可以得到
\begin{equation}
    m = \begin{cases}
        \pm \sqrt{\frac{3(T_\text{c}-T)}{T}}, \quad T < T_\text{c}, \\
        0, \quad T > T_\text{c},
    \end{cases}
    \label{eq:mf-stable-m}
\end{equation}
上式绘图如\autoref{fig:mf-m-and-t-relation}。

\begin{figure}
    \centering
    \begin{tikzpicture}
            
        % T横轴
        \draw[->] (-1,0) -- (6,0) node[right] {$T$};
        % 磁化强度纵轴
        \draw[->] (0,-3) -- (0,3) node[above] {$m$};
        
        % 绘制铁磁相的曲线
        \draw[samples=50, smooth, thick, domain=0:2.5] plot(\x, {sqrt(2.5-\x)});
        \draw[samples=50, smooth, thick, domain=0:2.5] plot(\x, {-sqrt(2.5-\x)});

        % 绘制无序相的曲线
        \draw[thick] (2.5, 0) -- (5,0);

        % 标出极小值点
        \node[dot, label=above right:$T_\text{c}$] at (2.5, 0) {};

    \end{tikzpicture}
    \caption{稳态时磁化强度序参量和温度的关系}
    \label{fig:mf-m-and-t-relation}
\end{figure}
\autoref{fig:mf-m-and-t-relation}中$x$轴和$y$轴的交点不是零;
我们没有把$T$外推到$0$,因为显然此时\eqref{eq:mf-stable-m}是错误的——它没有能够保证$-1<m<1$,而是给出了发散的结果。

在$T_\text{c}$附近的相变实际上是一个二级相变。这是因为$m$是连续的,而$F(m)$是$m$的函数,从而也是连续的;但是$F(m)$并不光滑,因为\eqref{eq:mf-stable-m}并不光滑,因此$F(m)$的导数不连续,从而是二级相变。

\subsubsection{一级相变}

上一节的结果都是假定了$B=0$,然后在这条等磁线上我们有一个二级相变。显然按照\eqref{eq:free-energy-of-m},磁场变化会让自由能的极小值点发生偏移,如\autoref{fig:mf-b-and-free-energy}所示。
当$T>T_\text{c}$时,只有一个偏离$m=0$的稳定极小值点;当$T<T_\text{c}$时,两个极小值点中的一个的自由能变高,变为亚稳态,或者甚至直接变成鞍点,而另一个则为全局最小值。
$B>0$时和$B<0$时热力学稳态分别位于$m>0$和$m<0$位置,$m$本身就不连续,因此这是一个一级相变。
当$T>T_\text{c}$时随着$B$的变化热力学稳态是连续变化的,即不存在相变。

\begin{figure}
    \centering
    \subfigure[$T>T_\text{c}, B>0$的自由能曲线]{
        \begin{tikzpicture}
            
            % m横轴
            \draw[->] (-2,0) -- (2.5,0) node[right] {$m$};
            % 自由能纵轴
            \draw[->] (0,-0.5) -- (0,5.5) node[above] {$f(m)$};
            
            % 画出自由能
            \draw[samples=50, smooth, domain=-1.9:2.5] plot(\x,{-1*\x+0.5*(\x*\x)+0.11*(\x*\x*\x*\x)});
    
            % 标出极小值点
            \node[circle,fill,inner sep=1.5pt] at (0.82, -0.43) {};
    
        \end{tikzpicture}
        \label{fig:small-t-positive-b-free-energy}
    }
    \subfigure[$T>T_\text{c}, B>0$的自由能曲线]{
        \begin{tikzpicture}
            
            % m横轴
            \draw[->] (-2.5,0) -- (2,0) node[right] {$m$};
            % 自由能纵轴
            \draw[->] (0,-0.5) -- (0,5.5) node[above] {$f(m)$};
            
            % 画出自由能
            \draw[samples=50, smooth, domain=-2.5:1.9] plot(\x,{1*\x+0.5*(\x*\x)+0.11*(\x*\x*\x*\x)});
    
            % 标出极小值点
            \node[circle,fill,inner sep=1.5pt] at (-0.82, -0.43) {};
    
        \end{tikzpicture}
        \label{fig:small-t-negative-b-free-energy}
    }
    \subfigure[$T<T_\text{c}, B>0$的自由能曲线]{
        \begin{tikzpicture}
            
            % m横轴
            \draw[->] (-2.5,0) -- (2.5,0) node[right] {$m$};
            % 自由能纵轴
            \draw[->] (0,-1.5) -- (0,4.5) node[above] {$f(m)$};
            
            % 画出自由能
            \draw[samples=50, smooth, domain=-2.2:2.2] plot(\x,{-0.25*\x-0.5*(\x*\x)+0.25*(\x*\x*\x*\x)});
    
            % 标出极小值点
            \node[circle,fill,inner sep=1.5pt] at (1.11, -0.51) {};
    
        \end{tikzpicture}
        \label{fig:big-t-positive-b-free-energy}
    }
    \subfigure[$T<T_\text{c}, B<0$的自由能曲线]{
        \begin{tikzpicture}
            
            % m横轴
            \draw[->] (-2.5,0) -- (2.5,0) node[right] {$m$};
            % 自由能纵轴
            \draw[->] (0,-1.5) -- (0,4.5) node[above] {$f(m)$};
            
            % 画出自由能
            \draw[samples=50, smooth, domain=-2.2:2.2] plot(\x,{0.25*\x-0.5*(\x*\x)+0.25*(\x*\x*\x*\x)});
    
            % 标出极小值点
            \node[circle,fill,inner sep=1.5pt] at (-1.11, -0.51) {};
    
        \end{tikzpicture}
        \label{fig:big-t-negative-b-free-energy}
    }
    \caption{磁场变化导致自由能曲线变化}
    \label{fig:mf-b-and-free-energy}
\end{figure}

这就意味着,我们可以得到如\autoref{fig:mf-phase-diagram}所示的相图:$T<T_\text{c}$时有明确区分的两相,$B=0$为两相共存曲线;而$T>T_\text{c}$时两相混合。
这和水的气液相变的相图非常相似。

\begin{figure}
    \centering
    \begin{tikzpicture}
            
        % T横轴
        \draw[->] (0,0) -- (5,0) node[right] {$T$};
        % B纵轴
        \draw[->] (0,-2) -- (0,2) node[above] {$B$};
        
        % 相变曲线
        \draw[thick] (0, 0) -- (2,0);

        % 标出二级相变点
        \node[dot, label=above:$T_\text{c}$] at (2, 0) {};

        % 标出两相
        \node[] at (1, 1) {$m>0$};
        \node[] at (1, -1) {$m<0$};
        \node[] at (4, 0.5) {disordered};

    \end{tikzpicture}
    \caption{平均场理论下的相图}
    \label{fig:mf-phase-diagram}
\end{figure}

\subsubsection{相变指数}

\subsection{平均场理论的可靠性}

随着维数增大,每个格点周围的格点变多,因此相互作用也就越接近平均场理论的预言。
在伊辛模型中,
\begin{itemize}
    \item $d=1$时平均场理论完全是错的,根本就没有相变;
    \item $d=2,3$时平均场理论给出了定性大致正确的结果,但是临界指数之类的细节都是错的;
    \item $d \geq 4$时平均场理论是对的。
\end{itemize}

\section{伊辛模型的有效场论}\label{sec:ising-effective-field}

\subsection{金斯堡-朗道理论}

\subsection{直接做平均}

\section{伊辛模型的重整化群}\label{sec:ising-rg}

\end{document}