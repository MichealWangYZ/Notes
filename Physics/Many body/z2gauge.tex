\documentclass[hyperref, UTF8, a4paper]{ctexart}

\usepackage{geometry}
\usepackage{titling}
\usepackage{titlesec}
\usepackage{paralist}
\usepackage{footnote}
\usepackage{enumerate}
\usepackage{amsmath, amssymb, amsthm}
\usepackage{bbm}
\usepackage{cite}
\usepackage{graphicx}
\usepackage{subfigure}
\usepackage{physics}
\usepackage{tikz}
\usepackage{autobreak}
\usepackage[ruled, vlined, linesnumbered, noend]{algorithm2e}
\usepackage[colorlinks, linkcolor=black, anchorcolor=black, citecolor=black]{hyperref}
\usepackage{prettyref}

% Page style
\geometry{left=3.18cm,right=3.18cm,top=2.54cm,bottom=2.54cm}
\titlespacing{\paragraph}{0pt}{1pt}{10pt}[20pt]
\setlength{\droptitle}{-5em}
\preauthor{\vspace{-10pt}\begin{center}}
\postauthor{\par\end{center}}

% Math operators
\DeclareMathOperator{\timeorder}{T}
\DeclareMathOperator{\diag}{diag}
\DeclareMathOperator{\legpoly}{P}
\DeclareMathOperator{\primevalue}{P}
\DeclareMathOperator{\sgn}{sgn}
\newcommand*{\ii}{\mathrm{i}}
\newcommand*{\ee}{\mathrm{e}}
\newcommand*{\const}{\mathrm{const}}
\newcommand*{\comment}{\paragraph{注记}}
\newcommand*{\suchthat}{\quad \text{s.t.} \quad}
\newcommand*{\argmin}{\arg\min}
\newcommand*{\argmax}{\arg\max}
\newcommand*{\normalorder}[1]{: #1 :}
\newcommand*{\pair}[1]{\langle #1 \rangle}
\newcommand*{\fd}[1]{\mathcal{D} #1}
\DeclareMathOperator{\bigO}{\mathcal{O}}

% prettyref setting
\newrefformat{sec}{第\ref{#1}节}
\newrefformat{note}{注\ref{#1}}
\newrefformat{fig}{图\ref{#1}}
\newrefformat{alg}{算法\ref{#1}}
\renewcommand{\autoref}{\prettyref}

% TikZ setting
\usetikzlibrary{arrows,shapes,positioning}
\usetikzlibrary{arrows.meta}
\usetikzlibrary{decorations.markings}
\tikzstyle arrowstyle=[scale=1]
\tikzstyle directed=[postaction={decorate,decoration={markings,
    mark=at position .5 with {\arrow[arrowstyle]{stealth}}}}]
\tikzstyle ray=[directed, thick]
\tikzstyle dot=[anchor=base,fill,circle,inner sep=1pt]

% Algorithm setting
\renewcommand{\algorithmcfname}{算法}
% Python-style code
\SetKwIF{If}{ElseIf}{Else}{if}{:}{elif:}{else:}{}
\SetKwFor{For}{for}{:}{}
\SetKwFor{While}{while}{:}{}
\SetKwInput{KwData}{输入}
\SetKwInput{KwResult}{输出}
\SetArgSty{textnormal}

\renewcommand{\emph}[1]{\textbf{#1}}
\newcommand*{\concept}[1]{\underline{\textbf{#1}}}
\newcommand*{\Ztwo}{$\mathbb{Z}_2$}

\title{\Ztwo规范场论}
\author{吴何友}

\begin{document}

\maketitle

\section{格点上的\Ztwo规范理论}

最广为人知的规范场论可能是电动力学,这是一个$U(1)$规范理论,其中电子场可以发生任意的局域相位转动,而与之配套的规范场——电磁场矢势——发生一个局域平移。
本文中我们不要$U(1)$这么大的对称性,而是只希望电子场或者不发生相位转动,或者相位就转动$\pi$,在这样的规范对称性——也就是\concept{\Ztwo规范对称性}下系统的动力学保持不变。
如果我们还是在通常的四维时空中工作那么局域\Ztwo变换就是不连续的:因为$0$和$\pi$不能连续过渡。
因此我们将在格点上工作,即研究格点规范场论。

格点上的电子的动能项无非是从一个点跃迁到另外一个点,即
\begin{equation}
    \hat{H}_0 = - \sum_{i, j, \alpha} t_{ij} \hat{c}_{i \alpha}^\dagger \hat{c}_{j \alpha}.
    \label{eq:hopping-hamiltonian}
\end{equation}
这个哈密顿量在局域\Ztwo变换下不是不变的。
为了加入局域\Ztwo对称性我们只能修改$t_{ij}$系数,使得它在\Ztwo变换下能够吸收掉电子场带来的变化。
容易看到,只需要指定
\[
    \hat{c}_{i \alpha} \longrightarrow \eta_{i} \hat{c}_{i \alpha}, \quad t_{ij} \longrightarrow \eta_i \eta_j t_{ij},
\]
就能够让哈密顿量具有局域\Ztwo对称性。由于$t_{ij}$只是在正负两种状态之间切换,可以引入一个规范联络$\sigma_{ij} = \pm 1$,于是用哈密顿量
\[
    \hat{H} = - \sum_{i, j, \alpha} t_{ij} \sigma_{ij} \hat{c}_{i \alpha}^\dagger \hat{c}_{j \alpha}
\]
做路径积分,分别以$\hat{c}, \hat{c}^\dagger$和$\sigma_{ij}$为积分变量即可得到一个\Ztwo规范理论。

现在我们回到正则量子化框架中,$\sigma_{ij}$在每一个格点引入了$\pm 1$两个状态,从而我们可以把它当成一个自旋$1/2$的自旋算符%
\footnote{实际上,这个“自旋算符”未必来自某个体系的内禀旋转不变性。
更加数学的说法是,由于每个格点都有两个状态,我们可以在每个格点引入一个$2\times 2$的厄米矩阵
\[
    \hat{\sigma} = \pmqty{1 & 0 \\ 0 & -1}
\]
作为规范场对应的算符,而这正是泡利矩阵中的$\hat{\sigma}^z$。后面引入$\hat{\sigma}^x$等算符的目的也只是用于翻转规范场的状态。
}%
,从而哈密顿量为
\begin{equation}
    \hat{H} = - \sum_{i, j, \alpha} t_{ij} \hat{\sigma}^z_{ij} \hat{c}_{i \alpha}^\dagger \hat{c}_{j \alpha}
    \label{eq:minimal-z2-couple}
\end{equation}
希尔伯特空间为电子的态空间直积上每一点的自旋$1/2$空间。\Ztwo规范变换为
\begin{equation}
    \hat{c}_{i \alpha} \longrightarrow \eta_{i} \hat{c}_{i \alpha}, \quad \hat{\sigma}_{ij}^z \longrightarrow \eta_i \eta_j \hat{\sigma}_{ij}^z.
\end{equation}

特别的,如果\eqref{eq:hopping-hamiltonian}实际上是一个紧束缚模型,\eqref{eq:minimal-z2-couple}就成为
\begin{equation}
    \hat{H} = - t \sum_{\pair{i, j}} \hat{\sigma}^z_{ij} \hat{c}_{i \alpha}^\dagger \hat{c}_{j \alpha} + \text{h.c.}.
    \label{eq:tight-binding-z2}
\end{equation}
此时$\sigma_{ij}^z$实际上仅仅定义在格子的边上。

\section{二维格子的情况}

\subsection{无物质场的情况}

现在我们讨论二维格子的情况。\eqref{eq:tight-binding-z2}中的电子由于相互作用是有能隙的,于是将\eqref{eq:tight-binding-z2}中的电子自由度积掉%
\footnote{我们在正则量子化框架中工作,因此积掉电子自由度实际上意味着原本是纯态的系统将成为混合态。
但是实际上这无关紧要,因为凝聚态理论向来分析有限温度情况,一开始系统就是处于混合态的。
我们只需要假装不知道电子存在,分析\Ztwo规范场的态空间,最后计算配分函数即可,并不需要真的处理混合态。}%
,得到仅仅关于\Ztwo规范场(而没有任何物质场)的一个低能有效理论。
严格做有关的计算是非常不现实的,但是无论如何,积掉电子自由度之后的哈密顿量本身肯定是\Ztwo规范不变的。我们首先先分析\Ztwo规范变换如何写成算符形式,然后分析积掉电子自由度之后的哈密顿量会是什么形式的。

电子自由度积掉之后规范变换就变成了
\begin{equation}
    \hat{\sigma}_{ij}^z \longrightarrow \eta_i \eta_j \hat{\sigma}_{ij}^z,
    \label{eq:pure-sigma-ztwo}
\end{equation}
也就是说对每一条边上的$\hat{\sigma}^z$本征态,规范变换或是不改变它,或是加一个负号。我们希望将\Ztwo规范变换写成算符的形式,为此注意到在自旋$1/2$中,算符$\hat{\sigma}^x$可以翻转$\hat{\sigma}^z$的本征态,且$\hat{\sigma}^x$是厄米算符,于是一条边上的规范场翻转就是
\[
    \hat{\sigma}_{ij}^z \longrightarrow \hat{\sigma}^x_{ij} \hat{\sigma}_{ij}^z \hat{\sigma}^x_{ij}.
\]
任何一个\Ztwo规范变换都可以拆解成一系列作用在格点上的规范变换相乘,而作用在格点$i$上的规范变换翻转和这个格点连接的四条边上的规范场,于是作用在格点$i$上的规范变换为
\begin{equation}
    \hat{Q}_i = \prod_{\pair{i, j}} \hat{\sigma}^x_{ij} = \prod_{j \in +_i} \hat{\sigma}^x_{ij},
\end{equation}
于是规范不变量就是和所有$\hat{Q}_i$对易的算符。由于是低能有效理论,我们考虑最低阶的两个\Ztwo规范不变量,得到
\begin{equation}
    \hat{H} = - K \sum_{\pair{i, j}} \hat{\sigma}^x_{ij} - J \sum_{\Box} \prod_{l \in \Box} \hat{\sigma}^z_{l}.
    \label{eq:z2-2d-hamiltonian}
\end{equation}
这就是只含有\Ztwo规范场的\emph{一个}有效理论(当然,实际上还有很多其它的\Ztwo规范理论,是取其它\Ztwo规范不变量得到的)。格点上的\Ztwo规范变换实际上是离散变换,因此相应的规范荷就是$\hat{Q}_i$。
关于为什么我们考虑了最低阶的两个\Ztwo规范不变量而不是别的(特别是,$\hat{\sigma}^x$是怎么被牵扯进来的),可以从以下角度考虑:
在零温情况下我们此处给出的\Ztwo理论对应一个三维经典统计理论,这个三维经典统计理论当然也应该具有\Ztwo规范不变性。
我们知道这个三维经典统计理论的自由度实际上就是将\eqref{eq:z2-2d-hamiltonian}的自由度加上一个虚时间指标之后得到的结果,即$\{\sigma^z_{ij}(\tau)\}$。
由于没有$z$方向上的边上定义了$\sigma^z$自由度,我们可以直接沿用\eqref{eq:pure-sigma-ztwo}作为三维经典统计理论的\Ztwo规范变换。
三维经典统计理论的形式应该是
\[
    Z = \sum_{\sigma^z} \exp(\sum_{\tau} (J_{xy} \sum_{\Box} \prod_{l \in \Box} \sigma_l^z(\tau)  ) )
\]
% TODO

无论是\eqref{eq:tight-binding-z2}还是\eqref{eq:z2-2d-hamiltonian}都具有\Ztwo规范不变性,如果我们认为规范自由度不具有物理含义(它实际上有没有物理含义取决于我们关心的物理量是不是只涉及规范不变量),那么这两个哈密顿量就含有额外的自由度。
我们要设法把规范等价的构型全部映射到同一个构型上,而把规范不等价的构型映射到不同的构型上。
为此,我们将每个格子赋予一个格点坐标$I$,从而诸$\{i\}$和诸$\{I\}$形成对偶格点坐标。
设$\Box_I$为$I$号格子,我们定义
\begin{equation}
    \hat{\tau}^x_I = \prod_{l \in \Box_I} \hat{\sigma}^z_l,
    \label{eq:def-tau}
\end{equation}
上标$x$看起来很奇怪,不过我们很快会发现其作用。这样\eqref{eq:z2-2d-hamiltonian}中的第二项就可以很容易地写出了。
至于第一项,如果将$\hat{\sigma}_{ij}^x$作用在某个$\hat{\sigma}^z$表象下的态上面,那么边$ij$上的$\sigma^z$反号,其余什么都不变,这就是说,设边$ij$由方格$I$和$J$共享,则由定义\eqref{eq:def-tau},$I$和$J$对应的$\tau^x$也反号,其余不变;
另一方面,将$\hat{\tau}^x$看成某个表象下的$x$方向泡利矩阵,并将$\hat{\tau}^z_I \hat{\tau}^z_J$作用在一个态上,则$I$和$J$对应的$\tau^x$均反号(同样依据泡利矩阵的性质,即$z$方向泡利矩阵可以翻转$x$方向泡利矩阵的本征态)。
两个算符的作用效果完全一样,所以实际上
\[
    \hat{\tau}^z_I \hat{\tau}^z_J = \hat{\tau}^x_{ij},
\]
从而我们得到
\begin{equation}
    \hat{H} = - K \sum_{\pair{I, J}} \hat{\tau}^z_I \hat{\tau}^z_J - J \sum_{I} \hat{\tau}^x_I.
    \label{eq:z2-2d-tau-hamiltonian}
\end{equation}

现在没有规范冗余了——$\hat{\tau}^x_{I}$和$\hat{\tau}^z_I$都是规范不变量。
要看出自由度减少了多少,注意到二维格子中一个格子有四条边,每条边由两个格子分享,因此如果有$N$个格子(从而有$N$个格点),那么有$2N$条边。另一方面,只有$N$个方格。
因此如果只以$\hat{\tau}^z_I$为动力学自由度,则我们将希尔伯特空间的维数从$2^{2N}$降到了$2^N$。
丢自由度是正常的,因为在以上过程中我们抛弃了规范自由度,但是需要验证只以$\hat{\tau}^z_I$为动力学自由度是不是把一些并非规范自由度的自由度(它们没有出现在哈密顿量中)也抛弃了。
换句话说,我们需要验证,规范不等价的态是否给出不同的$\hat{\tau}^z_I$取值。
% TODO

\eqref{eq:z2-2d-tau-hamiltonian}正是\concept{横场伊辛模型},它是一个二维量子模型,其零温配分函数的精确形式对应一个三维经典统计模型,实际上这个三维经典统计模型就是一个各向异性的伊辛模型(在虚时间上的最近邻相互作用和空间方向上的最近邻相互作用不同)。
我们知道三维伊辛模型一定会出现相变,有一个顺磁相和一个铁磁相,这来自其普适类%
\footnote{
    虽然\eqref{eq:z2-2d-tau-hamiltonian}对应的经典统计模型是各向异性的,这并不改变其普适类,因为总是可以适当调节$\beta$的尺度让该经典统计模型变成各向同性的。
}%
,因此结论是,零温下横场伊辛模型——从而\Ztwo规范场——也会有一个相变,随着参数$K / J$的变化,从一个相切换到另一个相。
现在的问题是,\Ztwo规范场在零温下的两个相都是什么?
三维伊辛模型的顺磁相对应无束缚的\Ztwo模型的状态,而铁磁相对应束缚的\Ztwo模型的状态。

\subsection{引入物质场}



\subsection{规范荷和“磁通量”}



\end{document}