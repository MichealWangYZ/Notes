\documentclass[hyperref, UTF8, a4paper]{ctexart}

\usepackage{geometry}
\usepackage{titling}
\usepackage{titlesec}
\usepackage{paralist}
\usepackage{footnote}
\usepackage{enumerate}
\usepackage{amsmath, amssymb, amsthm}
\usepackage{bbm}
\usepackage{cite}
\usepackage{graphicx}
\usepackage{subfigure}
\usepackage{physics}
\usepackage{tikz}
\usepackage{autobreak}
\usepackage[ruled, vlined, linesnumbered, noend]{algorithm2e}
\usepackage[colorlinks, linkcolor=black, anchorcolor=black, citecolor=black]{hyperref}
\usepackage{prettyref}

% Page style
\geometry{left=3.18cm,right=3.18cm,top=2.54cm,bottom=2.54cm}
\titlespacing{\paragraph}{0pt}{1pt}{10pt}[20pt]
\setlength{\droptitle}{-5em}
\preauthor{\vspace{-10pt}\begin{center}}
\postauthor{\par\end{center}}

% Math operators
\DeclareMathOperator{\timeorder}{T}
\DeclareMathOperator{\diag}{diag}
\DeclareMathOperator{\legpoly}{P}
\DeclareMathOperator{\primevalue}{P}
\DeclareMathOperator{\sgn}{sgn}
\newcommand*{\ii}{\mathrm{i}}
\newcommand*{\ee}{\mathrm{e}}
\newcommand*{\const}{\mathrm{const}}
\newcommand*{\comment}{\paragraph{注记}}
\newcommand*{\suchthat}{\quad \text{s.t.} \quad}
\newcommand*{\argmin}{\arg\min}
\newcommand*{\argmax}{\arg\max}
\newcommand*{\normalorder}[1]{: #1 :}
\newcommand*{\pair}[1]{\langle #1 \rangle}
\newcommand*{\fd}[1]{\mathcal{D} #1}
\DeclareMathOperator{\bigO}{\mathcal{O}}

% prettyref setting
\newrefformat{sec}{第\ref{#1}节}
\newrefformat{note}{注\ref{#1}}
\newrefformat{fig}{图\ref{#1}}
\newrefformat{alg}{算法\ref{#1}}
\renewcommand{\autoref}{\prettyref}

% TikZ setting
\usetikzlibrary{arrows,shapes,positioning}
\usetikzlibrary{arrows.meta}
\usetikzlibrary{decorations.markings}
\tikzstyle arrowstyle=[scale=1]
\tikzstyle directed=[postaction={decorate,decoration={markings,
    mark=at position .5 with {\arrow[arrowstyle]{stealth}}}}]
\tikzstyle ray=[directed, thick]
\tikzstyle dot=[anchor=base,fill,circle,inner sep=1pt]

% Algorithm setting
\renewcommand{\algorithmcfname}{算法}
% Python-style code
\SetKwIF{If}{ElseIf}{Else}{if}{:}{elif:}{else:}{}
\SetKwFor{For}{for}{:}{}
\SetKwFor{While}{while}{:}{}
\SetKwInput{KwData}{输入}
\SetKwInput{KwResult}{输出}
\SetArgSty{textnormal}

\renewcommand{\emph}[1]{\textbf{#1}}
\newcommand*{\concept}[1]{\underline{\textbf{#1}}}
\newcommand*{\Ztwo}{$\mathbb{Z}_2$}

\title{\Ztwo规范场论}
\author{吴何友}

\begin{document}

\maketitle

\section{格点上的\Ztwo规范理论}\label{sec:z2-on-lattice}

最广为人知的规范场论可能是电动力学,这是一个$U(1)$规范理论,其中电子场可以发生任意的局域相位转动,而与之配套的规范场——电磁场矢势——发生一个局域平移。
本文中我们不要$U(1)$这么大的对称性,而是只希望电子场或者不发生相位转动,或者相位就转动$\pi$,在这样的规范对称性——也就是\concept{\Ztwo规范对称性}下系统的动力学保持不变。
如果我们还是在通常的四维时空中工作那么局域\Ztwo变换就是不连续的:因为$0$和$\pi$不能连续过渡。
因此我们将在格点上工作,即研究格点规范场论。

格点上的电子的动能项无非是从一个点跃迁到另外一个点,即
\begin{equation}
    \hat{H}_0 = - \sum_{i, j, \alpha} t_{ij} \hat{c}_{i \alpha}^\dagger \hat{c}_{j \alpha}.
    \label{eq:hopping-hamiltonian}
\end{equation}
这个哈密顿量在局域\Ztwo变换下不是不变的。
现在假定出于某些原因,系统具有了\Ztwo规范对称性,那么为了加入局域\Ztwo对称性我们只能修改$t_{ij}$系数,使得它在\Ztwo变换下能够吸收掉电子场带来的变化。
容易看到,只需要指定
\[
    \hat{c}_{i \alpha} \longrightarrow \eta_{i} \hat{c}_{i \alpha}, \quad t_{ij} \longrightarrow \eta_i \eta_j t_{ij},
\]
就能够让哈密顿量具有局域\Ztwo对称性。由于$t_{ij}$只是在正负两种状态之间切换,可以引入一个规范联络$\sigma_{ij} = \pm 1$,于是用哈密顿量
\[
    \hat{H} = - \sum_{i, j, \alpha} t_{ij} \sigma_{ij} \hat{c}_{i \alpha}^\dagger \hat{c}_{j \alpha}
\]
做路径积分,分别以$\hat{c}, \hat{c}^\dagger$和$\sigma_{ij}$为积分变量即可得到一个\Ztwo规范理论。

现在我们回到正则量子化框架中,$\sigma_{ij}$在每一个格点引入了$\pm 1$两个状态,从而我们可以把它当成一个自旋$1/2$的自旋算符%
\footnote{实际上,这个“自旋算符”未必来自某个体系的内禀旋转不变性。
更加数学的说法是,由于每个格点都有两个状态,我们可以在每个格点引入一个$2\times 2$的厄米矩阵
\[
    \hat{\sigma} = \pmqty{1 & 0 \\ 0 & -1}
\]
作为规范场对应的算符,而这正是泡利矩阵中的$\hat{\sigma}^z$。后面引入$\hat{\sigma}^x$等算符的目的也只是用于翻转规范场的状态。
}%
,从而哈密顿量为
\begin{equation}
    \hat{H} = - \sum_{i, j, \alpha} t_{ij} \hat{\sigma}^z_{ij} \hat{c}_{i \alpha}^\dagger \hat{c}_{j \alpha}
    \label{eq:minimal-z2-couple}
\end{equation}
希尔伯特空间为电子的态空间直积上每一点的自旋$1/2$空间。\Ztwo规范变换为
\begin{equation}
    \hat{c}_{i \alpha} \longrightarrow \eta_{i} \hat{c}_{i \alpha}, \quad \hat{\sigma}_{ij}^z \longrightarrow \eta_i \eta_j \hat{\sigma}_{ij}^z.
\end{equation}

特别的,如果\eqref{eq:hopping-hamiltonian}实际上是一个紧束缚模型,\eqref{eq:minimal-z2-couple}就成为
\begin{equation}
    \hat{H} = - t \sum_{\pair{i, j}} \hat{\sigma}^z_{ij} \hat{c}_{i \alpha}^\dagger \hat{c}_{j \alpha} + \text{h.c.}.
    \label{eq:tight-binding-z2}
\end{equation}
此时$\sigma_{ij}^z$实际上仅仅定义在格子的边上,即不需要对不相邻的点对$\pair{i, j}$也对应对应的$\sigma_{ij}^z$。

\section{二维格子上的无物质场\Ztwo规范场理论}

\subsection{积掉电子}

现在我们讨论二维格子的情况。\eqref{eq:tight-binding-z2}中的电子由于相互作用是有能隙的,于是将\eqref{eq:tight-binding-z2}中的电子自由度积掉%
\footnote{我们在正则量子化框架中工作,因此积掉电子自由度实际上意味着原本是纯态的系统将成为混合态。
但是实际上这无关紧要,因为凝聚态理论向来分析有限温度情况,一开始系统就是处于混合态的。
我们只需要假装不知道电子存在,分析\Ztwo规范场的态空间,最后计算配分函数即可,并不需要真的处理混合态。}%
,得到仅仅关于\Ztwo规范场(而没有任何物质场)的一个低能有效理论。
严格做有关的计算是非常不现实的,但是无论如何,积掉电子自由度之后的哈密顿量本身肯定是\Ztwo规范不变的。我们首先先分析\Ztwo规范变换如何写成算符形式,然后分析积掉电子自由度之后的哈密顿量会是什么形式的。

电子自由度积掉之后规范变换就变成了
\begin{equation}
    \hat{\sigma}_{ij}^z \longrightarrow \eta_i \eta_j \hat{\sigma}_{ij}^z,
    \label{eq:pure-sigma-ztwo}
\end{equation}
也就是说对每一条边上的$\hat{\sigma}^z$本征态,规范变换或是不改变它,或是加一个负号。我们希望将\Ztwo规范变换写成算符的形式,为此注意到在自旋$1/2$中,算符$\hat{\sigma}^x$可以翻转$\hat{\sigma}^z$的本征态,且$\hat{\sigma}^x$是厄米算符,于是一条边上的规范场翻转就是
\[
    \hat{\sigma}_{ij}^z \longrightarrow \hat{\sigma}^x_{ij} \hat{\sigma}_{ij}^z \hat{\sigma}^x_{ij}.
\]
任何一个\Ztwo规范变换都可以拆解成一系列作用在格点上的规范变换相乘,而作用在格点$i$上的规范变换翻转和这个格点连接的四条边上的规范场,于是作用在格点$i$上的规范变换为
\begin{equation}
    \hat{Q}_i = \prod_{\pair{i, j}} \hat{\sigma}^x_{ij} = \prod_{j \in +_i} \hat{\sigma}^x_{ij},
    \label{eq:z2-charge}
\end{equation}
于是规范不变量就是和所有$\hat{Q}_i$对易的算符。由于是低能有效理论,我们考虑最低阶的两个\Ztwo规范不变量,得到
\begin{equation}
    \hat{H} = - K \sum_{\pair{i, j}} \hat{\sigma}^x_{ij} - J \sum_{\Box} \prod_{l \in \Box} \hat{\sigma}^z_{l}.
    \label{eq:z2-2d-hamiltonian}
\end{equation}
这就是只含有\Ztwo规范场的\emph{一个}有效理论(当然,实际上还有很多其它的\Ztwo规范理论,是取其它\Ztwo规范不变量得到的)。
关于为什么我们考虑了最低阶的两个\Ztwo规范不变量而不是别的(特别是,$\hat{\sigma}^x$是怎么被牵扯进来的),可以从以下角度考虑:
在零温情况下我们此处给出的\Ztwo理论对应一个三维经典统计理论,这个三维经典统计理论当然也应该具有\Ztwo规范不变性。
我们知道这个三维经典统计理论的自由度实际上就是将\eqref{eq:z2-2d-hamiltonian}的自由度加上一个虚时间指标之后得到的结果,即$\{\sigma^z_{ij}(\tau)\}$。
由于没有$z$方向上的边上定义了$\sigma^z$自由度,我们可以直接沿用\eqref{eq:pure-sigma-ztwo}作为三维经典统计理论的\Ztwo规范变换。
三维经典统计理论的形式应该是
\[
    Z = \sum_{\sigma^z} \exp(\sum_{\tau} (J_{xy} \sum_{\Box} \prod_{l \in \Box} \sigma_l^z(\tau)  ) )
\]
% TODO

\subsection{去除规范自由度}

无论是\eqref{eq:tight-binding-z2}还是\eqref{eq:z2-2d-hamiltonian}都具有\Ztwo规范不变性,如果我们认为规范自由度不具有物理含义(它实际上有没有物理含义取决于我们关心的物理量是不是只涉及规范不变量),那么这两个哈密顿量就含有额外的自由度。
我们要设法把规范等价的构型全部映射到同一个构型上,而把规范不等价的构型映射到不同的构型上。
为此,我们将每个格子赋予一个格点坐标$I$,从而诸$\{i\}$和诸$\{I\}$形成对偶格点坐标。
设$\Box_I$为$I$号格子($I$标记了以所有的格子的中心为格点形成的新格子的格点坐标,称为\concept{对偶格子}),我们定义
\begin{equation}
    \hat{\tau}^x_I = \prod_{l \in \Box_I} \hat{\sigma}^z_l,
    \label{eq:def-tau}
\end{equation}
上标$x$看起来很奇怪,不过我们很快会发现其作用。这样\eqref{eq:z2-2d-hamiltonian}中的第二项就可以很容易地写出了。
至于第一项,如果将$\hat{\sigma}_{ij}^x$作用在某个$\hat{\sigma}^z$表象下的态上面,那么边$ij$上的$\sigma^z$反号,其余什么都不变,这就是说,设边$ij$由方格$I$和$J$共享,则由定义\eqref{eq:def-tau},$I$和$J$对应的$\tau^x$也反号,其余不变;
另一方面,将$\hat{\tau}^x$看成某个表象下的$x$方向泡利矩阵,并将$\hat{\tau}^z_I \hat{\tau}^z_J$作用在一个态上,则$I$和$J$对应的$\tau^x$均反号(同样依据泡利矩阵的性质,即$z$方向泡利矩阵可以翻转$x$方向泡利矩阵的本征态)。
两个算符的作用效果完全一样,所以实际上
\[
    \hat{\tau}^z_I \hat{\tau}^z_J = \hat{\sigma}^x_{ij},
\]
从而我们得到
\begin{equation}
    \hat{H} = - K \sum_{\pair{I, J}} \hat{\tau}^z_I \hat{\tau}^z_J - J \sum_{I} \hat{\tau}^x_I.
    \label{eq:z2-2d-tau-hamiltonian}
\end{equation}

现在没有规范冗余了——$\hat{\tau}^x_{I}$和$\hat{\tau}^z_I$都是规范不变量。
要看出自由度减少了多少,注意到二维格子中一个格子有四条边,每条边由两个格子分享,因此如果有$N$个格子(从而有$N$个格点),那么有$2N$条边。另一方面,只有$N$个方格。
因此如果只以$\hat{\tau}^z_I$为动力学自由度,则我们将希尔伯特空间的维数从$2^{2N}$降到了$2^N$。
丢自由度是正常的,因为在以上过程中我们抛弃了规范自由度,但是需要验证只以$\hat{\tau}^z_I$为动力学自由度是不是把一些并非规范自由度的自由度(它们没有出现在哈密顿量中)也抛弃了。
换句话说,我们需要验证,规范不等价的态是否给出不同的$\hat{\tau}^z_I$取值。
% TODO

\eqref{eq:z2-2d-tau-hamiltonian}正是\concept{横场伊辛模型},它是一个二维量子模型,其零温配分函数的精确形式对应一个三维经典统计模型,实际上这个三维经典统计模型就是一个各向异性的伊辛模型(在虚时间上的最近邻相互作用和空间方向上的最近邻相互作用不同)。
我们知道三维伊辛模型一定会出现相变,有一个顺磁相和一个铁磁相,这来自其普适类%
\footnote{
    虽然\eqref{eq:z2-2d-tau-hamiltonian}对应的经典统计模型是各向异性的,这并不改变其普适类,因为总是可以适当调节$\beta$的尺度让该经典统计模型变成各向同性的。
}%
,因此结论是,零温下横场伊辛模型——从而\Ztwo规范场——也会有一个相变,随着参数$K / J$的变化,从一个相切换到另一个相。

\subsection{规范荷和磁通量}

在$U(1)$规范场论中,设通过一个方格的磁通量为$\Phi$,则
\[
    \ee^{\ii \Phi} = \prod_{l \in \Box} t_{ij},
\]
于是在本文涉及的\Ztwo规范场中可以如法炮制地定义
\begin{equation}
    \ee^{\ii \Phi_I} = \prod_{l \in \Box_I} \hat{\sigma}^z_{ij} = \hat{\tau}^x_I,
\end{equation}
也即,我们用电子在格子上转一圈发生的相位改变来定义磁通量。与$U(1)$的情况不同,\Ztwo规范场中磁通量只有$0$和$\pi$两种,因为四个$\sigma^z$相乘要么是$1$要么是$-1$。
现在我们看到了$\hat{\tau}^x$的另一重意义:它标记了一个格子上的磁通量。
电磁场中的磁通量如果量子化的话需要一定条件,但是\Ztwo规范场中的磁通量就是量子化的,而且只有两个状态。
注意到$\tau_I^x$取$1$时能量较低而取$-1$时能量较高,我们可以将某个格子的磁通量取$\pi$当成一种激发态,称为\concept{m激发},以体现它和磁通量的相似之处。

另一个可以模仿电磁场引入的概念是\Ztwo规范荷,我们已经看到,\Ztwo规范变换对应的规范荷为\eqref{eq:z2-charge},这个量的取值只有$\pm 1$(因为是四个$\sigma^x$的乘积)。
由于\eqref{eq:z2-charge}守恒,我们有如下\Ztwo规范场的高斯定律:
\begin{equation}
    \prod_{j \in +} \sigma^x_{ij} = \text{\Ztwo -charge at $i$} = \const,
    \label{eq:gauss-z2}
\end{equation}
这个常数可以取$1$也可以取$-1$,但是不能一会是$1$,一会是$-1$。
这个额外的条件将希尔伯特空间划分成没有重叠的很多支,不同分支的\Ztwo规范荷分布不同。
虽然规范荷通常是通过规范场和物质场的耦合项引入的,在积掉物质场之后还是可以构造出规范荷的表达式,正如在电磁场中,即使我们积掉了物质场,麦克斯韦方程中还是会有一个电荷守恒方程
\[
    \pdv{\rho}{t} + \div{\vb*{j}} = 0
\]
一样——即使我们不知道电磁场实际上和一个物质场发生了耦合,我们还是可以将电荷当成电场线的某种特殊分布(源和汇),而以它们为某种激发。%
\footnote{
    \eqref{eq:gauss-z2}和麦克斯韦方程导出的电荷守恒方程有一个重要的区别,就是前者要求规范荷在每一点都守恒,而后者允许规范荷的流动。但这实际上并没有什么物理意义。
    麦克斯韦方程本身并不规定电荷应该如何流动(这是本构关系应该做的事情),因此,每一点的电荷密度算符和纯电磁场的哈密顿量也是对易的。
    在本节中尚未引入任何真的携带\Ztwo规范荷的场,如果引入了,本节中的$\hat{H}$就只是\Ztwo规范场的哈密顿量而不是完整的哈密顿量了,此时\eqref{eq:gauss-z2}的第一个等号当然仍然成立,但是每个点上的\Ztwo规范荷就未必总是不变的了,虽然规范对称性要求规范荷总量保持不变。 % TODO:总量是乘起来的还是加起来的??
    我们将这种没有动力学的规范荷称为\concept{测试规范荷},这个名称的意味是显然的;它相比有动力学的规范荷更容易处理,后者的哈密顿量在坐标表象下基本上不会是对角化的(例如\prettyref{sec:z2-on-lattice}中那样)。
}%
同理,在\Ztwo规范场中,$Q_i$取$-1$意味着更高的能量(计算一下能量期望值就知道),那么我们可以认为某个点$i$处$Q_i=-1$意味着这里出现了某个激发,从而一个\Ztwo规范荷被放置在了这里,无论其背后的机制是什么,无论是不是真的有一个物质场和\Ztwo规范场发生了耦合。%
\footnote{
    我们在\prettyref{sec:z2-on-lattice}中是通过对电子的紧束缚模型引入局域\Ztwo规范对称性而得到一个\Ztwo规范理论的,但不难看出,我们这里引入的\Ztwo规范荷自由度肯定不是坐标表象下的电子激发,因为坐标表象下的电子激发的哈密顿量不是对角化的,或者直观地说电子会“四处乱跑”,而这里的\Ztwo规范荷只是测试规范荷。
}%
我们称这种激发为\concept{e激发},以体现它和电荷的相似性。

\subsection{自由相和禁闭相}

现在的问题是,\Ztwo规范场在零温下的两个相都是什么?三维伊辛模型具有一个顺磁相和一个铁磁相,由于铁磁序的形成需要更多相互作用,$J/K$超过某个点时会出现顺磁相,否则出现铁磁相。
映射回二维横场伊辛模型,顺磁相意味着对偶格子上的各个$\tau$基本指向$x$方向,即有确定的磁通量;铁磁相意味着对偶格子上的各个$\tau$基本指向$z$方向,且要么几乎都为$1$要么几乎都为$-1$,没有确定的磁通量。
当$J$相对$K$很大时,系统处于顺磁相,此时的基态几乎就是每个$\sigma^x$都取$1$的纯态,此时直觉上看,电子可以畅行无阻;而当$J$相对$K$很小时,系统处于铁磁相,此时的基态不是$\sigma^x$的本征态,投影在$\sigma^x$上有正有负,那么电子似乎会被“迷惑”住,不知道怎么走。
那么,我们可以合乎情理地猜测,三维伊辛模型的顺磁相对应着\Ztwo规范场模型的一个解禁闭相,而三维伊辛模型的铁磁相对应着\Ztwo规范场模型的一个禁闭相。

这样的论证当然是不够的,所以下面做一些半定量的论证。考虑\eqref{eq:z2-2d-hamiltonian}的格点路径积分(即虚时间轴也是离散化的),也即它对应的三维模型,定义
\begin{equation}
    W(C) = \expval{\prod_C \sigma^z_l(\tau)},
\end{equation}
其中期望值内部的算符乘积称为\concept{Wilson环算符}(如果它不闭合,那么就是\concept{弦算符}),$C$是一个在虚时间方向上有延展的环路。
容易看出$W(C)$给出了从某个虚时间点开始产生一对e激发,按照$C$指示的轨迹扩散,经过一段时间之后又湮灭的概率(在经典理论中这是良定义的,因为没有任何不确定关系;具体有没有手段可以用量子测量的标准方法测出这个概率则是另外一回事),随着$C$扩大它理所当然会衰减,如果随着$C$扩大它衰减得很快那么就说明e激发总是成对地被束缚在一起。
我们本来可以使用两点格林函数来表征e激发被束缚的强度的,但是由于两点格林函数不是规范不变的,任何两点格林函数都是零。
$W(C)$的衰减方式有两种典型的形式:一种是\concept{面积定律},即
\begin{equation}
    W(C) \sim \ee^{-A},
\end{equation}
其中$A$是$C$围成的面积,另一种是\concept{周长定律},即
\begin{equation}
    W(C) \sim \ee^{-L}.
\end{equation}
设$C$持续时间为$\tau$,则按照虚时间演化,应该有
\[
    W(C) \sim \ee^{-\tau \Delta E},
\]
其中$\Delta E$是两个e激发同时出现造成的能量上升。我们马上可以看出,由于
\[
    A \sim \tau R,
\]
其中$R$是两个e激发的距离,如果面积定律成立,必然有
\[
    \Delta E \sim R,
\]
这是典型的禁闭效应:两个e激发离得越远,能量越高,当能量高到一定程度时涨落会导致新的e激发产生,从而产生两对离得非常近的e激发。
另一方面,如果周长定律成立,则
\[
    \Delta E \sim 1 + \frac{R}{\tau},
\]
而由于粒子通常不会跑太远,可以认为$\Delta E$基本上是一个常数,因此没有禁闭效应。

那么,前述横场伊辛模型给出的两个相是不是分别对应面积定律和周长定律?
我们尝试将$W(C)$映射到横场伊辛模型中,因为$C$围成的区域内部的$\sigma^z$会被乘两次,于是就给出$1$,于是
\[
    \prod_{l \in C} \sigma^z_l = \prod_{l \in D} \sigma^z_l,
\]
依照$\tau^x$的定义即得到
\begin{equation}
    W(C) = \expval{\prod_{I \in D} \tau^x_I(\tau)},
\end{equation}
其中$D$是$C$围成的区域。
由于只有两个相,可以在$J/K$很小或很大时分别做微扰论,由此得到的关于相的结构的信息在整个相内部都是成立的。
$J \gg K$的情况对应顺磁相,系统基态形如
\[
    \ket*{\text{ground}} = \ket*{\tau^x = \rightarrow \rightarrow \cdots \rightarrow} + \frac{K}{J} \sum_{i, j} \ket*{\tau^x = \rightarrow \cdots \leftarrow_i \cdots \leftarrow_j \cdots \rightarrow} + \cdots.
\]
上式看起来很奇怪,不过真的去算期望会发现翻转两个$\tau^x$比翻转一个能量更低。(翻转两个的话,$\hat{\tau}^z_I \hat{\tau}^z_J$项的期望不为零)
如果格点$i$和$j$在$C$内部,那么它不会对期望值有什么贡献,因为$-1$平方还是$1$;格点$i$和$j$在$C$外部那肯定也不会有什么贡献。
唯一会让期望值偏离$1$的激发是$i, j$中一个在$C$内部一个在$C$外部,于是我们有
\[
    \expval{\prod_{I \in D} \hat{\tau}^x_I} \sim 1 - \frac{K}{J} L(C).
\]
那么,合理的猜测是,在顺磁相中应该有周长定律成立,于是没有禁闭。
铁磁相的讨论是类似的。 % TODO
我们仅仅讨论了两个极限情况,没有得到完整的$\ee$指数,但是一般的情况计算起来是非常困难的,此处略去。

总之,铁磁相对应禁闭的\Ztwo模型的状态,其中e激发被禁闭;顺磁相则对应非禁闭的\Ztwo模型的状态。禁闭相中,e激发不再是有意义的低能自由度。
由于禁闭相的基态非常接近$\hat{\tau}^z$的本征态,在其上讨论$\hat{\tau}^x$的排列模式——也就是m激发——并没有意义。
因此禁闭相中的“\Ztwo激发”——e激发和m激发——都没有意义,即在这里\Ztwo规范场的行为并不十分有趣。
当$J=0$时不可能出现很大的$J/K$,此时的二维\Ztwo规范场模型(称为\concept{Weigner模型})始终是禁闭的。

\subsection{拓扑激发}

有一点需要注意:虽然\eqref{eq:z2-2d-tau-hamiltonian}和\eqref{eq:z2-2d-hamiltonian}是对偶的,但在对称性、局域性等方面两者还是不一样的。例如,\eqref{eq:z2-2d-hamiltonian}有\Ztwo规范对称性,但\eqref{eq:z2-2d-tau-hamiltonian}只有全局\Ztwo对称性。
同样,在\eqref{eq:z2-2d-hamiltonian}中显而易见的局域约束,在\eqref{eq:z2-2d-tau-hamiltonian}中也会以一种非常不平凡的形式呈现出来。

在\eqref{eq:z2-2d-hamiltonian}中只有$\hat{\sigma}^x$和$\hat{\sigma}^z$,如果我们假定规范场和外界的耦合项仅仅显含这两个算符,那么m激发就是成对出现的:容易证明
\[
    \hat{\sigma}^z_i = \hat{\tau}^x_I \hat{\tau}^x_J,
\]
其中$i$是格子$I$和$J$共享的边,于是规范场和外界的耦合项只能够成对地产生和消灭m激发。m激发本身并没有禁闭(只需要在同一个格子的两条边上各作用一个$\hat{\sigma}^z$就能够创建一对不相连的m激发),但全局地看一定是成对的,从而全局地看,总磁通量一定是$0$。
只要\Ztwo规范场和外界通过$\hat{\sigma}^z$和$\hat{\sigma}^x$耦合,这就是一定成立的,也就是说这个性质——总磁通量为$0$,m激发成对产生——在任何局部的扰动下都是能够保持的。
需要注意的是如果我们将格点真的取为无限大,这个约束并没有意义,因此此时无法良定义一个总磁通量,或者直观地说,我们总是可以将成对的m激发的其中一个移动到无穷远,从而我们关心的区域中只有奇数个m激发。
在\eqref{eq:z2-2d-tau-hamiltonian}中m激发的个数是偶数(即$\tau^x$为$-1$的个数是偶数)完全不是显然的,因为“系统与外界的耦合通过$\hat{\sigma}^z$进行”在\eqref{eq:z2-2d-tau-hamiltonian}中无法简单地表示出来。

总之,磁通量为$\pi$的m激发对应一个横场伊辛模型中的磁子(magnon)激发,由于它在\Ztwo规范场模型中的对应物是定义在顶点上的,也可以称它为\concept{vison激发}(Vortex-Ising-son,三个词根分别表示此类激发定义在格点上、和伊辛场有关、是元激发)。
横场伊辛模型中,$\tau^z \tau^z$项可以视为m激发的跃迁项,因为容易验证它可以让一个m激发跃迁到临近的格点上。
m激发是有能隙的

另外通过适当选取规范还可以将无$J$的\Ztwo规范场弄成一系列一维伊辛模型的组合。
此时的禁闭相对应于顺磁相,因为伊辛模型中的$\hat{S}^z$就是磁通量。

\section{Toric-code模型}

\subsection{Toric-code哈密顿量与解析解}

Kitaev最早提出了一种模型,作为一种可能的量子计算纠错编码,他发现这个模型放在一个环面上可以有非常有趣的结果。
然而,事后发现这个模型实际上展现出了一个拓扑序。

考虑一个正方格点,在每条边(不是每个格点!)上放有一个自旋$1/2$的自旋自由度。
\concept{Toric-code模型}指的是如下哈密顿量:
\begin{equation}
    \hat{H} = - K \sum_s \hat{A}_s - J \sum_p \hat{B}_p,
    \label{eq:toric-code-hamiltonian}
\end{equation}
其中下标$s$表示格点(本节使用的记号和上一节有所不同),$\hat{A}_s$指的是格点$s$周围的四条边上的$x$方向上的自旋算符的乘积,即
\begin{equation}
    \hat{A}_s = \prod_{i\text{ near } s} \hat{\sigma}_i^x,
\end{equation}
而$p$表示格点中的一个最小正方形,$\hat{B}_p$指的是正方形$p$的四条边上的$z$方向上的自旋算符的乘积,即
\begin{equation}
    \hat{B}_p = \prod_\text{$i$ of $p$} \hat{\sigma}_i^z.
\end{equation}
很容易验证\eqref{eq:toric-code-hamiltonian}实际上也具有\Ztwo规范对称性。

\eqref{eq:toric-code-hamiltonian}是严格可解的。
首先可以验证$\{\hat{A}_s\}$和$\{\hat{B}_p\}$构成一组对易稳定子(即平方为1的一组彼此对易的厄米算符),这样就有
\begin{equation}
    \comm*{\hat{A}_s}{\hat{H}} = \comm*{\hat{B}_p}{\hat{H}} = 0.
\end{equation}
另一方面,平方为1的厄米算符的本征值是$\pm 1$,于是我们就可以用它们的本征值$A_s = \pm 1$和$B_p = \pm 1$标记体系的能量本征态。
实际上,在热力学极限下只需要$\{A_s\}$和$\{B_p\}$就可以唯一地标记体系的能量本征态。
这是因为设体系有$N$个格点,那么有$4N/2=2N$条边,于是体系的希尔伯特空间的维数为$2^{2N}$。
$s$和$p$均有$N$个,于是所有可能的$\{A_s\}$和$\{B_p\}$的组合总数为$2^N \cdot 2^N=2^{2N}$。
这样如果不考虑边界引入的微妙之处,只需要$\{A_s\}$和$\{B_p\}$就可以唯一地标记体系的能量本征态。
很容易看出体系的基态为所有$A_s$和$B_p$均为$1$的状态,于是我们可以把$A_s$和$B_p$为$-1$的情况看成激发态。这样我们就得到了\eqref{eq:toric-code-hamiltonian}的全部能量本征态,从而完全求解出了它。

与普通的\Ztwo规范场模型\eqref{eq:z2-2d-hamiltonian}相比,\eqref{eq:toric-code-hamiltonian}的$\hat{\sigma}^x$项是四算符乘积而不是单个算符,就是这个特征让toric-code模型的哈密顿量可以写成一系列彼此对易的算符之和,从而让它严格可解,不仅基态严格可解,整个能谱都是严格可解的。

\subsection{环面上的情况}

为了解析求解,我们施加一个周期性边界条件,这相当于把体系放在了一个二维环面上。
此时诸$\{A_s\}$和$\{B_p\}$实际上不是彼此独立的,因为此时显然有
\[
    \prod_s \hat{A}_s = 1,
\]
因为所有的$\{A_s\}$乘起来,每一条边被乘了两边,所以一定会得到$1$。类似的有
\[
    \prod_p \hat{B}_p = 1.
\]
这两个方程要求
\begin{equation}
    \prod_{s} A_s = \prod_{p} B_p = 1.
    \label{eq:toric-code-pair-condition}
\end{equation}
这就意味着$A_s$激发和$B_p$激发必须成对出现,否则乘积将会是$-1$。我们将$A_s$激发称为e粒子,而将$B_p$激发称为m粒子,因为在某种意义上可以将$A_s$激发类比为电荷而将$B_p$激发理解为磁通量子。
这两种粒子的性质和空间的拓扑结构显然关系很大,因此称它们为拓扑激发。

两种激发成对出现的事实意味着可以使用弦算符描述它们的产生和消灭。首先考虑由一条边连接的两个格点,这条边上的$\hat{\sigma}^z_i$算符可以将这条边上的$x$方向的自旋翻转,因此它可以做到以下三件事:
\begin{itemize}
    \item 如果两个格点上原本没有e粒子,那么在两个格点上同时产生e粒子;
    \item 如果两个格点上原本都有e粒子,那么在两个格点上同时消灭e粒子; 
    \item 如果两个格点一个有e粒子一个没有,那么该e粒子将被转移到原本没有e粒子的格点上。
\end{itemize}
这样设一系列首尾相连的边$\{i\}$连接了两个格点,则弦算符
\begin{equation}
    \hat{O}_\text{e} = \prod_{i} \hat{\sigma}_i^z
\end{equation}
同样可以做到以上三件事。
同样,将以上论述中的$\hat{\sigma}^z$换成$\hat{\sigma}^x$,“格点”换成“正方形格子”,“连接两个格点的边”换成“正方形格子共享的边”(我们可以在每个正方形格子中间放置一个点,从而m粒子也定义在一个格点上),同样可以定义弦算符
\begin{equation}
    \hat{O}_\text{m} = \prod_{i} \hat{\sigma}_i^x.
\end{equation}
以上讨论的都是开放的弦,闭合的弦的行为需要具体分析,且对闭弦有
\begin{equation}
    \hat{O}_\text{e} \ket{0} = \hat{O}_\text{m} \ket{0} = \ket{0}.
\end{equation}

通过弦算符可以检查e粒子和m粒子绕对方转一圈(实际上就是使用一个闭合的弦算符作用在一个有e粒子或者m粒子的格点上),都会多出来一个$\pi$的相位,这是因为如果一个m粒子闭弦和一个e粒子开弦有单个交点,那么它们反对易(因为同一个边上的$\hat{\sigma}^x$和$\hat{\sigma}^z$反对易)。
换而言之,e粒子和m粒子均为任意子激发:这是二维的特殊现象,因为二维的环路在二维平面上它围绕的区域被挖掉一个点之后就不可缩了,因此一个粒子转一圈之后可以有一个非零相位变化。
本节涉及的激发尚为阿贝尔统计,即转一圈之后得到的量子态和转之前只差一个$U(1)$变换;还有非阿贝尔统计,即转一圈可以转移到别的量子态上。

\subsection{任意子表}

现在的问题是,环面上的Toric-code模型中最多能够弄出来多少任意子?显然e粒子和m粒子都是任意子,虽然两者自己满足玻色统计,但它们之间有一个非平凡的相位。
我们下面将以拓扑性质分类激发,即,拓扑性质相同的激发算作一种。
可以用两个量来标记一种激发的拓扑性质:设$M_{ab}$为$b$绕着$a$转一圈导致的复数因子,$\theta_a$指的是交换两个$a$导致的复数因子(或者说一个$a$绕着另一个$a$转半圈导致的复数因子)。这样,有
\begin{equation}
    \theta_\mathrm{e} = \theta_\mathrm{m} = 1, \quad M_\mathrm{em} = - 1.
\end{equation}
除了e粒子和m粒子以外肯定还有一种$\mathrm{\epsilon}$粒子,它是一个e粒子和m粒子聚合%
\footnote{所谓聚合指的是将两个激发放得尽可能近,从而得到的复合激发。e粒子和m粒子定义在不同的格点上,因此一个e粒子和一个m粒子的聚合就是在一个正方格子中央放置一个m粒子,在它的某个角上放置一个e粒子之后得到的激发,从远处看这近似于一个粒子。}%
而成的粒子,即
\begin{equation}
    \mathrm{\epsilon} = \mathrm{e} \otimes \mathrm{m}.
\end{equation}
可以容易地验证
\begin{equation}
    M_\mathrm{e\epsilon} = M_\mathrm{m\epsilon} = -1, \quad \theta_\mathrm{\epsilon} = -1.
\end{equation}
e粒子、m粒子和$\mathrm{\epsilon}$粒子这三种拓扑激发都只能成对出现。
除了这三种激发以外还有一些平凡的激发,比如声子之类,将它们全部记为$\mathbbm{1}$。

实际上,e粒子、m粒子和$\epsilon$粒子和$\mathbbm{1}$就是全部拓扑激发。
由于$\mathbbm{1}$无论如何绕圈都不会产生附加的相位,就有
\[
    \mathbbm{1} \otimes a = a.
\]
两个e粒子放在一起,得到的就是某个边上的$\sigma^x$发生了翻转,这是一个普通的激发;m粒子和$\mathrm{\epsilon}$粒子也是如此,于是
\[
    \mathrm{e} \otimes \mathrm{e} = \mathrm{m} \otimes \mathrm{m} = \mathrm{\epsilon} \otimes \mathrm{\epsilon} = \mathbbm{1}.
\]
上式实际上说明了一个非常重要的事实:封闭流形上无论有多少拓扑激发,这个态都可以通过对基态作用一些产生算符得到,或者等价地说改变基态上某些格点的值得到,那么如果将这些拓扑激发聚合到一起,得到的只是基态上局域的一些点被改变了,也即得到了一个平凡的激发。
总之,封闭流形上所有的拓扑激发聚合在一起,只会得到平凡的激发。这就从另一个角度解释了为什么非平凡的拓扑激发一定成对出现。
$\mathrm{\epsilon}$和e粒子聚合,就相当于两个e粒子先聚合得到一个平凡的激发,剩下一个m粒子,$\mathrm{\epsilon}$粒子和m粒子聚合则会留下一个e粒子和一个平凡的激发,于是
\[
    \mathrm{\epsilon} \otimes \mathrm{e} = \mathrm{m}, \quad \mathrm{\epsilon} \otimes \mathrm{m} = \mathrm{e}.
\]
因此,e粒子、m粒子和$\epsilon$粒子和$\mathbbm{1}$在聚合运算$\otimes$下是封闭的。

\subsection{四重简并和Berry相}

回忆一下,体系的希尔伯特空间维数为$2^{2N}$。当$2N-2$个边的自旋已经确定之后,系统的状态实际上已经确定了,因为约束条件\eqref{eq:toric-code-pair-condition}会确定剩下两条边的自旋。
换而言之,实际物理的希尔伯特空间维数只有$2^{2N-2}$。
这就意味着总希尔伯特空间$2^{2N}$分裂成了4支,或者说每个状态都有四重简并。

用什么标记这四重简并?容易想到,完全可以定义一种全局性的闭弦算符,它贯穿整个环面,而由周期性边界条件它是闭弦算符。
分别沿着$x$轴和$y$轴定义
\begin{equation}
    \hat{L}^x_\text{e} = \prod_{x} \hat{\sigma}^z_i, \quad \hat{L}^x_\text{m} = \prod_{x} \hat{\sigma}^x_i,
\end{equation}
并可以验证它们和哈密顿量是对易的,且它们构成一对对易稳定子。
这就意味着它们的本征值均为$\pm 1$,这就唯一地标记了四重简并。

$\hat{L}^x_\text{e}$和$\hat{L}^x_\text{m}$将e粒子绕着$x$轴转动一圈,因此它们的本征值实际上给出了x方向类似于磁通量的一个通量,这个通量导致了一个Berry相位。

类似地还可以定义$\hat{L}^y_\text{e}$和$\hat{L}^y_\text{m}$,并且
\begin{equation}
    \acomm*{\hat{L}^x_\text{e}}{\hat{L}^y_\text{m}} = 0.
\end{equation}
我们知道
\begin{equation}
    \ket{0} = \ket{L_\text{e}^y=1, L_\text{m}^y=1},
\end{equation}
而使用这些关系可以证明,
\begin{equation}
    \begin{aligned}
        \hat{L}^x_\text{e} \ket{0} &= \ket{L_\text{e}^y=1, L_\text{m}^y=-1}, \\
        \hat{L}^x_\text{m} \ket{0} &= \ket{L_\text{e}^y=-1, L_\text{m}^y=1}, \\
        \hat{L}^x_\text{m} \hat{L}^x_\text{e} \ket{0} &= \ket{L_\text{e}^y=-1, L_\text{m}^y=-1}.
    \end{aligned}
\end{equation}
我们发现四重简并和四种基本的任意子正好能够对应上。这是拓扑序的一般特征:基态简并和任意子有对应,基态简并数目就是任意子数目的亏格次方。
我们这里是在亏格(洞的数目)为1的环面上工作,因此基态简并的数目为$4^1=4$种。
如果在亏格为0的球面上,基态简并的数目就是$4^0=1$种。
还有另一种方法也可以推导出这个结果。设亏格为$g$,由欧拉公式
\[
    V - E + F = 2 - 2g,
\]
于是
\[
    E - (V + F - 2) = 2g.
\]
而$V$是$A_s$格点的数目,$F$是$B_p$格点的数目,再减去\eqref{eq:toric-code-pair-condition}造成的两个约束,则$V+F-2$是一个二维表面Toric-code态的自由度个数。
Toric-code模型总的自由度个数为$E$,因此有$2g$个自由度用于标记简并态,由于每个自由度有两个取值,简并度为
\[
    2^{2g} = 4^g.
\]

\section{外加磁场的}


\end{document}