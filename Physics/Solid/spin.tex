\documentclass[hyperref, UTF8, a4paper]{ctexart}

\usepackage{geometry}
\usepackage{titling}
\usepackage{titlesec}
\usepackage{paralist}
\usepackage{footnote}
\usepackage{enumerate}
\usepackage{amsmath, amssymb, amsthm}
\usepackage{bbm}
\usepackage{cite}
\usepackage{graphicx}
\usepackage{subfigure}
\usepackage{physics}
\usepackage{tikz}
\usepackage{autobreak}
\usepackage[ruled, vlined, linesnumbered, noend]{algorithm2e}
\usepackage[colorlinks, linkcolor=black, anchorcolor=black, citecolor=black]{hyperref}
\usepackage{prettyref}

% Page style
\geometry{left=3.18cm,right=3.18cm,top=2.54cm,bottom=2.54cm}
\titlespacing{\paragraph}{0pt}{1pt}{10pt}[20pt]
\setlength{\droptitle}{-5em}
\preauthor{\vspace{-10pt}\begin{center}}
\postauthor{\par\end{center}}

% Math operators
\DeclareMathOperator{\timeorder}{T}
\DeclareMathOperator{\diag}{diag}
\DeclareMathOperator{\legpoly}{P}
\DeclareMathOperator{\primevalue}{P}
\DeclareMathOperator{\sgn}{sgn}
\newcommand*{\ii}{\mathrm{i}}
\newcommand*{\ee}{\mathrm{e}}
\newcommand*{\const}{\mathrm{const}}
\newcommand*{\comment}{\paragraph{注记}}
\newcommand*{\suchthat}{\quad \text{s.t.} \quad}
\newcommand*{\argmin}{\arg\min}
\newcommand*{\argmax}{\arg\max}
\newcommand*{\normalorder}[1]{: #1 :}
\newcommand*{\pair}[1]{\langle #1 \rangle}
\newcommand*{\fd}[1]{\mathcal{D} #1}
\DeclareMathOperator{\bigO}{\mathcal{O}}

% prettyref setting
\newrefformat{sec}{第\ref{#1}节}
\newrefformat{note}{注\ref{#1}}
\newrefformat{fig}{图\ref{#1}}
\newrefformat{alg}{算法\ref{#1}}
\renewcommand{\autoref}{\prettyref}

% TikZ setting
\usetikzlibrary{arrows,shapes,positioning}
\usetikzlibrary{arrows.meta}
\usetikzlibrary{decorations.markings}
\tikzstyle arrowstyle=[scale=1]
\tikzstyle directed=[postaction={decorate,decoration={markings,
    mark=at position .5 with {\arrow[arrowstyle]{stealth}}}}]
\tikzstyle ray=[directed, thick]
\tikzstyle dot=[anchor=base,fill,circle,inner sep=1pt]

% Algorithm setting
\renewcommand{\algorithmcfname}{算法}
% Python-style code
\SetKwIF{If}{ElseIf}{Else}{if}{:}{elif:}{else:}{}
\SetKwFor{For}{for}{:}{}
\SetKwFor{While}{while}{:}{}
\SetKwInput{KwData}{输入}
\SetKwInput{KwResult}{输出}
\SetArgSty{textnormal}

\renewcommand{\emph}[1]{\textbf{#1}}
\newcommand*{\concept}[1]{\underline{\textbf{#1}}}
\newcommand*{\Ztwo}{$\mathbb{Z}_2$}

\title{自旋和磁性}
\author{吴何友}

\begin{document}

\maketitle

\section{自旋模型和磁性}

\subsection{自旋自由度}

自旋自由度可能来自电子,也可能来自原子核。自由电子和离子实中的束缚电子都有自旋自由度,后者会合成产生一个角量子数很大的自旋。

自旋$1/2$的情况下有很多额外的优良性质。为简化公式形式,我们将认为$z$方向自旋可以取$\pm 1$而不是$\pm 1/2$,并且用$\sigma$表示有关的自由度,因为这是泡利矩阵的符号而泡利矩阵的本征值取$\pm 1$。
在自旋$1/2$的系统中有很多非常方便的性质,如$\hat{\sigma}^x$可以将$\hat{\sigma}^z$的本征态翻转,即将$\ket{\sigma^z=1}$变为$\ket{\sigma^z=-1}$,反之亦然。
此外,我们有
\begin{equation}
    \braket{\sigma^x}{\sigma^z} = \frac{1}{\sqrt{2}} \exp(\ii \pi \frac{1 - \sigma^x}{2} \frac{1 - \sigma^z_l}{2}),
\end{equation}
并且实际上上式是一个实数。

\subsection{自旋-粒子对应}

自旋的路径积分表述比较复杂(因为需要在$S^2$上做积分,而不是完全平直的场构型空间),而自旋模型中出现“波”的激发似乎也是非常合理的。
那么,很自然的想法就是,可以找到某种自旋-粒子对应,使得一方面可以很容易地做自旋的路径积分,一方面可以直接分析自旋模型的低能自由度。

\subsubsection{Jordan-Weigner变换}

\concept{Jordan-Weigner变换}可以将自旋$1/2$的自旋自由度转化为费米子自由度。
实际上,经过一些扩展,它可以把任意的自旋自由度转化为费米子自由度。

对$1/2$的自旋自由度,容易想到的是将自旋角动量指向$+z$方向当成一种激发,那么就需要设
\[
    \hat{\sigma}^z_i = 2 \hat{c}^\dagger_i \hat{c}_i - 1.
\]
这样还不能确定$\hat{c}$和$\hat{c}^\dagger$,首先尝试设
\[
    \left\{
        \begin{aligned}
            \hat{\sigma}^z_i &= 2 \hat{c}^\dagger_i \hat{c}_i - 1, \\
            \hat{\sigma}^x_i &= \hat{c}^\dagger_i + \hat{c}_i, \\
            \hat{\sigma}^y_i &= \ii (\hat{c}_i - \hat{c}^\dagger_i),
        \end{aligned}
    \right.
\]
容易验证,在假定上式成立的情况下自旋角动量代数成立等价于$\hat{c}_i^\dagger$和$\hat{c}_i$服从费米子的反对易关系,于是有关的$\{\hat{c}_i\}$确实存在。
上面的定义的不足之处在于,对$i \neq j$,$\hat{c}_i$和$\hat{c}_j$之间有对易关系而非费米子应该有的反对易关系。

为了解决这个问题,需要做一个Klein变换。首先设
\begin{equation}
    \left\{
        \begin{aligned}
            \hat{\sigma}^z_i &= 2 \hat{f}^\dagger_i \hat{f}_i - 1, \\
            \hat{\sigma}^x_i &= \hat{f}^\dagger_i + \hat{f}_i, \\
            \hat{\sigma}^y_i &= \ii (\hat{f}_i - \hat{f}^\dagger_i),
        \end{aligned}
    \right.
\end{equation}
从而
\begin{equation}
    \hat{f}_i^\dagger = \frac{1}{2} (\hat{\sigma}^x_i + \ii \hat{\sigma}^y_i), \quad \hat{f}_i = \frac{1}{2} (\hat{\sigma}^x_i - \ii \hat{\sigma}^y_i),
\end{equation}
则可以验证,若
\begin{equation}
    \hat{c}^\dagger_i = \ee^{\ii \pi \sum_{k < i} \hat{f}_k^\dagger \hat{f}_k} \hat{f}^\dagger_i, \quad \hat{c}_i = \ee^{- \ii \pi \sum_{k < i} \hat{f}_k^\dagger \hat{f}_k} \hat{f}_i,
\end{equation}
则$(\hat{c}^\dagger_i, \hat{c}_i)$服从正确的费米子对易关系。
我们设
\begin{equation}
    \hat{K}_i = \ee^{\pm \ii \pi \sum_{k < i} \hat{f}_k^\dagger \hat{f}_k} = \prod_{k < i} (1 - 2 \hat{f}^\dagger_k \hat{f}_k) = \prod_{k < i} (- \hat{\sigma}_k^z),
\end{equation}
就得到
\begin{equation}
    \left\{
        \begin{aligned}
            \hat{\sigma}^z_i &= 2 \hat{c}^\dagger_i \hat{c}_i - 1, \\
            \hat{\sigma}^x_i &= \hat{K}_i (\hat{c}^\dagger_i + \hat{c}_i), \\
            \hat{\sigma}^y_i &= \ii \hat{K}_i (\hat{c}_i - \hat{c}^\dagger_i).
        \end{aligned}
    \right.
\end{equation}
以上各式给出了Jordan-Weigner变换。
这样得到的费米子自由度并不是局域的,因为点$i$的费米子自由度和$i$个自旋自由度都有关系。

自旋的$\mathbb{Z}_2$对称性在经过Jordan-Weigner变换之后就是电子-空穴对称性。

\subsubsection{Holstein-Primakoff变换}

也可以通过一些变换将自旋自由度转化为玻色子自由度。本节讨论一般的、角量子数(设为$S$)可大可小的自旋自由度。
由于是将自旋自由度转化为玻色子自由度,我们可以直接在每个格点上工作,只需要验证最后不同格点上的玻色子产生湮灭算符相互对易就可以了。
以取最大$S^z$值的状态为真空态,每下降一点自旋就认为激发出了一个玻色子,那么非常合理地可以设
\[
    \hat{S}_i^z = S - \hat{a}^\dagger_i \hat{a}.
\]
由于$\hat{S}_i^\pm$是$\hat{S}^z_i$的升降算符,而$z$方向自旋越大,玻色子越少,则显然
\[
    \hat{S}_i^+ = \frac{1}{\sqrt{2}} (\hat{S}^x_i + \ii \hat{S}^y_i) \sim \hat{a}_i, \quad \hat{S}_i^- = \frac{1}{\sqrt{2}} (\hat{S}^x_i - \ii \hat{S}^y_i) \sim \hat{a}_i^\dagger.
\]
可以验证我们取
\begin{equation}
    \hat{S}^+_i = \sqrt{S} \sqrt{1 - \frac{\hat{a}_i^\dagger \hat{a}_i}{2S}} \hat{a}_i, \quad \hat{S}^-_i = \sqrt{S} \hat{a}_i^\dagger \sqrt{1 - \frac{\hat{a}_i^\dagger \hat{a}_i}{2S}}
\end{equation}
即可让角动量代数成立等价于$\hat{a}^\dagger, \hat{a}$满足玻色子对易关系。
这就是\concept{Holstein-Primakoff变换},这是一个完全局域的变换。当$S$很大时,可以做泰勒级数展开。

\subsection{自旋的动力学}

自旋可以和外加磁场发生耦合。我们知道外加弱磁场会导致一个线性响应
\[
    \hat{H}_\text{mag} = - \hat{\vb*{\mu}} \cdot \vb*{B},
\]
而由于磁矩$\hat{\vb*{\mu}}$和自旋$\hat{\vb*{S}}$是正比的,

自旋之间的相互作用主要来自\concept{交换能}。这是一个纯粹量子力学的概念,是由于全同电子波函数的反对称性。由于自旋是矢量,从矢量到标量的运算主要就是点乘,
% TODO

\section{常见模型}

\subsection{经典伊辛模型}

所谓\concept{经典伊辛模型}指的是
\begin{equation}
    \hat{H} = - \sum_{\pair{i, j}} J_{ij} \hat{\sigma}_i^z \hat{\sigma}_j^z + \sum_{i} B_i \hat{\sigma}_i^z,
\end{equation}
它被称为是\emph{经典}的是因为其能量本征态就是($z$方向上的)自旋本征态,因此一个伊辛模型的统计配分函数就是简单的
\begin{equation}
    Z = \sum_{\{s_i\}} \exp(\beta \sum_{\pair{i, j}} J_{ij} s_i s_j - \beta \sum_i B_i s_i),
\end{equation}
即我们可以将$\hat{\sigma}_i^z$当成经典变量。当然也可以使用量子的路径积分来写出配分函数,但由于这样明显麻烦,通常不会这么做。

\subsection{横场伊辛模型}

\concept{横场伊辛模型}定义为
\begin{equation}
    \hat{H} = - \sum_{\pair{i, j}} J_{ij} \hat{\sigma}_i^z \hat{\sigma}_j^z + \sum_{i} B_i \hat{\sigma}_i^x,
\end{equation}
唯一的区别就是磁场加在$x$方向而不是$z$方向(即\emph{横场}),但是这个区别意味着横场伊辛模型具有零温量子涨落。

一个$d$维量子横场伊辛模型对应一个$d+1$维的经典伊辛模型。要看出这是为什么,注意到
\[
    \begin{aligned}
        \mel{\sigma^z(\tau + \Delta \tau)}{\ee^{-\Delta \tau \hat{H}}}{\sigma^z(\tau)} &= \mel{\sigma^z(\tau + \Delta \tau)}{\ee^{-\Delta \tau \sum_i B_i \hat{\sigma}^x_i} \ee^{\Delta \tau \sum_{\pair{i, j}} J_{ij} \hat{\sigma}_i^z \hat{\sigma}_j^z}}{\sigma^z(\tau)} \\
        &= \ee^{\Delta \tau \sum_{\pair{i, j}} J_{ij} \sigma_i^z \sigma_j^z} \mel{\sigma^z(\tau + \Delta \tau)}{\ee^{-\Delta \tau \sum_i B_i \hat{\sigma}^x_i}}{\sigma^z(\tau)} \\
        &= \ee^{\Delta \tau \sum_{\pair{i, j}} J_{ij} \sigma_i^z \sigma_j^z} \sum_{\{\sigma^x_i\}} \ee^{-\Delta \tau \sum_i B_i \sigma^x_i} \braket{\sigma^z(\tau + \Delta \tau)}{\sigma^x} \braket{\sigma^x}{\sigma^z(\tau)} \\
        &= \ee^{\Delta \tau \sum_{\pair{i, j}} J_{ij} \sigma_i^z \sigma_j^z} \prod_{i} \sum_{\sigma^x_i} \ee^{-\Delta \tau B_i \sigma^x_i} \braket{\sigma^z_i(\tau + \Delta \tau)}{\sigma^x_i} \braket{\sigma^x_i}{\sigma^z_i(\tau)} \\
        &= \ee^{\Delta \tau \sum_{\pair{i, j}} J_{ij} \sigma_i^z \sigma_j^z} \prod_{i} \sum_{\sigma^x_i = \pm 1} \ee^{-\Delta \tau B_i \sigma^x_i} \frac{1}{2} \ee^{\ii \pi \frac{1 - \sigma^x_i}{2} \left( \frac{1 - \sigma^z_i(\tau)}{2} + \frac{1 - \sigma^z_i(\tau + \Delta \tau)}{2} \right)} \\
        &= \frac{1}{2^N} \ee^{\Delta \tau \sum_{\pair{i, j}} J_{ij} \sigma_i^z \sigma_j^z} \prod_i \left( \ee^{- \Delta \tau B_i} + \ee^{\Delta \tau B_i} \ee^{\ii \pi \frac{1 - \sigma^z_i(\tau)}{2}} \ee^{\ii \pi \frac{1 - \sigma^z_i(\tau + \Delta \tau)}{2}} \right) \\
        &= \frac{1}{2^N} \ee^{\Delta \tau \sum_{\pair{i, j}} J_{ij} \sigma_i^z \sigma_j^z} \prod_i \left( \ee^{- \Delta \tau B_i} + \ee^{\Delta \tau B_i} \sigma^z_i(\tau) \sigma^z_i(\tau + \Delta \tau) \right),
    \end{aligned}
\]
请注意$\sigma_i^z$只取$\pm 1$,于是我们有
\[
    \cosh J_\tau + \sinh J_\tau \sigma^z_i(\tau) \sigma^z_i(\tau + \Delta \tau) = \ee^{J_\tau \sigma^z_i(\tau) \sigma^z_i(\tau + \Delta \tau)},
\]
那么只需要定义
\begin{equation}
    \tanh J^\tau_i = \ee^{2 \Delta \tau B_i}, 
\end{equation}
就有
\[
    \begin{aligned}
        \mel{\sigma^z(\tau + \Delta \tau)}{\ee^{-\Delta \tau \hat{H}}}{\sigma^z(\tau)} &\propto \ee^{\Delta \tau \sum_{\pair{i, j}} J_{ij} \sigma_i^z \sigma_j^z} \prod_i \ee^{J_i^\tau \sigma_i^z(\tau) \sigma_i^z(\tau + \Delta \tau)} \\
        &= \ee^{\Delta \tau \sum_{\pair{i, j}} J_{ij} \sigma_i^z \sigma_j^z} \ee^{\sum_i J_i^\tau \sigma_i^z(\tau) \sigma_i^z(\tau + \Delta \tau)},
    \end{aligned}
\]
容易看出这最终导致一个三维各向异性(虚时间维的$J$和空间维的$J$无关)经典伊辛模型的配分函数。

\subsection{海森堡模型}

二维磁针是XY模型,三维是海森堡模型。

一个\concept{海森堡模型}指的是这样的一个哈密顿量:
\begin{equation}
    \hat{H} = \sum_{\pair{i, j}} \hat{\vb*{S}}_i \cdot \hat{\vb*{S}}_j
\end{equation}

\subsection{XXZ模型}

一个\concept{XXZ模型}指的是这样的一个哈密顿量:
\begin{equation}
    \hat{H} = J_x \sum_i (\hat{\sigma}_{i}^x \hat{\sigma}_{i+1}^x + \hat{\sigma}_{i}^y \hat{\sigma}_{i+1}^y) + J_z \sum_{i} \hat{\sigma}_i^z \hat{\sigma}_{i+1}^z.
\end{equation}

\section{一维自旋链}

\end{document}