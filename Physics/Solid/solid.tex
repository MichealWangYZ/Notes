\documentclass[hyperref, UTF8, a4paper]{ctexart}

\usepackage{geometry}
\usepackage{titling}
\usepackage{titlesec}
\usepackage{paralist}
\usepackage{footnote}
\usepackage{enumerate}
\usepackage{amsmath, amssymb, amsthm}
\usepackage{cite}
\usepackage{graphicx}
\usepackage{subfigure}
\usepackage{physics}
\usepackage{tikz}
\usepackage[colorlinks, linkcolor=black, anchorcolor=black, citecolor=black]{hyperref}
\usepackage{prettyref}

\geometry{left=3.18cm,right=3.18cm,top=2.54cm,bottom=2.54cm}
\titlespacing{\paragraph}{0pt}{1pt}{10pt}[20pt]
\setlength{\droptitle}{-5em}
\preauthor{\vspace{-10pt}\begin{center}}
\postauthor{\par\end{center}}

\DeclareMathOperator{\timeorder}{T}
\DeclareMathOperator{\diag}{diag}
\newcommand*{\ii}{\mathrm{i}}
\newcommand*{\ee}{\mathrm{e}}
\newcommand*{\const}{\mathrm{const}}
\newcommand*{\comment}{\paragraph{注记}}
\newcommand*{\suchthat}{\quad \text{s.t.} \quad}
\newcommand*{\argmin}{\arg\min}

\newrefformat{sec}{第\ref{#1}节}
\newrefformat{note}{注\ref{#1}}
\renewcommand{\autoref}{\prettyref}

\newenvironment{bigcase}{\left\{\quad\begin{aligned}}{\end{aligned}\right.}

\title{固体理论}
\author{wujinq}

\begin{document}

\maketitle

记号约定:费米子的产生湮灭算符为$\hat{c}^\dagger$和$\hat{c}$,而如果是关于位置的产生湮灭算符,则为$\hat{\psi}^\dagger$和$\hat{\psi}$。

\section{固体的组成部分}

\subsection{相互作用电子气}

考虑非相对论极限下的电子-电磁场耦合系统,电子由薛定谔场完全描述,电磁场由库伦势完全描述。由此产生的系统具有$U(1)$对称性,因此无粒子数生灭,可以直接从单粒子量子力学出发构造其哈密顿量。该体系称为\textbf{相互作用电子气}。

% TODO:怎么没考虑自旋??
使用自然单位制。单体哈密顿量为电子的动能项加上单体势能项:
\[
    \hat{H}_1 = \frac{\hat{\vb*{p}}}{2m} + V(\vb*{r}),
\]
在坐标表象下它就是
\[
    \hat{H}_1 = - \frac{\laplacian}{2m} + V(\vb*{r}).
\]
二体哈密顿量为电子两两作用而产生的库伦势能是
\[
    \hat{H}_2 = \frac{e^2}{\abs{\vb*{r}_1 - \vb*{r}_2}},
\]
从而,相互作用电子气的二次量子化哈密顿量为
\begin{equation}
    \begin{aligned}
        \hat{H} = &\int \dd[3]{\vb*{r}} \hat{\psi}^\dagger(\vb*{r}) \left( - \frac{\laplacian}{2m} + V(\vb*{r}) \right) \hat{\psi}(\vb*{r}) \\
        &+ \frac{1}{2} \int \dd[3]{\vb*{r}_1} \int \dd[3]{\vb*{r}_2} \hat{\psi}^\dagger (\vb*{r}_1) \hat{\psi}^\dagger (\vb*{r}_2) \frac{e^2}{\abs{\vb*{r}_1 - \vb*{r}_2}} \hat{\psi} (\vb*{r}_2) \hat{\psi}(\vb*{r}_1). 
        \label{eq:electron-gas-hamiltonian}
    \end{aligned}
\end{equation}
其中$\hat{\psi}^\dagger(\vb*{r})$是薛定谔场的场算符,它也是在位置为$\vb*{r}$的位置产生一个电子的产生算符。这个哈密顿量当然也可以通过QED的低能近似得到,但并没有必要这么做。请注意电子是费米子。
\eqref{eq:electron-gas-hamiltonian}实际上不是对角的,因为它的单粒子项涉及一个梯度算符。

\subsubsection{Hatree-Fock近似}

假定体系\eqref{eq:electron-gas-hamiltonian}的基态近似为
\begin{equation}
    \ket{\text{HF}} = \sum_\alpha \hat{c}_\alpha \ket{0},
    \label{eq:hatree-fock-ansatz}
\end{equation}
其中参与求和的$\hat{c}_\alpha$共有$n$个,$n$是事先给定的系统中的粒子数。
这个拟设等于是说,体系的基态和某个乘积态非常接近。一般而言,不能够保证\eqref{eq:electron-gas-hamiltonian}真的有形如\eqref{eq:hatree-fock-ansatz}的本征态,但是我们总是可以让\eqref{eq:hatree-fock-ansatz}的能量期望值取最小值,也即求解以下问题:
\begin{equation}
    \argmin_{\hat{\psi}} \mel{\text{HF}}{\hat{H}}{\text{HF}} \suchthat \text{$\hat{\psi}$ is a field operator}.
    \label{eq:minimize-energy}
\end{equation}
由于Hatree-Fock态是乘积态,可以使用Wick定理,并注意到依照定义有
\[
    \mel{0}{\hat{\psi}(\vb*{r})\hat{c}^\dagger_\alpha}{0} = \braket{\vb*{r}}{\phi_\alpha} = \phi_\alpha(\vb*{r}),
\]
$\phi_\alpha(\vb*{r})$为产生算符$\hat{c}^\dagger_\alpha$产生的粒子在坐标空间中的波函数,可以计算出
\[
    E_\text{HF} = \mel{\text{HF}}{\hat{H}}{\text{HF}} = 
\]
于是最优化问题\eqref{eq:minimize-energy}就变成一个约束优化问题:
\[
    \argmin_{\phi_\alpha(\vb*{r})} E_\text{HF} \suchthat \int \dd[3]{\vb*{r}} \abs{\phi_\alpha(\vb*{r})}^2 = 1, \quad \text{for all $\alpha$}.
\]

% 无能隙的激发能够把特别奇异的相互作用屏蔽掉(RPA近似),从而让微扰展开收敛得非常快。如果担心RPA不对,只需要把对的部分加上做微扰就可以了
% 通常称玻色型的模式为“集体激发”,而称费米型的模式为“准粒子”

\end{document}