\documentclass[a4paper]{article}

\usepackage{geometry}
\usepackage{caption}
\usepackage{subcaption}
\usepackage{abstract}
\usepackage{amsmath, amssymb}
\usepackage{qtree}
% \usepackage{gb4e}
\usepackage[colorlinks, linkcolor=black, anchorcolor=black, citecolor=black]{hyperref}
\usepackage{prettyref}

\geometry{left=3.18cm,right=3.18cm,top=2.54cm,bottom=2.54cm}

\DeclareMathOperator{\pmergel}{PMergeL}
\DeclareMathOperator{\pmerger}{PMergeR}
\DeclareMathOperator{\hmergel}{HMergeL}
\DeclareMathOperator{\hmerger}{HMergeR}
\DeclareMathOperator{\copyl}{CopyL}
\DeclareMathOperator{\copyr}{CopyR}
\DeclareMathOperator{\agree}{Agree}
\DeclareMathOperator{\move}{Move}
\DeclareMathOperator{\undefined}{undefined}
\setlength{\arraycolsep}{2pt}
\newcommand*{\synbracket}[2][{}]{[_\mathrm{#1} \; \begin{matrix} #2 \end{matrix} \; ]}

\newrefformat{fig}{Figure \ref{#1}}

\newcommand*{\concept}[1]{\underline{\textbf{#1}}}

\title{The Framework of Minimalist Syntax}
\author{wujinq}

\begin{document}

\maketitle

\begin{abstract}
    This article is a summary of Minimalist syntax in modern generative grammar.
    We review the mutually recognized primitives of the Minimalist Program, its main motivation, and its main achievements.
    We also present main disputes - both theoretical and empirical issues around this research program, and list some dominant frameworks trying to settle these issues.  
\end{abstract}

\section{Why Minimalism}

The Principle and Parameter 

\section{Syntactic objects and operations}\label{sec:syn-obj}

This section deals with syntactic objects, like morphemes, words, phrases, and the operations used to build them.
We are not to give a formalization of Minimalist syntax here so we will not use much technical terms, 
and often discussion about a certain concept will involves concepts yet to be defined.

\subsection{Syntactic objects and their structures}

A \concept{syntactic object} may be a morpheme, a word or a constituent.
It is either taken directly from the \concept{lexicon}, 
or formed by syntactic operations with other syntactic objects.
A \concept{morphemes} is a minimal unit in the lexicon, which is a bundle of features.
A \concept{feature} is a key-value pair specifying attributes of a certain syntactic object.
It may express something about the object itself 
(for example whether it is referential, whether it is in single form or plural form, or if it is countable.), 
and in this case it is called a \concept{categorical feature};
or it may instruct how its host can be merged with other objects 
(for example a transitive verb is to be merged with a nominal expression), 
and in this case it is called a \concept{selector feature}
 - correspondingly, categorical features of the object to be selected to merge is called \concept{selectee features}.
A \concept{word} is a object without selector features that select morphemes (or, equivalently, trigger head merge), 
which may be a single morpheme or a tree of morphemes constructed by head merge.
A \concept{constituent} is an object 
that is a word or is created by phrasal merging of a word and a constituent, or two constituents.
A constituent without any selector features that drive phrasal merge is called a \concept{phrase}.

Constituents are also called as \concept{projections}, 
because they can be formed by merging a single word with a constituent, and merging the result with some constituent else, and on and on, so the single word in the beginning \concept{projects into a constituent}.
And since a phrase no longer selects other constituent, it is called a \concept{maximal projection}.
In contrary, non-phrase constituents are called \concept{intermediate projections}.
\prettyref{fig:x-bar-structure-one} shows a constituent formed directly by one word.
\prettyref{fig:x-bar-structure-two}, \prettyref{fig:x-bar-structure-three} and \prettyref{fig:x-bar-structure-four} shows a series of constituents with increasing numbers of arguments.
Take \prettyref{fig:x-bar-structure-four} as an example.
X is a word, and it is first merged with YP, and then ZP, and finally WP. 
As the merge process goes on, the constituent's tree diagram expands, 
just like the expansion of the light cone of a projector as we goes far.
That is why constituents are called projections. 

\begin{figure}
    \centering
    \begin{minipage}[b]{0.4\linewidth}
        \Tree [.XP {X \\ the head} ]
        \subcaption{Bare head as a constituent}
        \label{fig:x-bar-structure-one}
    \end{minipage}
    \begin{minipage}[b]{0.4\linewidth}
        \Tree [.XP {X \\ the head} {YP \\ an argument} ]
        \subcaption{One argument}
        \label{fig:x-bar-structure-two}
    \end{minipage}
    \vfill
    \vspace{2em}
    \begin{minipage}[b]{0.4\linewidth}
        \Tree [.XP [.X' {YP \\ an argument} {X \\ the head } ] !\qsetw{2cm} {ZP \\ an argument} ]
        \subcaption{Two arguments}
        \label{fig:x-bar-structure-three}
    \end{minipage}
    \begin{minipage}[b]{0.4\linewidth}
        \Tree [.XP [.X' {ZP \\ an argument} !\qsetw{2cm} [.X' {X \\ the head} {YP \\ an argument} ] ] !\qsetw{2.2cm} {WP \\ an argument} ]
        \subcaption{Three arguments}
        \label{fig:x-bar-structure-four}
    \end{minipage}
    \caption{X-Bar structures}
    \label{fig:x-bar-structure}
\end{figure}

In tree diagrams of constituents we usually use XP to denote a phrase, 
and use X' to denote an intermediate projection.
This X-X'-\dots-X'-XP scheme is therefore called \concept{X-Bar scheme}, 
and notations X, X', XP, Y, Y', YP, \dots are called \concept{labels}.
X is called the \concept{head} of the X' or XP containing it.
Since features of all heads are clearly stored in the lexicon, 

\begin{figure}
    \centering
    \begin{minipage}[b]{0.4\linewidth}
        \Tree [.{$>$} $\theta$ [.$<$ $\alpha$ [.$<$ $\beta$ $\gamma$ ] ] ]
        \subcaption{A bare phrase structure}
        \label{fig:bps-example-origin}
    \end{minipage}
    \begin{minipage}[b]{0.4\linewidth}
        \Tree [.{$\alpha$} $\theta$ [.{$\alpha$} $\alpha$ [.{$\beta$} $\beta$ $\gamma$ ] ] ]
        \subcaption{\autoref{fig:bps-example-origin} with heads annotated}
    \end{minipage}
    \vfill
    \vspace{2em}
    \begin{minipage}[b]{0.4\linewidth}
        \Tree [.AP $\theta$ [.AP $\alpha$ [.BP $\beta$ $\gamma$ ] ] ]
        \subcaption{Corresponding X-Bar structure}
    \end{minipage}
    \caption{X-Bar structure and bare phrase structures}
    \label{fig:x-bar-and-bps}
\end{figure}

\subsection{Structure building operations}

Now we discuss basic structure building operations.
In \autoref{sec:syn-obj} we already learned that all operations must be triggered by certain features.
Since discrete features can be tricky to deal with in probabilistic grammars, 
in this section we will not consider which feature triggers which operation.
We will use flexible terms like ``$\alpha$ selects $\beta$'', which can be easily assigned a probabilistic meaning.

\concept{Phrasal merge} means to merge a word to a maximal projection, 
or to merge an intermediate projection to a maximal projection.
% TODO: left selection and right selection
\begin{equation}
    \pmergel(\alpha, \beta) = \begin{cases}
        \synbracket[<]{\alpha & \beta}, & \quad \text{if $\alpha$ is a word and $\beta$ is a phrase and $\alpha$ selects $\beta$}, \\
        \undefined, & \quad \text{otherwise}
    \end{cases}.
    \label{eq:pmergel-def}
\end{equation}

\concept{Head merge} means to merge two morphemes together to form a larger object,
which takes all features on the two morphemes except features driving the merge.
This means head merge may change the argument structure of a certain head.
For example, suppose $\alpha$ is of category $A$ and selects one argument of category $B$,
and can be optionally merged with $\beta$ with category $B$ which selects an argument of category $C$.
Head merging between $\alpha$ and $\beta$ creates $\alpha \beta$, which selects two arguments, 
of which one is of category $B$ and the other of category $C$.

\begin{figure}
    \centering
    \begin{minipage}[b]{0.4\linewidth}
        \Tree [.X Z [.X X Y ] ]
        \subcaption{A complex head}
    \end{minipage}
    \begin{minipage}[b]{0.4\linewidth}
        \Tree [.XP [.X' MP [.X Z [.X X Y ] ] ] !\qsetw{2cm} NP ]
        \subcaption{Complex head in a constituent}
    \end{minipage}
    \caption{Head merge}
    \label{fig:head-merge}
\end{figure}

\begin{equation}
    \copyl (\synbracket[\alpha]{\ldots \beta \ldots}) = \begin{cases}
        \synbracket[<]{\beta & \synbracket[\alpha]{\ldots \beta \ldots}} & \quad \text{if $\synbracket[\alpha]{\ldots \beta \ldots}$ selects $\beta$}, \\
        \undefined & \quad \text{otherwise}
    \end{cases}.
\end{equation}

\subsection{Probabilistic derivation}

\concept{Syntactic derivation} denotes the process of taking morphemes out of the lexicon (called \concept{numeration}), merging them into words and constituents, and transferring them into phonetic forms. 
In each stage of derivation, there is one or more syntactic object on building.
The set of all syntactic objects completely describe the state of the current state, so we call the set \concept{workspace}.
% Therefore, the derivation is a Markov process, and workspaces are states. 
% TODO: why do we never care about the time, or in other words, the number of steps?
% I guess I get the point. It is not a Markov process?!
% It is a Markov process in its stable status?

The concept of adjunction is completely absent in this section, because we do not view adjunction as a primitive operation in syntactic derivation. Rather, we take it as a secondary phenomenon emerging from % TODO

\section{Skeleton of sentences}

\subsection{Categories}

\subsection{The lexical structure}

% VP and light verb 

\subsection{The periphery hierarchy}

% TODO:topic, focus, force

\subsection{Adjunction and conjunction}

% TODO: [C A], [C B] -> [C [_&P A & B ]]
% For example:
% He gets a zero as his score and gets mad.
% [_vP he gets a zero as his score ] + [_vP he gets mad] -> [_TP He [_&P [gets a zero as his score] and [gets mad] ] ]

\end{document}