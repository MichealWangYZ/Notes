\documentclass[a4paper]{article}

\usepackage{geometry}
\usepackage{caption}
\usepackage{subcaption}
\usepackage{abstract}
\usepackage{amsmath, amssymb}
\usepackage{qtree}
\usepackage{gb4e}
\usepackage[colorlinks, linkcolor=black, anchorcolor=black, citecolor=black]{hyperref}
\usepackage{prettyref}

\geometry{left=3.18cm,right=3.18cm,top=2.54cm,bottom=2.54cm}

\DeclareMathOperator{\pmergel}{PMergeL}
\DeclareMathOperator{\pmerger}{PMergeR}
\DeclareMathOperator{\hmergel}{HMergeL}
\DeclareMathOperator{\hmerger}{HMergeR}
\DeclareMathOperator{\copyl}{CopyL}
\DeclareMathOperator{\copyr}{CopyR}
\DeclareMathOperator{\agree}{Agree}
\DeclareMathOperator{\move}{Move}
\DeclareMathOperator{\undefined}{undefined}
\setlength{\arraycolsep}{2pt}
\newcommand*{\synbracket}[2][{}]{[_\mathrm{#1} \; \begin{matrix} #2 \end{matrix} \; ]}

\newrefformat{fig}{Figure \ref{#1}}

\newcommand*{\concept}[1]{\underline{\textbf{#1}}}

\title{The Framework of Minimalist Syntax}
\author{wujinq}

\begin{document}

\maketitle

\begin{abstract}
    This article is a summary of Minimalist syntax in modern generative grammar.
    We review the mutually recognized primitives of the Minimalist Program, its main motivation, and its main achievements.
    We also present main disputes - both theoretical and empirical issues around this research program, and list some dominant frameworks trying to settle these issues.  
\end{abstract}

% TODO:
% successive cyclic
% head movement

\section{Introduction}

The tradition of the \concept{Principle and Parameter} (henceforward P\&P) approach has achieved huge success since 1980s. 
It, however, is definitely not the end of the generative enterprise, with too many principles and parameters required to cover all empirical observation, sometimes rendering P\&P simply curve-fitting of complicated phenomena instead of an explanatory theory.
If the goal of P\&P was to explore and explain the universal rules of all human languages, then the \concept{Minimalist Program} (henceforward MP) is focused on explore and explain P\&P.
What are obligatory? What can be explained using other concepts? 
We investigate the necessity of every concept used in P\&P, tring to find the real building blocks of human language.

That is why MP is called a \emph{research program} instead of a theory: there are, of course, something basic stipulations in MP that fall in the range of Chomskyan derivational syntax.
For example, there is Merge as the fundamental structure building operation, and Move is the origin of non-locality.
Besides these, however, over many issues no mutual consensus is achieved. We present some of these issues in this article and relevant solutions.

\section{The structure of grammar}

This section deals with syntactic objects, like morphemes, words, phrases, and the operations used to build them.
We are not to give a formalization of Minimalist syntax here so we will not use much technical terms, 
and often discussion about a certain concept will involves concepts yet to be defined.

\subsection{The Y-shaped model}

\begin{exe}
    \ex
    \gll Wenn jemand in die W\"uste zieht ... \\
    If someone in the desert draws and lives ... \\
    \trans `if one retreats to the desert and ... '
\end{exe}

\subsection{Purely syntactic operations}

\subsubsection{Merge: External and Internal}

\concept{Maximal projection}

\subsubsection{Agree}

% feature assignment

\subsubsection{Anything else?}

Merge, Move, Agree, and possibly Copy - do these form a exhaustive list of purely syntactic operations? 
The answer differs within different frameworks (See \prettyref{sec:framework} for some examples), and any possible additional operations are highly controversial (See \prettyref{sec:issue-op} for some discussion).
Here we only list some possible choices, reserving debate about their existence to \prettyref{sec:issue-op}.

Frequently proposed operations other than Merge, Move, Agree and Copy include:

\begin{itemize}
    \item \concept{Adjoin}, to attach an \concept{adjunct} - something optional and semantically interpreted differently with complements and specifiers - to a maximal projection. 
    \item \concept{Head movement}, to move a head to another head and adjoin these two heads, forming a complex head.
    \item \concept{Match}, to link two constituents' reference to the same object.
\end{itemize}

Some purely syntactic operations can be substituted by post-syntactic operation, and vice versa.

\subsection{Syntax-phonology interface and syntax-morphology interface}

\subsubsection{Transferring and phases}

\subsubsection{Post-syntactic operations}

It should be noted that whether post-syntactic operations really exist is still questionable. (See \prettyref{sec:post-syn-op} for some discussion)
This section only list some possible post-syntactic operations, which may be just phenomenological.

\subsection{Categories}

\subsection{Constraints}

\subsubsection{Constituent selection}

\subsubsection{What triggers a movement?}

\subsubsection{Cyclic-successive derivation}

A constituent should be moved to the edge of the current constituent.

\subsubsection{Locality and anti-locality of movements}

\section{Frameworks and programs within MP}\label{sec:framework}

\subsection{Cartography}

\subsection{Antisymmetry}

\subsubsection{The failiure of LCA, and a possible alternative}

\subsection{Distributed Morphology}

\subsection{Nanosyntax}

\section{Issues regarding operations}\label{sec:issue-op}

\subsection{Head movement}

% TODO: snowball roll-up

\subsubsection{Alternative: post-syntactic operations}

\subsubsection{Alternative: remnant movement}

\subsection{Adjunction}

\subsection{Post-syntactic operations}\label{sec:post-syn-op}

\subsection{The mechanism of Agree}

\subsection{Puzzles about the lexicon}

\section{Morphosyntax phenomena}

\subsection{Word formation, or what is a word}

\subsubsection{The Lexicalist Hypothesis and why it is wrong}

\subsection{Agreement and concord}

\subsection{Pronouns, reflexives and others}

\section{Sentence structures}

\subsection{Word order}

\subsection{Modifying and adjunction}

\subsection{Relative clauses}

\subsection{Asymmetry}

% more SOV, less VOS

\section{Conclusion: where we are and where to proceed}

% TODO:reference and bibtex

\end{document}